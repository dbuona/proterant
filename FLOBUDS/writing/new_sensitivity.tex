\documentclass[11pt]{article}
\renewcommand{\baselinestretch}{1.8}
\usepackage{textcomp}
\usepackage{fontenc}
\usepackage{graphicx}
\usepackage{caption} % for Fig. captions
\usepackage{gensymb} % for \degree
\usepackage{placeins} % for \images
\usepackage[margin=1in]{geometry} % to set margins
\usepackage{setspace}
\usepackage{lineno}
%\usepackage{cite}
\usepackage{amssymb} % for math symbols
\usepackage{amsmath} % for aligning equations
\usepackage[sort&amp;compress]{natbib}
\usepackage{xr-hyper}
\externaldocument{diffsens_SUPP}

\bibliographystyle{..//..//sub_projs/refs/styles/newphyto.bst}

\linenumbers


\title{Differences in flower and leaf bud responses to the environment drive shifts in spring phenological sequences of temperate woody plants}\\

\date{}
\author{D.M. Buonaiuto $^{1,2,a}$, E.M. Wolkovich$^{3}$}

\begin{document}
\maketitle
\section*{Introduction}
One of the most widely documented biological effects of anthropogenic climate change are shifts in phenology, the timing of life cycle events, in plants \citep{}. While phenology is generally advancing with climate change, the strength of these phenological shifts can vary substantially among specific phenological phases \citep{}. These differences alter the timing of phases relative to each other, changing the the duration of inter-phase periods that make up phenological sequences \citep{}. Phenological sequences are a major driver of plant fitness that impact plant life history, resource allocation, demography and ecosystem processes \citep{}. Shifts in phenological sequences will likely alter many of these processes, but the effects these shifts depend both on the direction (whether distinct phases are shifting closer together or farther apart) and magnitude (how much they are shifting relative to each other).\\ %EMWAug6: Nice!

Among deciduous woody plants, the relative timing of flower and leaf phenology, or flower-leaf sequences (FLSs), may be particularly consequential to fitness in temperate regions where flowering prior to leaf development is common \citep{Rathcke_1985,Gougherty2018}. Long-term phenological observations over the last several decades indicate that, like other phenological sequences, FLSs are shifting due to anthropogenic climate change \citep{Buonaiuto2020} suggesting that some of the critical functions of FLSs may become compromised. However, observed FLS shifts vary among species, which may put some species at greater risk while benefiting others \citep{Buonaiuto2020}.\\  %EMWAug6: May need to inject a clause in the first few sentences of this paragraph to help readers along ...  maybe change "Long-term phenological observations over the last several decades indicate that, like other phenological sequences, FLSs are shifting due to anthropogenic climate change \citep{Buonaiuto2020} suggesting that some of the critical functions of FLSs may become compromised." to Long-term phenological observations over the last several decades indicate that, like other phenological sequences, FLSs are shifting due to anthropogenic climate change \citep{Buonaiuto2020}---[quick clause here saying that the time between flowering and leafing is getting shorter or such (solidfy for readers what you're talking about when you say FLS)]. These changes could affect important functions of FLSs may become compromised.

For example, in wind-pollinated taxa, flowering before leaf development may be a critical adaptation for pollination efficiency by eliminating pollen interception by the forest canopy \citep{Whitehead1969}. In insect-pollinated taxa, flowering-first may increase the visibility of flowers to pollinators \citep{Janzen1967,Savage2019}. Species with decreasing FLS interphases with climate change may experience increased pollen limitation as more wind pollen is intercepted by vegetative structures and flowers are obscured by developing leaves. Conversely, pollination effeciency could improve for species with lengthening FLS interphases (direction). A change in the FLS interphase of just a few days would likely have little impact on these processes, but if shifts were on the order of weeks, the impact on the pollination biology of a species could be highly significant (magnitude).\\

Predicting the direction and magnitude of any FLS shifts requires identifying the underlying mechanisms that drive the different responses to climate change among these phenophases for a diversity of woody plant species. %There is mounting evidence that plants respond more strongly to different environmental cues during different parts of their annual cycle. Cues use may vary among phenophases across the season because reliability of specific cues to indicate appropriate conditions changes over the season as well \citep{}. For example, woody plants rely more strongly on photoperiod as a cue for autumn phenological phases than for spring phases indicating that daylength is a more reliable cue for the cessation of growth than the start of growth\citep{}. However, this hypothesis does not explain different cue use among phenophases like spring flowering and leafing that occur close to each other in time under relatively similar conditions \citep{}.\\
Decades of research suggests that for woody plants in temperate regions, cool winter temperatures (chilling), warm spring temperatures (forcing) and day-length (photoperiod) are the primary drivers of both reproductive and vegetative phenology \citep{Forrest2018,Flynn2018}. However, observed FLS shifts indicate that there must be differences in how these cues influence phenological activity in floral and leaf buds. % but these differences have not been well characterized in wild species. 
Identifying these differences is a necessary step for predicting the direction and magnitude, and ultimately fitness impacts of FLS shifts with climate change.\\
 
\noindent Studies that have attempted to identify the differences between reproductive and vegetative phenology in woody plants have mostly focused on crop species and two common, yet competing, findings have emerged:\\

\noindent What we call the \textbf{precocity hierarchy hypothesis (PHH)} suggests that reproductive and vegetative buds respond similarly to most environmental cues, but have consistently different forcing requirements for the commencement of phenological activity \citep{Guo_2014,COSMULESCU:2020aa,Cosmulescu:2018aa}. By contrast, what we call the \textbf{differential sensitivity hypothesis (DSH)} suggests that flower and leaf buds differ in the strength of their phenological responses to the multiple environmental cues \citep{Citadin2001,Gariglio2006,Crepinsek2011,Aslani2009,Mehlenbacher:1991aa}. \\%EMWAug6: I would name the hypotheses after you introduce them ... some possible edits also:
% What we call the \textbf{precocity hierarchy hypothesis (PHH)} suggests that reproductive and vegetative buds respond similarly to most environmental cues, but have consistently different forcing requirements for the commencement of phenological activity --> One hypothesis suggests that reproductive and vegetative buds have the same underlying cues to the environment, but have different threshold responses to forcing, with whichever budtype bursts later---leaves or flowers---having a higher threshold. Under this hypothesis, which we call the precocity hierarchy hypothesis (PHH), leaf and flower buds share the same suite of cues and develop similarly to non-forcing cues but they differ in the thermal units required for budburst. [And then similar edits to the next part of the paragraph.]

\noindent While these mechanisms may produce similar phenological patterns under historic climate conditions, they have different implications regarding the potential for FLS shifts with climate change. The PHH suggests that FLS variation is largely a product of climate variation during the interphase. If spring temperatures increase with climate change, the second phenophase of the FLS with be accelerated relative to the first and the FLS interphases will decrease, but given the relative auto-correlation of spring temperatures \citep{}, these shifts should be relatively muted. \\

\noindent The DSH suggests that with significant cue use differences among bud types, there will be strongly localized effects of climate change on FLSs. While on average the climate is warming, chilling and forcing may increase or decrease at different locations and on different time scales \citep{Ettinger}. Shifts in FLS variation will depend on the direction and rate of change in cues at specific locations and the differential sensitivity of reproductive and vegetative phenology to cue combinations. This hypothesis allows not only for larger magnitude shift in FLS, it also suggest that the magnitude of shifts may be highly divergent among populations of the same species.\\

%\noindent Simple simulations suggest that each mechanism will produce different, recognizable signatures for phenological patterns under experimental conditions. For the precocity hierarchy \\

\noindent In this study we test these hypotheses by comparing the phenological response to changing environmental conditions between flower and leaf buds for a suite of temperate shrubs and trees. We leverage these data to to make generalized projections for how FLSs may shift with climate change and discuss these shifts may affect plant function in the future.\\ 



\end{document}