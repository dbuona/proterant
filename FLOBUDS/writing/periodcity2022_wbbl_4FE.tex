\documentclass[11pt]{article}
%Required: You must have these
\usepackage{graphicx}
\usepackage{tabularx}
\usepackage{natbib}
\usepackage{array}
\usepackage{amsmath}
\setkeys{Gin}{width=0.8\textwidth}
\setlength{\abovecaptionskip}{10pt}
\setlength{\belowcaptionskip}{10pt}
 \topmargin -1.5cm 
 \oddsidemargin -0.04cm 
 \evensidemargin -0.04cm 
 \textwidth 16.59cm
 \textheight 23.94cm 
 \parskip 7.2pt 
\renewcommand{\baselinestretch}{1.2} 	
\parindent 0pt
\usepackage{lineno}
\linenumbers

\bibliographystyle{..//refs/styles/besjournals.bst}
\usepackage{xr-hyper}
\usepackage{hyperref}


\externaldocument{SUPPperiodicity2022}
\externaldocument{FE_rev_resp_partII}

\title{Experimental designs for testing the interactive effects of temperature and light in ecology: the problem of periodicity }
\date{}
\author{D.M. Buonaiuto $^{1,2,3,a}$, M. Donahue$^{4}$, E.M. Wolkovich$^{2,3,5}$}

\begin{document}
\section*{Abstract}
\begin{enumerate}
\item Temperature and light cues interact to control many biological processes. Experiments give researchers the ability to manipulate these environmental cues independently, and can be designed to robustly quantify their individual and interactive effects on any particular biological activity. Such experiments have produced important insights into the environmental controls on numerous biological processes in both plant and animal taxa across terrestrial and aquatic environments. Testing the interactive effects of multiple environmental cues, however, requires experimental treatments to be fully independent; any unmeasured experimental covariation among treatments can result in incorrect conclusions.

\item  Using a database of controlled environment experiments on the spring phenology of woody plants as a case study, we highlight how a common experimental set-up, designed to parse the interactive effects of temperature and photoperiod on time to budburst, introduces a latent experimental covariation of these treatments by coupling photo- and thermo- periodicity. Using data simulations, algebraic corrections and a comparative analysis of published experiments, we demonstrate how this unmeasured experimental covariation biases statistical inference regarding the relative contribution of light and temperature cues to phenological variation.

\item We identify this experimental covariation in more than 40\% of published phenology studies that manipulate photoperiod. Our analyses demonstrate that the coupling of thermo- and photo- periodicity results in the overestimation of the effect of photoperiod, the underestimation of forcing effects, and misleading conclusions about their interactions on phenology. This may, in part, explain why the significance of photoperiod cues for spring phenology is currently debated in the literature.

\item Accurate forecasting of how varying environmental conditions will impact the dynamics of biological events requires accurately quantifying cue responses. To this end, we present\linelabel{lowcost6} several options for statistical corrections and alternative experimental designs that can provide more robust estimates of the relative effects of temperature and photoperiod on phenology, and many other biological processes controlled by temperature and light.
%284 words
\end{enumerate}

\textbf{Keywords:}  forcing, full-factorial,  growth chamber, light,  phenology, photoperiod, temperature, thermoperiod \\
\pagebreak
\section*{Introduction}

\noindent Across the tree of life, temperature and light availability shape a number of important biological processes including growth and metabolic rates \citep{MacLean:2019aa}, sex determination \citep{Brown:2014vn}, acclimatization to seasonal environments \citep{Hamilton2016} and the timing of life cycle transitions \citep[i.e., phenology,][]{Forrest2010}. These biological responses in turn dictate broad-scale ecological processes and patterns ranging from biogeochemical cycling \citep{Piao2007} to species range limits \citep{Chuine2001}. Characterizing the specific dynamics of how these environmental factors synergistically affect biological processes across a wide range of taxa has become even more important as anthropogenic global change continues to expose organisms to novel environmental conditions \citep{Portner:2008vd}.

\noindent Because temperature and light availability often co-vary in the field \citep[for example, in most temperate ecosystems, daylength and temperature both increase as the season progresses,][]{Rosenberg1974}, it can be difficult to disentangle their relative contributions to biological processes. In contrast, experimental manipulations of climate variables in artificial environments can mechanistically characterize biological responses to environmental fluctuations \citep{Ettinger:2020aa,Primack2015}. Researchers have used controlled environments of all shapes and sizes to this end \citep{Downs:1980us}; these efforts have greatly advanced our collective understanding of the fundamental biology of a wide variety of organisms and ability to predict ecological and evolutionary responses to current and future climate change \citep{Stewart:2013wz}.

However, controlled environment experiments have their own challenges. Experimentalists must balance biological realism with robust inference, experimental effort with statistical power, and account for the effects of unmanipulated or unmeasured variables \citep{schneiner2001}. Because biological responses to the environment are generally the product of complex interactions between multiple environmental signals \citep{Casal:2002vz}, seemingly small choices about experimental designs can generate significant differences in outcomes. Experimental treatments are rarely standardized among researchers, even within disciplines \citep{Wolkovich:2022ue}, and these complexities may in part contribute to the many discrepancies between experimental studies and observational data \citep{Poorter:2016aa}. Even with these limitations, controlled environment studies remain a powerful tool to mechanistically assess organismic responses to the environment, provided that the implications of treatment designs are well understood and well matched with the scope of the research question. 

As technology advances and experiments become more complex, researchers can manipulate more variables and multiple axes of variation (e.g., temperature, amplitude, periodicity, wavelength) at the same time. Yet these efforts may present a tradeoff between biological realism and robust inference. Through investigating the literature on experiments with plant phenology, we show that experiments that manipulate both photo- and thermo- periodicity often introduce a latent experimental covariation between light and temperature treatments, which may misrepresent the effects of each of these environmental variables and the interaction between them. We begin by briefly detailing how temperature and light treatments are generally applied in phenology experiments and review the minimum experimental elements required to robustly test interactions between two or more environmental variables. We then detail the problem of inference that can arise when manipulating the periodicity of both temperature and light in experiments, and demonstrate the extent to which this is an issue through data simulations, a mathematical correction, and a comparative analysis of published experiments. Finally, we conclude by outlining methods\linelabel{lowcost5} that correct for this issue, along with alternative experimental designs that can overcome the problem of periodicity. 

While our case study deals with phenology of temperate woody plants, it provides insights into a number of other systems with parallel issues. Studies of aquatic algae \citep{XU2019167}, insects \citep{ANDUAGA201846}, amphibians \citep{WRIGHT200433} and fish \citep{Olemda_2009} have similarly struggled to disentangle the effects of thermo- and photo- periodicity.  Thus, we believe the potential problems and solutions we present here are broadly applicable to studies on other organisms and biological processes that utilize temperature and light signals. 

\section*{Case study: Estimating phenological cues from experiments}
Decades of experimental work in controlled environments have demonstrated that temperature (both cool temperatures in fall/winter and warming temperatures in spring) and photoperiod are the primary phenological cues for plants in the temperate/boreal zones \citep{Ettinger:2020aa}. While exposure to cool winter temperatures (chilling) strongly impacts phenology \citep{Laube2014}, we focus here on warm temperature and light treatments, because controlled chilling treatments with light are uncommon \citep{Wolkovich:2022ue}. Choices about how to apply warm temperature and light treatments, in particular, can compromise inference on their effects, so we focus on these two cues.

While a large variety of experimental designs have been used to study plant phenology, generally experiments tend to manipulate two major axes of light and warm temperature variation:
\begin{enumerate}
\item \underline{Intensity:} The amount or quality of a variable. Here we define temperature intensity as the amount of heat present in the system (measured in degrees). In the phenology literature this measurement is generally referred to as forcing. We define light intensity as the luminosity or irradiance present in the system (measured in lumens or watts). 
\item \underline{Periodicity:} The interval at which the intensity of the variable is applied. Hereafter, we refer to the periodicity of light as photoperiod (often used synonymously with ``daylength") and the periodicity of temperature as thermoperiod. 
\end{enumerate}
For phenology, photoperiodicity is generally considered the primary light cue for plants \citep{WAY:2015aa}, \citep[though regarding light intensity and phenology see][]{Brelsford2018,Cober1996}. For temperature, conventionally both intensity and periodicity drive phenological activity and several metrics (e.g. growing degree hours, thermal sums, growing degree days)  that combine these two axes have been developed \citep{Gu:2016wa}. The importance of thermo-intensity and periodicity is well supported; under natural conditions diurnal temperature fluctuations in temperate regions can be quite large in the spring, and studies have found that diurnal temperature variation strongly influences plant phenology \citep{Burghardt:2016uy}. In fact, even if thermoperiodicity is not an explicit treatment variable (i.e., manipulated systematically), incorporating it in experiments can be essential for translating experimental results into real world predictions \citep{plants9101312}.

Like many other biological processes, recent advances have demonstrated that plant phenological responses are nonlinear, due largely to interactions between cues \citep{Wolkovich:2022ue,fu2015}, highlighting the need for experiments designed to evaluate the strength of these interactions. To have the statistical power to partition the individual and interactive effects of two or more variables, an experiment must:
\begin{enumerate}
\item Have a minimum of two treatment levels of at least two variables.
\item Treatment levels must be full factorial (Fig. \ref{fig:examp}a.). Full factorial designs are both balanced (Fig. \ref{fig:examp}b.)  and orthogonal (Fig. \ref{fig:examp}c.); meaning that all possible treatment combinations are applied and each treatment is independent of all others \citep{cheng2016}.
\end{enumerate}

These two critical elements may seem obvious but are conspicuously absent from many published studies. Of the 136 studies contained in a published database of woody plant phenological experiments \citep[OSPREE: Observed Spring Phenological Responses in Experimental Environments,][]{wolkovich2019}, a recent\linelabel{ospree1} study by \citet{Wolkovich:2022ue} found that only 37\% of the studies that manipulated more than one variable did so with a design that was both balanced and orthogonal. But even experiments that are designed to be full factorial frequently violate the assumption of orthogonality when both photo- and thermo- periodicity are built into experiments. We detail this problem below.

\section*{The problem of periodicity}
A common approach in phenology experiments that seems to balance prior knowledge about the underlying physiology of phenology, biological realism and experimental inference is to vary photoperiodicity, and thermal intensity and periodicity \citep[e.g.,][]{Flynn2018,Sanz-Perez:2009aa,Basler:2014aa}. This\linelabel{leveler1} design could include any number of treatment levels for each variable (e.g., 8, 12, 16...\emph{n} hours of photoperiod and 20/10^{\circ}$C, 22/12^{\circ}$C, 25/15^{\circ}$C...$n_{day}/n_{night}$ day/night temperatures). To consider a simple example, we use a hypothetical experiment with two treatment levels of each variable. Consider a basic experiment that includes, at minimum, a long (16 hours) and short (8 hours) photoperiod treatment and a high (25/15$^{\circ}$C day/night) and low (20/10$^{\circ}$C day/night) forcing treatment. In this case, the thermoperiodicity is not an explicit treatment (both high and low temperature treatments use a diurnal fluctuation of 10 $^{\circ}$C), and is simply incorporated in the design to enhance biological realism. At first glance, this design appears to meet the criteria of a full factorial design, multiple treatment levels that are balanced and orthogonal, with high/low temperature treatments (mean 20$^{\circ}$C and 15$^{\circ}$C respectively) and long/short photoperiod treatments applied in all possible combinations.

Yet the orthogonality of this design is based on the assumption of a 12 hour thermoperiod. If, rather the thermoperiod is coupled with the photoperiod, the temperature treatment is non-orthogonal because the daily mean temperature of the long/high treatment will be higher than that of the short/high treatment, and the long/low treatment slightly warmer than the short/low. We refer to this experimental set-up as a coupled design (i.e. thermoperiod and photoperiod are coupled with each other). 

Coupled designs introduce an experimental covariation between photoperiod and forcing treatments. This experimental covariation is clearly illustrated when temperature treatment levels are converted to thermals sums. We calculate thermal sums (also called growing degree hours), by multiplying hourly temperatures above a certain base temperature threshold by the number of hours for which they are applied over a 24 hour period \citep{Parent:2019ug}. For example, given a base temperature of 0$^{\circ}$C, a low forcing treatment of 20/10$^{\circ}$C day/night accrues 400 thermal units per 24 hours  when crossed with the long (16 hour) photoperiod treatment and only 320 thermal units when crossed with the short (8 hour) photoperiod treatment. While this experimental covariation among the photoperiod and temperature treatments is biologically realistic, it makes it statistically impossible to differentiate the independent and interactive effects of temperature and photoperiod on any given biological process.

This problem of inference that arises from the experimental covariation of thermo- and photo- periodicity is not limited only to studies seeking to directly compare the effects of photoperiod and forcing; it applies in any study evaluating the influence of photoperiod on biological activity, even if it is the only manipulated cue. Experimentally isolating the effect of photoperiod assumes that all other environmental variables are held constant.  Similar to the case described above, the coupling of photoperiod and thermoperiod in an experiment where forcing is intended to be a consistent, background condition (e.g., two or more levels of photoperiod treatments\linelabel{leveler2} (e.g. 8, 12, and 16 hours), all at a background temperature of 20/10$^{\circ}$C day/night) would yield a situation in which longer photoperiod treatments were also receiving more---unmeasured---heating than shorter photoperiod treatments. In this case, some amount of the perceived photoperiod effect is due to the latent, increased forcing, and the experiment will not isolate the true effect of photoperiod.

We queried the OSPREE database to identify experiments that applied different day and night temperatures in their studies without designating diurnal temperature variation as an explicit experimental treatment. Of the 51 experiments in the OSPREE database that manipulated photoperiod experimentally, up to 43\% of them appear to include an experimental covariation with thermoperiod. Of the 18 studies that manipulated both photoperiod and temperature interactively, we found that up to 55\% of them appear to have this issue, suggesting that the true interactive effects of these cues on spring phenology is quite poorly characterized. This may be in part why the relative contribution of temperature and photoperiod cues to spring phenology remains a contentious debate in the phenology literature \citep{koerner2010a,CHUINE:2010wg,Jennifer:2010un}.

\section*{Periodicity and inference}
If the lack of orthogonality introduced to experiments when photoperiod and thermoperiod are coupled is overlooked, regression models will always overestimate the photoperiod effect and underestimate the forcing effect (Fig. \ref{fig:3d}a,b.). This\linelabel{reason1} is because forcing is the variable with latent, unmeasured variation. In the case of phenology, this is particularly significant because studies repeatedly suggest that forcing is a more dominant cue than photoperiod for spring phenology \citep{CHUINE:2010wg,Zohner:2016uz,Gauzere2019}. The influence\linelabel{leveler3} of this experimental covariation of periodicity on generating incorrect estimates of temperate and photoperiod cue effect-sizes pervades experiments with any number of treatment levels (see Fig. \ref{fig:suplines} for an example with three treatment levels of forcing and photoperiod), and may be even more difficult to identify as experimental complexity increases.

If experiments are designed to quantify the interaction between photoperiod and forcing, here too, the experimental covariation of periodicity will result in an erroneous estimation of the interaction. (Fig. \ref{fig:3d}c,d.). Our simulation depicts a particularly troublesome case where a true sub-additive interaction is interpreted as a supra-additive one (Fig. \ref{fig:3d}c,d.), however, other outcomes are possible. Experimental covariation of light and temperature treatments due to coupling thermo- and photo- periodicity will generally result in the incorrect estimation of the interaction term, but the exact nature of this statistical issue depends on the sign and strength of the interaction. 

We can attempt to estimate how much of a photoperiod effect is due to forcing in experiments where they covary by making several major assumptions. First we assume that forcing and photoperiod effects are additive and linear (i.e., there is no interaction). While this may not be true in nature, it gives us insight into the potential effect of the experimental covariation of periodicity by allowing us to solve algebraically for the separate effects of forcing and photoperiod. We replace the qualitative factor (high/low forcing) by the quantitative effect of forcing (thermal sums) to properly account for the difference in forcing between short and long photoperiods (see Supporting Information: Estimating the effects of experimental periodicity covariance mathematically). Using the data from one experiment that experimentally coupled thermo- and photo-period, \citet{Flynn2018}, we found that 33\% of the published photoperiod effect of 4.5 days could be due to forcing. 

Our algebraic solution cannot be as readily applied in experiments that assume photoperiod and forcing interact. However, we can generally assess the scope of the problem of inference due to experimental covariation of periodicity by comparing studies that used a coupled design to those with alternative approaches. While we are aware of no experiments that explicitly compare the effects of experimentally coupling vs. uncoupling photo- and thermo- periods, we identified two phenology experiments that utilized many overlapping treatment levels and species from the same sampling sites, however in one study, \citet{Flynn2018}, photo- and thermo- period experimentally co-vary, while in the other, \citet{Buonaiuto:2021ug},  photo- and thermo- period were varied independently (see Supporting Information: Modeling Methods\linelabel{common3} for details on treatment similarities and differences between the studies). Comparing the cue estimates from these two studies offers an opportunity to test our theoretical and mathematical predictions, and further understand the uncertainty in cue estimates due to coupled periodicities.

We subset each dataset to include only the species shared among the two studies, and re-analyzed the data using Bayesian hierarchical models to compare the difference in the photoperiod, forcing and interaction estimates (see Supporting Information: Modeling Methods). We found that the estimated differences in the mean response to photoperiod and forcing and their interactions among study designs were on the same order as our predictions above for misestimated cue effects due to experimental covariation between light and temperature treatments. We estimated a substantially weaker (less negative) photoperiod effect, and marginally stronger forcing effect for the uncoupled vs. coupled experimental design (Fig. \ref{fig:compy}). The interaction term we estimated for the uncoupled design was negative, suggesting the interaction between photoperiod and forcing is supra-additive, while the estimated interactive effect from the coupled design was sub-additive (Fig. \ref{fig:compy}).

Unlike in our simulations (Fig. \ref{fig:3d}), in this comparison we cannot assess what the ``true" effects of these variables are. There are almost certainly other factors driving the differences between these experiments. Both were conducted in different years, sampled different individuals from the population, and used different methods for applying chilling pre-treatments \citep{Flynn2018,Buonaiuto:2021ug}. However, because this comparison is well matched to our predictions and prior knowledge about how temperature and photoperiod are expected to interacting in phenology, we argue that the influence of experimental covariation on statistical inference is apparent enough to take seriously.\\

\section*{Paths Forward}
We have demonstrated that experiments that coupled thermoperiod and photoperiod cannot robustly differentiate the individual or interactive effects of temperature and photoperiod on spring phenology (or any other biological process) due to an unmeasured experimental covariation among temperature and light treatments. Given the paucity of interactive studies in the literature, it is clear that we need more well designed studies to better characterize the effects of these cues. At\linelabel{lowcost1} the same time, there are straight-forward statistical approaches for accounting for this experimental covariation that can be adopted immediately and will rapidly improve scope of inference possible with controlled environment experiments. Below we detail these approaches, and then offer several generalized experimental designs that improve statistical orthogonality of controlled environment experiments, which could be further developed and adjusted to fit the needs of experimentalists across many sub-fields of ecology and evolutionary\linelabel{lowcost2} biology. 

\subsection*{Effect-size inference and statistical corrections:}
It\linelabel{lowcost3} may be that the experimental design that best balances environmental realism, statistical inference and translatability to observational studies are designs that continue to couple periodicity to mimic natural systems. Fundamentally, simply recognizing the issues that arise when thermo- and photoperiods are experimentally co-varied and accounting for this in interpreting effect-sizes and reporting uncertainty is a powerful start for improving inference from experiments. This awareness can be applied both forward and backward: to future experiments that seek to understand the interactive effects of temperature and photoperiod and to synthesizing and interpreting the near-century's worth of research in this area that has already been published.

The growing awareness of this issue has prompted the development of several simple statistical corrections to deal with this experimental covariation. Some recent studies adjust temperature effect-sizes in their statistical models by daylength treatments \citep[e.g.,][]{Ettinger:2020aa}. For example, if a day/night forcing treatment of 25/15$^{\circ}$C was applied in conjunction with 8 and 16 hour photoperiod treatment levels, the mean daily temperature of the forcing level can be weighted by hours for which it was applied, in this case resulting in forcing treatments with mean temperatures of 18.3$^{\circ}$C and 21.6$^{\circ}$C respectively. This approach does not remove the covariation between temperature and photoperiod (i.e., the higher photoperiod treatment is still getting more heat that the lower photoperiod treatment, e.g, Fig. \ref{fig:examp}b), but this covariation is no longer latent, and can be accounted for in a regression model. This simple approach could be adopted by any experimentalist and will substantially increase the utility of such experiments for\linelabel{lowcost4} ecological forecasting.

\subsection*{Experimental Re-designs:}
For researchers interested in taking on this problem of periodicity head-on in their experiments, there are several experimental designs that can either eliminate the problem of experimental covariation of photo- and thermo-period entirely, or more robustly address it at the experimental stage. Below,\linelabel{multiexamp1} we provide general details about alternative experimental designs with representative examples in Fig. \ref{fig:designs}. As in our previous examples, here we depict experiments with two levels of two experimental variables (photoperiod and temperature), but importantly, these generalized schemes can be readily adapted for experiments with any number of treatment levels number of variables, as long as they are full-factorial. Further, these designs could be adapted for any experimental variables for which both intensity and periodicity can be manipulated\linelabel{multiexamp2} (e.g. light, humidity, heat or freeze shock etc).
\begin{enumerate}

\item \textbf{Manipulate photoperiod and temperature intensity with no thermoperiodicity}. The simplest way to evaluate the individual and combined effects of temperature and photoperiod in experimental settings is to remove thermoperiodicity from studies entirely by maintaining constant day/night temperatures within temperature treatments (Fig. \ref{fig:designs}a.). This approach allows for the maintenance of statistical orthogonality across treatment combinations. The main drawback is that this design sacrifices the realism of diurnal temperature variation, which may make it more difficult to translate estimates from experiments to real-world applications. However, many aspects of physiology\linelabel{clar1} and development do not appear to respond explicitly to diurnal temperature variation \citep[e.g.,][]{Hellmers:1966wz,Warrington:1977vt,Bhatt:2019uu}, so in many cases this experimental simplification may be worthwhile to improve inference on the overall individual and combined effect of temperature and photoperiod on biological processes.

\item \textbf{Uncouple thermoperiod and photoperiod}. By varying thermoperiod and photoperiod independently, statistical orthogonality can be maintained across treatments. For example, a study could apply photoperiod treatment levels of 8 vs. 16 hour day/night with a 12 hour thermoperiod regime across temperature treatments (Fig. \ref{fig:designs}b.). While this approach allows for more robust evaluation of cue effects and interactions, uncoupling photoperiod and thermoperiod can require newer and more expensive technologies which many not be widely available. Further, this approach may also introduce new artifacts that occur from the biological rather than statistical interactions between light and temperature \citep{Chew:2012wj}. There is evidence that increasing temperatures in the first two hours of daylight can be almost as effective for stimulating shoot elongation as similar temperature increases for the whole photoperiod \citep{Erwin1998}. With this design, treatments must inherently differ in the amount of time the warmer daytime temperature extends into the dark, nighttime light regime (or vice versa), introducing a new axis of non-orthogonality to consider.

\item \textbf{Include thermoperiodicity as an explicit experimental treatment}. 
In many study systems in which both photoperiod and thermoperiod influence biological processes, experimentalists often include thermoperiodicity as an explicit experimental treatment with both constant and varying day/night temperatures applied as separate treatment variables \citep[e.g.,][]{Zaslavksi_1995}. Such experiments\linelabel{why1} are the best way to assess the comparative importance of temperature intensity and periodicity, which would provide important insights towards parsing the relative strength of temperature and light cues and their interactions,  (Fig. \ref{fig:designs}c.). However, executing such an experiment with a full-factorial design would substantially increase the size of a study and its experimental effort and, given the availability of the statistical tools and simpler experimental designs we discuss above, may only be worthwhile when researchers are explicitly interested in the relative contributions of periodicity and intensity in a particular study system\linelabel{why2}.
\end{enumerate}

Any\linelabel{pointy1} of these designs can be implemented for any number of treatment levels, and incorporate additional manipulated variables. This is particularly important in the context of global change because shifts in the magnitude, and even direction, of climate change will vary spatially, and in some systems, two level studies may struggle to estimate interactions due to under-sampled treatment levels \citep{Collins:2022ve}. Researchers seeking to capture such environmental dynamics could adapt these design to include at least a ``control", ``increase" and ``decrease" treatment level for experimental variables. Additionally, with increasing numbers of treatments and treatment levels, full-factorial experiments can be implement with a \emph{response surface methodology}, \citep[e.g,][]{SCHUBERT200984,Begoude:2007td}; a powerful approach for detecting interactions and non-linearities among variables. Experimentalists should leverage their knowledge of the natural history of their study organisms, historical climate observations and climate change projections to determine the treatment levels that are most appropriate for their study\linelabel{pointy2}.

In correcting one problem, each of the designs that we outlined above introduces another, which may in fact be an intrinsic property of any experimental manipulation. 

It would be useful for researchers to explicitly test how cue estimates vary among experimental designs, and which design is most useful for predicting biological responses to environmental cues in the field under current and future climate conditions. In the meantime, we hope that this issue is a reminder that, as experimentalists, we must continue to be thoughtful about matching our experimental designs to the goals of a study, and be transparent about uncertainty around our experimental inference.

%\bibliography{..///refs/periodicity.bib}
\begin{thebibliography}{49}
\providecommand{\natexlab}[1]{#1}

\bibitem[{Anduaga \emph{et~al.}(2018)Anduaga, Nagy, Costa \&
  Kyriacou}]{ANDUAGA201846}
Anduaga, A.M., Nagy, D., Costa, R. \& Kyriacou, C.P. (2018) Diapause in
  drosophila melanogaster -- photoperiodicity, cold tolerance and metabolites.
  \emph{Journal of Insect Physiology} \textbf{105}, 46--53.

\bibitem[{Basler \& K{\"o}rner(2014)}]{Basler:2014aa}
Basler, D. \& K{\"o}rner, C. (2014) Photoperiod and temperature responses of
  bud swelling and bud burst in four temperate forest tree species. \emph{Tree
  physiology} \textbf{34}, 377--388.

\bibitem[{Begoude \emph{et~al.}(2007)Begoude, Lahlali, Friel, Tondje \&
  Jijakli}]{Begoude:2007td}
Begoude, B.A.D., Lahlali, R., Friel, D., Tondje, P.R. \& Jijakli, M.H. (2007)
  Response surface methodology study of the combined effects of temperature,
  ph, and aw on the growth rate of trichoderma asperellum. \emph{Journal of
  Applied Microbiology} \textbf{103}, 845--854.

\bibitem[{Bhatt \emph{et~al.}(2019)Bhatt, Bhat, Suleiman \&
  Santo}]{Bhatt:2019uu}
Bhatt, A., Bhat, N.R., Suleiman, M.K. \& Santo, A. (2019) Effects of storage,
  mucilage presence, photoperiod, thermoperiod and salinity on germination of
  farsetia aegyptia turra (brassicaceae) seeds: implications for restoration
  and seed banks in arabian desert. \emph{Plant Biosystems - An International
  Journal Dealing with all Aspects of Plant Biology} \textbf{153}, 280--287.

\bibitem[{Brelsford \& Robson({2018})}]{Brelsford2018}
Brelsford, C.C. \& Robson, T.M. ({2018}) {Blue light advances bud burst in
  branches of three deciduous tree species under short-day conditions}.
  \emph{{Trees-Structure and Function}} \textbf{{32}}, {1157--1164}.

\bibitem[{Brown \emph{et~al.}(2014)Brown, Baumann \& Conover}]{Brown:2014vn}
Brown, E.E., Baumann, H. \& Conover, D.O. (2014) Temperature and photoperiod
  effects on sex determination in a fish. \emph{Journal of Experimental Marine
  Biology and Ecology} \textbf{461}, 39--43.

\bibitem[{Buonaiuto \& Wolkovich(2021)}]{Buonaiuto:2021ug}
Buonaiuto, D.M. \& Wolkovich, E.M. (2021) Differences between flower and leaf
  phenological responses to environmental variation drive shifts in spring
  phenological sequences of temperate woody plants. \emph{Journal of Ecology}
  \textbf{109}, 2922--2933.

\bibitem[{Burghardt \emph{et~al.}(2016)Burghardt, Runcie, Wilczek, Cooper, Roe,
  Welch \& Schmitt}]{Burghardt:2016uy}
Burghardt, L.T., Runcie, D.E., Wilczek, A.M., Cooper, M.D., Roe, J.L., Welch,
  S.M. \& Schmitt, J. (2016) Fluctuating, warm temperatures decrease the effect
  of a key floral repressor on flowering time in arabidopsis thaliana.
  \emph{New Phytologist} \textbf{210}, 564--576.

\bibitem[{Casal(2002)}]{Casal:2002vz}
Casal, J.J. (2002) Environmental cues affecting development. \emph{Current
  Opinion in Plant Biology} \textbf{5}, 37--42.

\bibitem[{Cheng(2016)}]{cheng2016}
Cheng, C.S. (2016) \emph{Theory of Factorial Design}. Chapman and Hall/CRC.

\bibitem[{Chew \emph{et~al.}(2012)Chew, Wilczek, Williams, Welch, Schmitt \&
  Halliday}]{Chew:2012wj}
Chew, Y.H., Wilczek, A.M., Williams, M., Welch, S.M., Schmitt, J. \& Halliday,
  K.J. (2012) An augmented arabidopsis phenology model reveals seasonal
  temperature control of flowering time. \emph{New Phytologist} \textbf{194},
  654--665.

\bibitem[{Chiang \emph{et~al.}(2020)Chiang, B{\aa}nkestad \&
  Hoch}]{plants9101312}
Chiang, C., B{\aa}nkestad, D. \& Hoch, G. (2020) Reaching natural growth: The
  significance of light and temperature fluctuations in plant performance in
  indoor growth facilities. \emph{Plants} \textbf{9}.

\bibitem[{Chuine \& Beaubien({2001})}]{Chuine2001}
Chuine, I. \& Beaubien, E. ({2001}) {Phenology is a major determinant of tree
  species range}. \emph{Ecology Letters} \textbf{{4}}, {500--510}.

\bibitem[{Chuine \emph{et~al.}(2010)Chuine, Morin, Bugmann, Koerner \&
  Basler}]{CHUINE:2010wg}
Chuine, I., Morin, X., Bugmann, H., Koerner, C. \& Basler, D. (2010) Warming,
  photoperiods, and tree phenology. \emph{Science} \textbf{329}, 277--278.

\bibitem[{Cober \emph{et~al.}({1996})Cober, Tanner \& Voldeng}]{Cober1996}
Cober, E., Tanner, J. \& Voldeng, H. ({1996}) {Soybean photoperiod-sensitivity
  loci respond differentially to light quality}. \emph{{Crop Science}}
  \textbf{{36}}, {606--610}.

\bibitem[{Collins \emph{et~al.}(2022)Collins, Whittaker \&
  Thomas}]{Collins:2022ve}
Collins, S., Whittaker, H. \& Thomas, M.K. (2022) The need for unrealistic
  experiments in global change biology. \emph{Current Opinion in Microbiology}
  \textbf{68}, 102151.

\bibitem[{Downs(1980)}]{Downs:1980us}
Downs, R.J. (1980) Phytotrons. \emph{Botanical Review} \textbf{46}, 447--489.

\bibitem[{Erwin({1998})}]{Erwin1998}
Erwin, J. ({1998}) {Temperature and light effects on stem elongation}.
  \emph{{Journal of the Japanses Society For Horticultural Science}}
  \textbf{{67}}, {1113--1120}, {Session of Plant Growth Regulation by Physical
  and Mechanical Stimuli at the Commemorative Symposium of the 75th Anniversary
  of Japanese-Society-for-Horticultural-Science, JAPAN, APR 03-04, 1998}.

\bibitem[{Ettinger \emph{et~al.}(2020)Ettinger, Chamberlain, Morales-Castilla,
  Buonaiuto, Flynn, Savas, Samaha \& Wolkovich}]{Ettinger:2020aa}
Ettinger, A.K., Chamberlain, C.J., Morales-Castilla, I., Buonaiuto, D.M.,
  Flynn, D.F.B., Savas, T., Samaha, J.A. \& Wolkovich, E.M. (2020) Winter
  temperatures predominate in spring phenological responses to warming.
  \emph{Nature Climate Change} \textbf{10}, 1137--1142.

\bibitem[{Flynn \& Wolkovich(2018)}]{Flynn2018}
Flynn, D.F.B. \& Wolkovich, E.M. (2018) Temperature and photoperiod drive
  spring phenology across all species in a temperate forest community.
  \emph{New Phytologist} \textbf{219}, 1353--1362.

\bibitem[{Forrest \& Miller-Rushing(2010)}]{Forrest2010}
Forrest, J. \& Miller-Rushing, A.J. (2010) Toward a synthetic understanding of
  the role of phenology in ecology and evolution. \emph{Philosophical
  Transactions of the Royal Society of London B: Biological Sciences}
  \textbf{365}, 3101--3112.

\bibitem[{Fu \emph{et~al.}(2015)Fu, Zhao, Piao, Peaucelle, Peng, Zhou, Ciais,
  Huang, Menzel, Uelas, Song, Vitasse, Zeng \& Janssens}]{fu2015}
Fu, Y.S.H., Zhao, H.F., Piao, S.L., Peaucelle, M., Peng, S.S., Zhou, G.Y.,
  Ciais, P., Huang, M.T., Menzel, A., Uelas, J.P., Song, Y., Vitasse, Y., Zeng,
  Z.Z. \& Janssens, I.A. (2015) Declining global warming effects on the
  phenology of spring leaf unfolding. \emph{Nature} \textbf{526}, 104--107.

\bibitem[{Gauzere \emph{et~al.}(2019)Gauzere, Lucas, Ronce, Davi \&
  Chuine}]{Gauzere2019}
Gauzere, J., Lucas, C., Ronce, O., Davi, H. \& Chuine, I. (2019) Sensitivity
  analysis of tree phenology models reveals increasing sensitivity of their
  predictions to winter chilling temperature and photoperiod with warming
  climate. \emph{Ecological Modelling} \textbf{411}, 108805.

\bibitem[{Gu(2016)}]{Gu:2016wa}
Gu, S. (2016) Growing degree hours - a simple, accurate, and precise protocol
  to approximate growing heat summation for grapevines. \emph{International
  Journal of Biometeorology} \textbf{60}, 1123--1134.

\bibitem[{Hamilton \emph{et~al.}(2016)Hamilton, El~Kayal, Hart, Runcie,
  Arango-Velez \& Cooke}]{Hamilton2016}
Hamilton, J.A., El~Kayal, W., Hart, A.T., Runcie, D.E., Arango-Velez, A. \&
  Cooke, J.E. (2016) {The joint influence of photoperiod and temperature during
  growth cessation and development of dormancy in white spruce (Picea glauca)}.
  \emph{Tree Physiology} \textbf{36}, 1432--1448.

\bibitem[{Hellmers(1966)}]{Hellmers:1966wz}
Hellmers, H. (1966) Growth response of redwood seedlings to thermoperiodism.
  \emph{Forest Science} \textbf{12}, 276--283.

\bibitem[{Koerner \& Basler(2010)}]{koerner2010a}
Koerner, C. \& Basler, D. (2010) Phenology under global warming. \emph{Science}
  \textbf{327}, 1461--1462.

\bibitem[{K{\"o}rner \& Basler(2010)}]{Jennifer:2010un}
K{\"o}rner, C. \& Basler, D. (2010) Response---warming, photoperiods, and tree
  phenology. \emph{Science} \textbf{329}, 278--278.

\bibitem[{Laube \emph{et~al.}(2014)Laube, Sparks, Estrella, H{\"o}fler, Ankerst
  \& Menzel}]{Laube2014}
Laube, J., Sparks, T.H., Estrella, N., H{\"o}fler, J., Ankerst, D.P. \& Menzel,
  A. (2014) Chilling outweighs photoperiod in preventing precocious spring
  development. \emph{Global Change Biology} \textbf{20}, 170--182.

\bibitem[{L{\'o}pez‐Olmeda \& S{\'a}nchez‐V{\'a}zquez(2009)}]{Olemda_2009}
L{\'o}pez‐Olmeda, J.F. \& S{\'a}nchez‐V{\'a}zquez, F.J. (2009) Zebrafish
  temperature selection and synchronization of locomotor activity circadian
  rhythm to ahemeral cycles of light and temperature. \emph{Chronobiology
  International} \textbf{26}, 200--218.

\bibitem[{MacLean \& Gilchrist(2019)}]{MacLean:2019aa}
MacLean, H.J. \& Gilchrist, G.W. (2019) Temperature, photoperiod and life
  history traits in drosophila subobscura. \emph{bioRxiv} p. 717967.

\bibitem[{Parent \emph{et~al.}(2019)Parent, Millet \& Tardieu}]{Parent:2019ug}
Parent, B., Millet, E.J. \& Tardieu, F. (2019) The use of thermal time in plant
  studies has a sound theoretical basis provided that confounding effects are
  avoided. \emph{Journal of Experimental Botany} \textbf{70}, 2359--2370.

\bibitem[{Piao \emph{et~al.}(2007)Piao, Friedlingstein, Ciais, Viovy \&
  Demarty}]{Piao2007}
Piao, S., Friedlingstein, P., Ciais, P., Viovy, N. \& Demarty, J. (2007)
  Growing season extension and its impact on terrestrial carbon cycle in the
  northern hemisphere over the past 2 decades. \emph{Global Biogeochemical
  Cycles} \textbf{21}.

\bibitem[{Poorter \emph{et~al.}(2016)Poorter, Fiorani, Pieruschka,
  Wojciechowski, van~der Putten, Kleyer, Schurr \& Postma}]{Poorter:2016aa}
Poorter, H., Fiorani, F., Pieruschka, R., Wojciechowski, T., van~der Putten,
  W.H., Kleyer, M., Schurr, U. \& Postma, J. (2016) Pampered inside, pestered
  outside? differences and similarities between plants growing in controlled
  conditions and in the field. \emph{New Phytologist} \textbf{212}, 838--855.

\bibitem[{P{\"o}rtner \& Farrell(2008)}]{Portner:2008vd}
P{\"o}rtner, H.O. \& Farrell, A.P. (2008) Physiology and climate change.
  \emph{Science} \textbf{322}, 690--692.

\bibitem[{Primack \emph{et~al.}({2015})Primack, Laube, Gallinat \&
  Menzel}]{Primack2015}
Primack, R.B., Laube, J., Gallinat, A.S. \& Menzel, A. ({2015}) {From
  observations to experiments in phenology research: investigating climate
  change impacts on trees and shrubs using dormant twigs}. \emph{{Annals of
  Botany}} \textbf{{116}}, {889--897}.

\bibitem[{Rosenberg(1974)}]{Rosenberg1974}
Rosenberg, N.J. (1974) \emph{Microclimate: the biological environment}. Wiley,
  New York.

\bibitem[{Sanz-Perez \emph{et~al.}(2009)Sanz-Perez, Castro-Diez \&
  Valladares}]{Sanz-Perez:2009aa}
Sanz-Perez, V., Castro-Diez, P. \& Valladares, F. (2009) Differential and
  interactive effects of temperature and photoperiod on budburst and carbon
  reserves in two co-occurring {M}editerranean oaks. \emph{Plant Biology}
  \textbf{11}, 142--51.

\bibitem[{Scheiner \& Gurevitch(2001)}]{schneiner2001}
Scheiner, S.M. \& Gurevitch, J. (2001) \emph{Design and analysis of ecological
  experiments}. Oxford University Press.

\bibitem[{Schubert \emph{et~al.}(2009)Schubert, Mourad, Fink \&
  Schwarze}]{SCHUBERT200984}
Schubert, M., Mourad, S., Fink, S. \& Schwarze, F. (2009) Ecophysiological
  responses of the biocontrol agent trichoderma atroviride (t-15603.1) to
  combined environmental parameters. \emph{Biological Control} \textbf{49},
  84--90.

\bibitem[{Stewart \emph{et~al.}(2013)Stewart, Dossena, Bohan, Jeppesen, Kordas,
  Ledger, Meerhoff, Moss, Mulder, Shurin, Suttle, Thompson, Trimmer \&
  Woodward}]{Stewart:2013wz}
Stewart, R.I.A., Dossena, M., Bohan, D.A., Jeppesen, E., Kordas, R.L., Ledger,
  M.E., Meerhoff, M., Moss, B., Mulder, C., Shurin, J.B., Suttle, B., Thompson,
  R., Trimmer, M. \& Woodward, G. (2013) \emph{Chapter Two - Mesocosm
  Experiments as a Tool for Ecological Climate-Change Research}, vol.~48, pp.
  71--181. Academic Press.

\bibitem[{Warrington \emph{et~al.}(1977)Warrington, Peet, Patterson, Bunce,
  Haslemore \& Hellmers}]{Warrington:1977vt}
Warrington, I., Peet, M., Patterson, D., Bunce, J., Haslemore, R. \& Hellmers,
  H. (1977) Growth and physiological responses of soybean under various
  thermoperiods. \emph{Functional Plant Biology} \textbf{4}, 371--380.

\bibitem[{Way \& Montgomery(2015)}]{WAY:2015aa}
Way, D.A. \& Montgomery, R.A. (2015) Photoperiod constraints on tree phenology,
  performance and migration in a warming world. \emph{Plant, Cell \&
  Environment} \textbf{38}, 1725--1736.

\bibitem[{Wolkovich \emph{et~al.}(2022)Wolkovich, Chamberlain, Buonaiuto,
  Ettinger \& Morales-Castilla}]{Wolkovich:2022ue}
Wolkovich, E.M., Chamberlain, C.J., Buonaiuto, D.M., Ettinger, A.K. \&
  Morales-Castilla, I. (2022) Integrating experiments to predict interactive
  cue effects on spring phenology with warming. \emph{New Phytologist}
  \textbf{235}, 1719--1728.

\bibitem[{Wolkovich \emph{et~al.}(2019)Wolkovich, Ettinger, Flynn, Savas,
  Chamberlain, Buonaiuto \& Samaha}]{wolkovich2019}
Wolkovich, E.M., Ettinger, A.K., Flynn, D., Savas, T., Chamberlain, C.,
  Buonaiuto, D. \& Samaha, J. (2019) Observed spring phenology responses in
  experimental environments (\uppercase{OSPREE}). doi:10.5063/F1CZ35KB.

\bibitem[{Wright \& Bruni(2004)}]{WRIGHT200433}
Wright, M.L. \& Bruni, N.K. (2004) Influence of the photocycle and thermocycle
  on rhythms of plasma thyroxine and plasma and ocular melatonin in late
  metamorphic stages of the bullfrog tadpole, rana catesbeiana.
  \emph{Comparative Biochemistry and Physiology Part A: Molecular & Integrative
  Physiology} \textbf{139}, 33--40.

\bibitem[{Xu \emph{et~al.}(2019)Xu, Song, Li, Pan, Zhu \& Zhang}]{XU2019167}
Xu, D., Song, X., Li, F., Pan, X., Zhu, W. \& Zhang, X. (2019) Effects of
  desiccation, diurnal temperature changes and irradiance on archeospore
  production of pyropia yezoensis. \emph{Aquaculture} \textbf{509}, 167--170.

\bibitem[{Zaslavski \emph{et~al.}(1995)Zaslavski, Zinovjeva, Reznik \&
  Umarova}]{Zaslavksi_1995}
Zaslavski, V.A., Zinovjeva, K.B., Reznik, S.Y. \& Umarova, T.Y. (1995) Effect
  of photoperiod and thermoperiod on the eclosion rhythm of trichogramma
  evanescens. \emph{Entomologia Experimentalis et Applicata} \textbf{74},
  99--104.

\bibitem[{Zohner \emph{et~al.}(2016)Zohner, Benito, Svenning \&
  Renner}]{Zohner:2016uz}
Zohner, C.M., Benito, B.M., Svenning, J.C. \& Renner, S.S. (2016) Day length
  unlikely to constrain climate-driven shifts in leaf-out times of northern
  woody plants. \emph{Nature Climate Change} \textbf{6}, 1120--1123.

\end{thebibliography}


\begin{figure}[h!]
    \centering
 \includegraphics[width=.8\textwidth]{..//Plots/periodicity_figures/factorial.jpeg}
    \caption{Idealized experimental designs demonstrate three approaches for varying temperature and light treatment levels in controlled environment experiments. Design \textbf{a)} is full factorial in that treatments levels are balanced and orthogonal. This design is appropriate for testing interactions between two (or more) variables. In \textbf{b)} the design is balanced but not orthogonal. Non-orthogonality in experiments can arise when experimental covariation among the manipulated variables is not accounted for. In \textbf{c)}, the experimental design is orthogonal but unbalanced. Lack of balance in experiments often arises due to time, space or resource limitations.}
    \label{fig:examp}
\end{figure}

 \begin{figure}[h!]
    \centering
 \includegraphics[width=.8\textwidth]{..//Plots/periodicity_figures/apparent.jpeg}
    \caption{Estimated effects of photoperiod and forcing on spring phenology based on a simulated experiment in which the coupling of photoperiod and thermoperiod introduce an experimental covariation between the temperature and light treatments. The dotted lines in \textbf{a)} and \textbf{c)} depict the true effects of forcing at each photoperiod level, and the solid lines depict the estimated effects. \textbf{a)} depicts a scenario where forcing and photoperiod effects do not interact, while \textbf{c)} includes an interactive effect. \textbf{b)} and \textbf{d)} depict the estimated effects of forcing and photoperiod if the experimental covariation due to periodicity coupling in \textbf{a)} and \textbf{c)}, respectively, is unacknowledged. For an example of this principle in experiments with more than two treatment levels see Fig. \ref{fig:suplines}.}
    \label{fig:3d}
\end{figure}
 
\begin{figure}[h!]
    \centering
 \includegraphics[width=.8\textwidth]{..//Plots/periodicity_figures/modelcomps_winter.jpeg}
    \caption{Estimated phenological sensitivity ($\Delta$ day of leaf expansion/$\Delta$ unit increase in cue level), using alternative methods of varying thermoperiod relative to photoperiod. Points indicate the estimated mean effect and bars the 90\% uncertainty intervals. The full posterior distributions for each parameter are also depicted as an additional display of uncertainty. The coupled thermo- photo- period design is from \citet{Flynn2018} and the uncoupled design is from \cite{Buonaiuto:2021ug}.}
    \label{fig:compy}
\end{figure}

\begin{figure}[h!]
    \centering
 \includegraphics[width=\textwidth]{..//Plots/periodicity_figures/designs.jpeg}
    \caption{Conceptualized experimental designs to test temperature and daylength interactions on a biological response. Design \textbf{a)}  manipulates temperature intensity only (no thermoperiodicity). In design \textbf{b)}, consistent diurnal temperature fluctuations are maintained but thermoperiod and photoperiod are decoupled and varied independently, maintaining orthogonality in daily temperature treatments. In \textbf{c)}, thermoperiod is included as an additional, explicit experimental treatments to evaluate the individual and interactive effect of photoperiod, thermoperiod and temperature.}
    \label{fig:designs}
\end{figure}
 

\end{document}


