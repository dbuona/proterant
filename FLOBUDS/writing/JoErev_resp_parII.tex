\documentclass[11pt]{article}
%Required: You must have these
\usepackage{graphicx}
\usepackage{tabularx}
\usepackage{natbib}
\usepackage{pdflscape}
\usepackage{array}
\usepackage{authblk}
\usepackage{gensymb}
\usepackage{amsmath}
%\usepackage[backend=bibtex]{biblatex}
\usepackage[small]{caption}

\setkeys{Gin}{width=0.8\textwidth}
\setlength{\captionmargin}{30pt}
\setlength{\abovecaptionskip}{10pt}
\setlength{\belowcaptionskip}{10pt}

\topmargin -1.5cm 
\oddsidemargin -0.04cm 
\evensidemargin -0.04cm 
\textwidth 16.59cm
\textheight 21.94cm 
\parskip 7.2pt 
\renewcommand{\baselinestretch}{1} 	
\parindent 0pt
\usepackage{setspace}
\usepackage{lineno}
#\bibliographystyle{..//..//refs/bibstyles/amnat.bst}
\usepackage{xr-hyper}
\usepackage{hyperref}

\begin{document}
\emph{Reviewer comments are in italics.} Author responses are in plain text.\\

\emph{Handling Editor Comments for Authors:\\
Both reviewers felt this ms has merit, and I agree that this is an interesting study. However, I also agree with Reviewer 1 that the current text could do much more to motivate the examination of variation in flower-leaf sequences. In particular, Reviewer 1 notes that the importance of flower-leaf sequences is much less obvious for insect-pollinated species (which constitute 6 of the 10 species analyzed, with only 2 being strictly wind-pollinated and the other 2 being both wind- and insect-pollinated). Reviewer 2 offers several suggestions to improve the figures and presentation of results.}\\

Thanks. We have made significant adjustments to the abstract, introduction and conclusion to elaborate on the complexities and caveats of the functional significance of FLS variation, with a particular focus on how this function may differ for pollinator syndrome.\\

\emph{I would also like to see the authors address the phylogenetic non-independence of their study species, several of which are congeners. My understanding is that using species as a "grouping factor" accounts for the non-independence of observations of the same species but does not account for phylogenetic relatedness.}\\

Explain and adjust methods. Maybe do some pics.\\

\emph{Finally, I have a few additional suggestions to clarify aspects of the Abstract:L5-6: Unclear why observed variation in FLS (among species, populations, individuals) indicates that the relative timing of these events is important. Couldn't such variation also suggest relative timing is unimportant; otherwise, why would variation persist?}\\

Yes. We can change this languate and rather say studies suggest that its important.\\


\emph{L8: Here it is asserted that anticipating the extent of these shifts is key, but why is it key? I did not feel this had been demonstrated by the preceding text.}\\

Connect this to the function\\

\emph{L16: Please clarify here if flower and leaf buds respond with differential sensitivity within species/among species/both.}\\

Within here, but both matter\\

\emph{L22: Unclear why expectation is that these taxa will experience reproductive declines; please explain briefly.}\\
Clarifiy that the direction of shifts in WP species is maladaptvie.\\

\textbf{Reviewer: 1}\\

\emph{COMMENTS FOR THE AUTHOR\\
I think this paper takes an intriguing approach to considering the phenology of leaves and flowers by examining whether they share similar or different cues. It makes a strong argument for why wind-pollinated species that flower first might be more strongly impacted by climate change and has an interesting discussion of mechanisms that could drive FLS. Overall, I don’t have much to comment on about the analyses and data, but I do have some questions about how things were framed. I outlined two main points of consideration and some minor comments below.}\\

Thanks!\\

\emph{In general, I would like to see a stronger argument for why FLS matters or have the paper focus more on just wind pollination (see below) or on differences between vegetative and reproductive phenology (not the interval).  The argument that FLS matters to wind pollinated plants seems reasonable (and less complicated) but for insect-pollinated plants seems less clear. Are there more citations that support its importance to insect pollination? Savage 2019 doesn’t show this (even though it is cited here). Janzen does suggest this but in the middle of a longer list and only suggests that it ``may" matter. In the context of the current study, FLS is between budburst and flowering, and for many species the overlap would only be with young leaves. Would this be enough to claim visibility issues? Would this even matter depending on the non-visual cues for pollinator attraction? It also seems that when discussing climate change and insect-pollinated flowers, it becomes tricky without tracking changes in the insects. There are so many things that impact insect-pollinated flowers (that are well-studied) including what other species are flowering (how likely is a plant to get pollen from the right species?), the size of floral displays and synchrony. I am left wondering whether FLS matters for insect-pollinated flowers. Is the presence of flowering first in the early spring more about flowering early instead of the coordination of leaves and flowers?  I may be wrong and there might be a stronger argument out there for FLS, but I did not feel that it came across well in this paper at least for insect-pollinated species. I think it is important to take the time to build the argument of why FLS matters in general or the papers should focus more on wind-pollination and talk about insect-pollination by comparison (or set insect-pollination up as more questionable).}

We thank the Reviewer one for this valuable point and agree fully that the function of FLS variation may be of less importance in insect pollinated species, and have in fact, addressed this issue in some of our previous work \citep{Buonaiuto2020}. We felt that the Reviewer's points meritted a more in depth discussion in our manuscript, and have added several paragraphs to the Introduction, Methods and Discussion to more explictly address the nuances of FLS function across species.\\

In \lineref{wind1}-\lineref{wind2} we now discussing the function of FLS variation in wind-pollinated and insect-pollinated taxa seperate so that we can more acurately present the uncertaintly surround the function of FLS variation for insect pollinated species. We also qualify in \lineref{wind3}-\lineref{wind4} that even shifts of the same magnitude will likely have different impacts in wind vs. insect pollinated species given the importance of FLS variation in each of these groups.\\

At the same time, we have also added two more citations that we feel are better evidence for the insect-visibility hypothesis. We agree with the reviewer that this hypothesis remains largely speculative and it's is complicated by the fact that the overlap between flower and leafing may occur when leaves are small and it is unknown at what stage any impact would occur. Because this is also an issue for the wind-pollination effeciency hypothesis we have added a more explicit discussion of this issue in \lineref{wind5}-\lineref{wind6}.\\

Finally, we have added a more explicit discussion about why we choose species with different pollinator syndromes (as well as bud types) to our Methods section to highlight this contrast (lines \lineref{}-\lineref{}).\\

As the reviewer suggested we have tried to more clearly demarcate our discussion of FLS variation in wind-pollinated and insect-pollinated taxa, focusing more on the contrasts between them and highlighting that the role of FLS variation in insect-pollinated species is more uncertain. We feel this comparision approach appropriately leverages our data (which as mentioned contains mostly insect pollinated species) while providing a more nuanced and realistic framing for the role of FLS shifts in forest communities. We thank the reviewer for guiding us to this shift in approach, and feel that the manuscript is much improved because of it.\\

\emph{I think that ``bud type" needs to be brought up and addressed earlier. It is first mentioned in the methods with no explanation with what it refers to (line 164) and then was mentioned to be important to the results with no reference to the analysis and how it was relevant (lines 283-284). It seems to me that bud type is very important. Whether the flower and leaf bud are united should determine whether it is possible for a plant to flower before it has leaves.  As seen in the summary table, most of the species that flower before they produce leaves have separate buds. Separate buds have the flexibility to have different leaf and flower phenology in a way that mixed buds likely do not (physically and physiologically). I think it would be good if the paper included a brief explanation of bud type and their implications to the plasticity of FLS in the introduction. This would make the discussion of bud type in the discussion clearer and not seem tangential.}\\

We agree with the reviewer on this point as well. Our initial omission of the discussion of phyiscal constaints on phenological shifts was a missed opportunity to address a factor that is likely important to our specific questions and to the study of phenology and climate change in general. To address this we now explicitly introduce and define bud type, as well as several other physical constrant in lines \lineref{bud1}-\lineref{bud2} in our Introduction, and highlight them again in greater depth in our Discussion (lines \lineref{}-\lineref{}). Additionally, we have added to our Methods more information about why we included multiple bud types in this study (lines \lineref{}-\lineref{}). \\


\emph{Smaller comments:}\\

\emph{Line 150 – I assume dormancy is when plants lose their leaves? There are multiple stages of dormancy in buds and it might be simpler to explain this in terms of the visual cue you are talking about instead of using the term dormancy.}

We agree with the reviewer and have made this change in line \lineref{}.\\

\emph{Line 166 – It is not really ''preventing" cavitation by cutting. It can minimize embolism in the stem by cutting off the end.}

Thanks to the reviewer for providing more precise language for this methodological step. We have adjusted the sentence accordingly (now line \lineref{}).\\


\emph{Line  177 – This would be more clear if you said ``leaf budburst". I am a bit confused why two similar stages were not picked for flowers and leaf buds. Why not pick a more equivalent budburst stage for flowers instead of open flowers or a leaf unfolding stage more similar to open flowers? This doesn’t matter much but did confuse me a bit when reading the paper because of terminology. For example, on line 274, it says ``leaf and flower bud phenological responses". This made me stop for a second and double check that bud phenology was not measured for flowers. I realize this is a small point, but it might be clearer to write ``leaf bud and flower phenological responses". I noticed this in several places in the paper.}\\

We thank the reviewer for this point and their suggestion that we clarify the our description of these phenological stages. We based our phenological observations on the BBCH scale, choosing BBCH 07 which is the first of the 6 stages describing leaf expansion and BBCH 60, which is the first of the 7 stages thatdescribe the progression of flowering \citep{}. Because each of stage is first one of its respective sequences we feel these stages do reflect some eqivlency between them and the verbal descriptions of ``budburst" and ``first flower opening" are perhaps unfortunately confusing, but conventional descriptions. While the BBCH scale does describe flower bud development in (stage 50-59) this stage was not observable for many of the species in our study. With that said, we agree with the Reviewer that without a in depth knowledge of the BBCH scale, our descriptions omay confuse readers and have clarified our language throughout the manuscript. DO THIS.\\




\emph{Lines  357-358. What are the implications for this mismatch for insect-pollinated species? Does it result in changes in flower synchrony, which could be a bigger problem than changes in FLS because of the ability of plants to outcross? The next paragraph seems primarily tied to wind-pollination. Are you just talking about wind-pollinated species here?}
 
Not sure what to do here. R1s points relate to flower phenology shifts in generally, not FLS specifically. Maybe add a paragraph about the impacts of population level variation in phenology in generally messing with synchrony.\\
 
\emph{lines 408-409. I still wonder how much this is the case. It would be nice to see leaf out instead of bud burst because open buds will not interfere with pollen dispersal (or minimally). How much later is leaf expansion in these species? Is there ever a reversal of leaf expansion and flowering? I know this likely cannot be addressed with the current data but would be worth discussing. Do you know anything of the time frame of flowering? Would the predicted shifts likely interfere with pollen and/or shorten the effective pollen dispersal window?}\\

\lineref{wind5}-\lineref{wind7}

I could potentially redo analysis with leafout measurement. Or talk about the average time from budburst to leaf out is small in these species. Also add some qualification about how we don't know exacly when leaves become an impediment, community matters etc. Also Add something about the duration of flowering in wind pollinated species.\\

\emp{Line 413 – I agree with this fully, but this is the first time this is really said in the paper. There is lots of hand waving about insect-pollination. Is that needed or can you frame things more strongly in terms of wind pollination?}\\

I could, but the problem is we only got data for 3 wind pollinated species, so I'd have to hardcore set up the abiotic vs. biotic contrast in the intro which I will try to do.\\

\textbf{Reviewer: 2}

\emph{COMMENTS FOR THE AUTHOR}\\
\emph{Buonaiuto and Wolkovich present a study of how three factors affect the relative phenology of leaves (budburst) and flowers (first opening).  This was an experimental study conducted in a growth chamber, conducted on ten woody plant species found in the New England study site. They used their fully factorial study design to test two competing hypotheses regarding what drives differences in leaf and flower phenology. The authors took the work one step further and attempted to project how relative phenology will change under different future climate scenarios.\\

Across the conditions they tested, the authors found that the factors associated with temperature (chilling, forcing) were more influential than photoperiod. The study design allows for compelling evaluation of the ``DSH" and ``FHH", and the conclusions they reach are convincing. The manuscript has many strengths, including the presentation of the competing hypotheses, the study design, statistical analysis, the test of whether the FHH is a special case of the DSH, reconciling seemingly contradictory findings from previously published articles, and the careful consideration of what this work means for species of with different life history characteristics. I would especially like to commend the authors on the section 'Hypotheses for FLS variation', which is written in a style that I appreciate and that I associate with the work of Wolkovich.}\\

Thanks!

\emph{There are some changes required before the manuscript is ready for publication. Below I go through critiques and suggested edits that I breakdown into categories: major, intermediate, and minor/optional. All of these critiques can be addressed with the data already analyzed and presented.}\\

Will do!\\

\emph{MAJOR COMMENTS:\\
In general, there are elements of the figures and tables that are presented too informally.}

Agreed. Thank you for point this out we have made these changes.\\

\emph{-Some labels are inconsistent with the wording used in the text. In Figure 2, the term 'light' is used when it should be 'photoperiod' (as in the text) or 'photo' (as in Figure 1).}\\

Search document and figures for light and change.\\

\emph{-Figure 4, the names of the different scenarios should be more consistent with the variables being modeled (especially 'warming only')}

Change scenario names.\\

\emph{-Table S2, I do not understand the reference to the 'Utah model'}\\

Maybe add to the caption, but not sure what do to here.\\


\emph{-Figure S1, why is the word 'scenario' included in the x-axis label? Also, for Fig S1, an 89\% credible interval should not be presented, as the 50\% interval is used elsewhere.}\\

Remove axis lable. change CI.\\

\emph{-Figure S2, an explanation of the labels on the x-axis is needed in the figure caption.}\\

add to it.\\


\emph{It is my understanding that all of the data presented are standardized values. It would be valuable to have summaries of the unstandardized values presented at least once, even if it is in the supplement. An unstandardized version of Table S5 would be great.}\\

Do this and change caption because interactions aren't truely in the model. Maybel also do this for figure S3 and 4\\ 


\emph{INTERMEDIATE COMMENTS:}\\
\emph{I spent some time staring at the figures of this manuscript wondering if it would be helpful to present the graphical results in terms of the length of the FLS as a single variable. This is especially true for Fig 2, and somewhat for Fig 4. The information is available as the gap between the triangles and the circles, but with so may pairs it is hard to follow. An additional panel, figure, or supplemental figure showing the FLS value would be helpful.}\\

I can do this as a supplimental figure, or use the FLS model itself to show it.\\

\emph{For Fig 1, I might prefer presenting the panels in a different order: FHH, DSH, and the combination}\\

Sure thing!\\

\emph{I’d like a bit more emphasis on results in the Abstract. Nothing too long, for example I recommend highlighting consistency of results across species.}\\

Add a line to abstract.\\

\emph{Line 266 and elsewhere. When discussing the photoperiod treatment, the language of ‘increasing’ photoperiod is a bit awkward or counterintuitive. In terms of departing from the historic natural conditions, the treatment (and possible expectation with climate change) is that photoperiod will 'decrease'. I understand the point from a statistical perspective, and I don’t dispute it. But because it is not intuitive, the reader might be helped with some guidance/reminders.}

This is a fair point. Talk about changes in photoperiod.\\


MINOR/OPTIONAL COMMENTS:
\emph{When I first read the title, I was not sure if 'Differences' referred to the different bud types or the different species. A slight re-wording might be in order.}

Cool cool. Think on this.

\emph{Line 63-67, the authors state that a decrease in FLS interphase could harm reproduction in wind-pollinated species. Is there any reason to believe that an increase could also be harmful?}

Hmm. Not sure We did not have enough data to compare female and male phenological responses, which may be more or less tied to leaf phenology (ie they are mixed buds). This could tie in to R1s point but might open up a can of worms.\\

\emph{Line 125, change 'with' to 'will'}\\

Okay.\\


\emph{Line 133, 353, and elsewhere. The authors make the point that under the DSH, shifts in FLS could be quite local. I would expect that they would be more 'regional' than 'local'. The latter seems a too-narrow scale. Perhaps this perspective is shaped by the very flat, elevationally homogenous places I've lived.}\\

We can change this wording.\\

\emph{Line 186, change 'standardized' to 'standardize'}\\

We made this change.\\


\emph{Line 224, the authors refer to a 5 deg C increase in temperature although their study design uses a 6 deg increase. Is this an error? Do they mean that climate projections predict a 5 deg increase for the study site?}\\

CHeck on this.\\

\emph{The authors do a good job of explaining how the implications of their findings may differ for wind-pollinated versus insect-pollinated species, and for flowering-first versus leafing-first species. I would like to see them (briefly) consider tree versus shrub.}\\

Think on this, but maybe say we can evaluate it but being lower growing the effects might be even worse.\\

\emph{Line 281, change 'showed' to 'show'}\\

K.\\

\emph{Line 304-307, although a point is made later, I think it is worth mentioning here that both species shrubs, flowering-first, and wind-pollinated}\\

Will do.\\

\emph{Line 347, missing a comma after 'temperatures' (optional)}

K.\\

\emph{Line 372-374, another supporting point may be that photoperiod had less of an effect}.\\

Excellent point. We have added this.\\


\emph{Line 413, delete 'which is of little relevance to abiotically pollinated taxa'}\\

Okay, but it felt like R1 liked that.\\

\emph{Supplemental Methods Line 13, change the wording of "In this scenario we let increased both chilling..."}

Great.\\


\end{document}