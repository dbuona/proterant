\documentclass[11pt]{article}
%Required: You must have these
\usepackage{graphicx}
\usepackage{tabularx}
\usepackage{natbib}
\usepackage{pdflscape}
\usepackage{array}
\usepackage{authblk}
\usepackage{gensymb}
\usepackage{amsmath}
%\usepackage[backend=bibtex]{biblatex}
\usepackage[small]{caption}

\setkeys{Gin}{width=0.8\textwidth}
\setlength{\captionmargin}{30pt}
\setlength{\abovecaptionskip}{10pt}
\setlength{\belowcaptionskip}{10pt}

\topmargin -1.5cm 
\oddsidemargin -0.04cm 
\evensidemargin -0.04cm 
\textwidth 16.59cm
\textheight 21.94cm 
\parskip 7.2pt 
\renewcommand{\baselinestretch}{1} 	
\parindent 0pt
\usepackage{setspace}
\usepackage{lineno}
\bibliographystyle{..//..//sub_projs/refs/styles/besjournals.bst}
\usepackage{xr-hyper}
%\usepackage{hyperref}
\externaldocument{SUPP_FLS_flobud}
\externaldocument{FLS_flobud}

\begin{document}
\emph{Reviewer comments are in italics.} Author responses are in plain text.\\

\textbf{Handling Editor Comments for Authors:}\\

\emph{Both reviewers felt this ms has merit, and I agree that this is an interesting study. However, I also agree with Reviewer 1 that the current text could do much more to motivate the examination of variation in flower-leaf sequences. In particular, Reviewer 1 notes that the importance of flower-leaf sequences is much less obvious for insect-pollinated species (which constitute 6 of the 10 species analyzed, with only 2 being strictly wind-pollinated and the other 2 being both wind- and insect-pollinated). Reviewer 2 offers several suggestions to improve the figures and presentation of results.}\\

\noindent We thank the Editor and both Reviewers for their feedback and are pleased that they found our manuscript interesting and valuable. Based on their suggestions, we have made significant adjustments to the manuscript and figures to elaborate on the complexities and caveats of the functional significance of FLS variation, with a particular focus on how this function may differ depending on pollinator syndrome. We detail these changes further below.\\

\emph{I would also like to see the authors address the phylogenetic non-independence of their study species, several of which are congeners. My understanding is that using species as a ``grouping factor'' accounts for the non-independence of observations of the same species but does not account for phylogenetic relatedness.}\\

\noindent We appreciate the Editor's concerns on this issue, as it has been demonstrated that FLSs can be strongly influenced by phylogeny \citep{Buonaiuto2020,Gougherty2018}. When including congenerics in the study, we chose species specifically to avoid issues with phylogenetic non-independence. As phylogenetic non-independence is primarily an issue when a trait (in this case FLS) is inherited from a common ancestor we chose specific congenerics that have different FLSs, suggesting each species' FLS evolved independently \citep{Revell:2010aa}. Thus our use of congenerics was meant to be a strength of the study design, but we failed to explain this in our original submission, and we think that many readers will (and should) share the Editor's concerns about phylogenetic non-independence! We have thus added text in the Methods (line \lineref{phylo1}-\lineref{phylo2}) to clarify our selection criteria, and appreciate the editor pointing this out.\\ 

\noindent To triple check that phylogenetic non-independence was not impacting the inference of our analyses, we have also calculated the phylogenetic signal for mean FLS interphase among species using a phylogeny pruned from \citet{Zanne2013} and the R package ``phytools" \citep{Revell2012}. With the species in our study, the phylogenetic signal for the trait ``mean FLS interphase" was low (lambda=0.129) suggesting that for our species assemblage, phylogenetic non-independence is not influencing our results. We did not include this analysis in the manuscript, but would be happy to add it if the Editor thinks it would be helpful for readers.\\

\emph{Finally, I have a few additional suggestions to clarify aspects of the Abstract:L5-6: Unclear why observed variation in FLS (among species, populations, individuals) indicates that the relative timing of these events is important. Couldn't such variation also suggest relative timing is unimportant; otherwise, why would variation persist?}\\

\noident We are grateful to the Editor for helping us fine tune our Abstract and found these suggestions very helpful. We agree that variation does not inherently indicate functional importance of a trait. We were basing this assertion on previous work in the literature rather than first principles, and have adjusted the language accordingly (line \lineref{ab1}).\\ %EMW24Mar2021: I think what you say is fine here, we just don't neccessarily need to cut it do we?

\emph{L8: Here it is asserted that anticipating the extent of these shifts is key, but why is it key? I did not feel this had been demonstrated by the preceding text.}\\

\noindent We agree with the Editor that this point was unclear. We have re-worked this sentence, merging it with the preceeding sentence to make the connection between potential impacts of changes in FLS on reproduction, recruitment and survival of individuals and species performance clearer (line \lineref{abs2}). % We change the sentence to simply suggest that predicting FLS shifts require a better understanding of how FLS patterns are dictated by the environment

\emph{L16: Please clarify here if flower and leaf buds respond with differential sensitivity within species/among species/both.}\\

\noindent We have altered this sentence to clarify the intra- and inter-specific differences in FLS responses (line \lineref{abs3}-\lineref{abs4}).

\emph{L22: Unclear why expectation is that these taxa will experience reproductive declines; please explain briefly.}\\
 
\noindent We have elaborated and clarified this assertion. This portion of the Abstract now reads:\\
 
 ``When we projected how FLSs are likely to shift under several generalized climate change scenarios, we found that FLS shifts were largest in wind-pollinated species that flower before leafing, with flower-leaf interphases substantially shortened under all scenarios. This flower-leaf interphase is critical for effective pollen transfer in wind-pollinated taxa, and the direction and magnitude of shifts we found for these species raises the possibility that, more generally, wind-pollinated taxa may experience reproductive declines due to FLS shifts in the decades to come."\\

\textbf{Reviewer: 1}\\

\emph{COMMENTS FOR THE AUTHOR\\
I think this paper takes an intriguing approach to considering the phenology of leaves and flowers by examining whether they share similar or different cues. It makes a strong argument for why wind-pollinated species that flower first might be more strongly impacted by climate change and has an interesting discussion of mechanisms that could drive FLS. Overall, I don’t have much to comment on about the analyses and data, but I do have some questions about how things were framed. I outlined two main points of consideration and some minor comments below.}\\

\noindent We thank the Reviewer for their thoughtful comments on our manuscript and glad they found our work interesting. We appreciated the Reviewer's point about the framing of the paper, and have made a number of adjustments which we detail below. \\

\emph{In general, I would like to see a stronger argument for why FLS matters or have the paper focus more on just wind pollination (see below) or on differences between vegetative and reproductive phenology (not the interval). The argument that FLS matters to wind pollinated plants seems reasonable (and less complicated) but for insect-pollinated plants seems less clear. Are there more citations that support its importance to insect pollination? Savage 2019 doesn’t show this (even though it is cited here). Janzen does suggest this but in the middle of a longer list and only suggests that it ``may" matter. In the context of the current study, FLS is between budburst and flowering, and for many species the overlap would only be with young leaves. Would this be enough to claim visibility issues? Would this even matter depending on the non-visual cues for pollinator attraction? It also seems that when discussing climate change and insect-pollinated flowers, it becomes tricky without tracking changes in the insects. There are so many things that impact insect-pollinated flowers (that are well-studied) including what other species are flowering (how likely is a plant to get pollen from the right species?), the size of floral displays and synchrony. I am left wondering whether FLS matters for insect-pollinated flowers. Is the presence of flowering first in the early spring more about flowering early instead of the coordination of leaves and flowers?  I may be wrong and there might be a stronger argument out there for FLS, but I did not feel that it came across well in this paper at least for insect-pollinated species. I think it is important to take the time to build the argument of why FLS matters in general or the papers should focus more on wind-pollination and talk about insect-pollination by comparison (or set insect-pollination up as more questionable).}

\noindent We thank the Reviewer for this valuable point and agree fully that the function of FLS variation may be of less importance in insect pollinated species, and have in fact, addressed this issue in some of our previous work \citep{Buonaiuto2020}. We felt that the Reviewer's points merited a more in depth discussion in our manuscript, and have added several paragraphs to the Introduction, Methods and Discussion to more explicitly address the nuances of FLS function across species.\\

\noindent Beginning at line \lineref{wind1} we now discuss the function of FLS variation in wind-pollinated and insect-pollinated taxa separate so that we can more accurately present the uncertainty surround the function of FLS variation for insect pollinated species. We also qualify in lines \lineref{wind3}-\lineref{wind4} that even shifts of the same magnitude will likely have different impacts in wind vs. insect pollinated species given the importance of FLS variation in each of these groups.\\

\noindent At the same time, we have also added two more citations that we feel are better evidence for the insect-visibility hypothesis. We agree with the Reviewer that this hypothesis remains largely speculative. Further it is complicated by the fact that the overlap between flower and leafing may occur when leaves are small and it remains unknown at what stage any impact would occur. Because this is also an issue for the wind-pollination efficiency hypothesis, we have added a more explicit treatment of this issue in lines \lineref{wind5}-\lineref{wind6}.\\ 

\noindent As the reviewer suggested, we have tried to more clearly demarcate our discussion of FLS variation in wind-pollinated and insect-pollinated taxa, focusing more on the contrasts between them and highlighting that the role of FLS variation in insect-pollinated species is more uncertain. We feel this comparative approach appropriately leverages our data (which contains mostly insect pollinated species) while providing a more nuanced and realistic framing for the role of FLS shifts in forest communities. We thank the Reviewer for guiding us to this shift in approach, and feel that the manuscript is much improved because of it.\\

\emph{I think that ``bud type" needs to be brought up and addressed earlier. It is first mentioned in the methods with no explanation with what it refers to (line 164) and then was mentioned to be important to the results with no reference to the analysis and how it was relevant (lines 283-284). It seems to me that bud type is very important. Whether the flower and leaf bud are united should determine whether it is possible for a plant to flower before it has leaves.  As seen in the summary table, most of the species that flower before they produce leaves have separate buds. Separate buds have the flexibility to have different leaf and flower phenology in a way that mixed buds likely do not (physically and physiologically). I think it would be good if the paper included a brief explanation of bud type and their implications to the plasticity of FLS in the introduction. This would make the discussion of bud type in the discussion clearer and not seem tangential.}\\

\noindent We agree with the Reviewer on this point as well. Our initial omission of the discussion of physical constraints on phenological shifts was a missed opportunity to address a factor that is likely important to our specific questions and to the study of phenology and climate change in general. To address this we now explicitly introduce and define bud type, as well as several other physical constraints in lines \lineref{buds1}-\lineref{buds2} Additionally, we have added to our Methods a more explicit statement about why we included multiple bud types in this study (line \lineref{traits1}). We agree this make the mention of bud type in the Discussion feel more relevant to the paper as a whole. \\


\emph{Smaller comments:}\\

\emph{Line 150 – I assume dormancy is when plants lose their leaves? There are multiple stages of dormancy in buds and it might be simpler to explain this in terms of the visual cue you are talking about instead of using the term dormancy.}

\noindent We agree with the Reviewer and have made this change in line \lineref{harvest}.\\

\emph{Line 166 – It is not really ``preventing" cavitation by cutting. It can minimize embolism in the stem by cutting off the end.}

\noindent Thanks to the Reviewer for providing more precise language for this methodological step. We have adjusted the sentence accordingly (now line \lineref{embolism}).\\


\emph{Line  177 – This would be more clear if you said ``leaf budburst". I am a bit confused why two similar stages were not picked for flowers and leaf buds. Why not pick a more equivalent budburst stage for flowers instead of open flowers or a leaf unfolding stage more similar to open flowers? This doesn’t matter much but did confuse me a bit when reading the paper because of terminology. For example, on line 274, it says ``leaf and flower bud phenological responses". This made me stop for a second and double check that bud phenology was not measured for flowers. I realize this is a small point, but it might be clearer to write ``leaf bud and flower phenological responses". I noticed this in several places in the paper.}\\

\noindent We thank the Reviewer for this point and their suggestion that we clarify the our description of these phenological stages. We based our phenological observations on the BBCH scale, choosing BBCH 07 which is the first of the 6 stages describing leaf expansion and BBCH 60, which is the first of the 7 stages that describe the progression of flowering \citep{Finn2007}. Because each of stage is first one of its respective sequences we feel these stages do reflect some equivalency between them. %and the verbal descriptions of ``budburst" and ``first flower opening" are perhaps unfortunately confusing, but conventional descriptions. While the BBCH scale does describe flower bud development in (stage 50-59) this stage was not observable for many of the species in our study. With that said, 
We agree with the Reviewer that without an in depth knowledge of the BBCH scale, our descriptions may confuse readers and have clarified our language throughout the manuscript now referring to budburst as ``leaf budburst" throughout the manuscript.\\

\noindent Additionally, we did observe a later stage of leaf development (BBCH 15 or ``leafout") in our experiment. In light of the the Reviewer's feedback and the fact that the functional hypothesis for FLS and wind-pollination efficiency hinges on developing leaves being sufficiently large, we expanding our models and climate projects to include this additional stage in our analyses. We have re-created all relevant figures (Fig. \ref{fig:model}, Fig. \ref{fig:preddy}, Fig. \ref{fig:preddy_sp}) to display these results along with our previous analyses, and in the main text we now report and discuss the sensitivity of leafout to environmental cues in additional to the other two phases and explain our rationale for this at line \lineref{caveat1}.\\ 

\emph{Lines  357-358. What are the implications for this mismatch for insect-pollinated species? Does it result in changes in flower synchrony, which could be a bigger problem than changes in FLS because of the ability of plants to outcross? The next paragraph seems primarily tied to wind-pollination. Are you just talking about wind-pollinated species here?}
 
\noindent Here, we also agree with the Reviewer that this point may primarily apply to wind-pollinated species. We have clarified this in the manuscript in line \lineref{polly1} and added a sentence addressing this population level shifts in FLSs for biotically-pollinated species (line \lineref{polly2}) to further the contrast of FLS functions among these two functional groups and emphasize the uncertainty and need for more research regarding how FLS variation impacts biotic pollination. \\
 
\emph{lines 408-409. I still wonder how much this is the case. It would be nice to see leaf out instead of bud burst because open buds will not interfere with pollen dispersal (or minimally). How much later is leaf expansion in these species? Is there ever a reversal of leaf expansion and flowering? I know this likely cannot be addressed with the current data but would be worth discussing. Do you know anything of the time frame of flowering? Would the predicted shifts likely interfere with pollen and/or shorten the effective pollen dispersal window?}\\

\noindent We thank the reviewer for this point. We believe the Reviewer is correct that the early stages of leaf expansion that we measured would not alter the structure of the canopy much and such impacts would be expected at later stages of leafout, but it is unclear at exactly what point in leaf development expanding leaves would become a barrier to pollen transport.
Based on this, as mentioned above, we incorporated models and climate change projections for leafout (``BBCH 15") to our analyses and Discussion, explicitly addressing the idea that leaves must be a certain size before they would be expected to impact canopy structure in lines \lineref{wind5}-\lineref{wind7}.\\
%We chose to observed BBCH 07 for vegetative development to maximize our sample size as we have found in in both small pilot studies and previous published work that there is often a portion of twig cuttings that reach BBCH 07 but fail before they reach later stages.

\emph{Line 413 – I agree with this fully, but this is the first time this is really said in the paper. There is lots of hand waving about insect-pollination. Is that needed or can you frame things more strongly in terms of wind pollination?}\\

\noindent We have tried to more strongly differentiate the implications of FLS shifts for wind vs. insect pollinated taxa. We hope that the Reviewer feels that our revised submission appropriately addresses the importance of FLS shifts for wind-pollinated species throughout the manuscript.\\

\textbf{Reviewer: 2}

\emph{COMMENTS FOR THE AUTHOR}\\
\emph{Buonaiuto and Wolkovich present a study of how three factors affect the relative phenology of leaves (budburst) and flowers (first opening).  This was an experimental study conducted in a growth chamber, conducted on ten woody plant species found in the New England study site. They used their fully factorial study design to test two competing hypotheses regarding what drives differences in leaf and flower phenology. The authors took the work one step further and attempted to project how relative phenology will change under different future climate scenarios.}\\

\emph{Across the conditions they tested, the authors found that the factors associated with temperature (chilling, forcing) were more influential than photoperiod. The study design allows for compelling evaluation of the ``DSH" and ``FHH", and the conclusions they reach are convincing. The manuscript has many strengths, including the presentation of the competing hypotheses, the study design, statistical analysis, the test of whether the FHH is a special case of the DSH, reconciling seemingly contradictory findings from previously published articles, and the careful consideration of what this work means for species of with different life history characteristics. I would especially like to commend the authors on the section `Hypotheses for FLS variation', which is written in a style that I appreciate and that I associate with the work of Wolkovich.}\\

\noindent We thanks the Reviewer for their time and feedback on this manuscript and are pleased that they identified several strengths in our initial submission and enjoyed our writing style.\\

\emph{There are some changes required before the manuscript is ready for publication. Below I go through critiques and suggested edits that I breakdown into categories: major, intermediate, and minor/optional. All of these critiques can be addressed with the data already analyzed and presented.}\\

\noindent We thank the Reviewer for their feedback and suggestions, and agree that the changes they suggested strongly improve the presentation of our findings. The changes we have made to the manuscript are detailed below.\\

\emph{MAJOR COMMENTS:\\
In general, there are elements of the figures and tables that are presented too informally.}

\noindent We have made several change to the figures and tables to improve them. Specifically, we now use more technical language to describe the relevant phenological phases (leaf budburst, leafout and flowering instead of ``leaf" vs. ``flower") and describe our environmental treatments.  \\

\emph{-Some labels are inconsistent with the wording used in the text. In Figure 2, the term `light' is used when it should be `photoperiod' (as in the text) or 'photo' (as in Figure 1).}\\

\noindent We have changed the labels to ``Photo" for consistency with the rest of the paper.\\

\emph{-Figure 4, the names of the different scenarios should be more consistent with the variables being modeled (especially 'warming only')}

\noindent We agree with the Reviewer that these labels were confusing. We now describe the scenarios based on the treatment levels that we applied to simulate each scenario.\\

\emph{-Table S2, I do not understand the reference to the `Utah model'}\\

\noindent We think this is a good point, and in our orginal submission we did not adaquately reference the three major models used to calculate chilling (the Utah Model is one of them). We have added a citation that is a comprehensive review of the three different models to the caption and adjusted the language of the caption accordingly.\\

\emph{-Figure S1, why is the word `scenario' included in the x-axis label? Also, for Fig S1, an 89\% credible interval should not be presented, as the 50\% interval is used elsewhere.}\\

\noident We thank the reviewer for pointing out this inconsistency. We have adjusted the figure to portray the 50\% credible interval and removed the x-axis label, which was printed unintentionally in our previous version of figure.\\

\emph{-Figure S2, an explanation of the labels on the x-axis is needed in the figure caption.}\\

\noindent We agree with the Reviewer that these labels do not clearly describe the climate scenarios we are projecting in this figure. As this figure is reflecting the same projections as Fig. \ref{fig:preddy} in the main text but grouped by species rather than FLS type, in the new version of this we have adopted the same labels that Reviewer 2 recommended for Fig. \ref{fig:preddy} above. We hope this change clarifies the interpretation of this plot.\\


\emph{It is my understanding that all of the data presented are standardized values. It would be valuable to have summaries of the unstandardized values presented at least once, even if it is in the supplement. An unstandardized version of Table S5 would be great.}\\

\noindent We are very grateful to the Reviewer for addressing this issues as it helped us identify a larger issue in our manuscript. In our preliminary analyses, we ran our models several times both with unscaled variables (that is, in natural units, such as `days') and using several different methods of standardization (e.g., z-scores) to make sure our results were robust (which they were). The results that we ended up reporting in the paper are in fact not standardized---this sentence in our Methods description should have been removed, and we thank the Reviewer for catching this error.\\

\noindent All of our results and figures are in fact regular units scaled by the treatment levels. We have added these units to to our reporting in the Result section and changed the captions of all relevant figures and tables to explicitly define sensitivity as ``$\Delta$ day of phenological event/ $\Delta$ environmental cue; 4 weeks chilling/6 \degree C forcing/4 hours photoperiod". We hope this clarifies our results, and apologize for the confusion in our original submission.\\ 

\emph{INTERMEDIATE COMMENTS:}\\
\emph{I spent some time staring at the figures of this manuscript wondering if it would be helpful to present the graphical results in terms of the length of the FLS as a single variable. This is especially true for Fig 2, and somewhat for Fig 4. The information is available as the gap between the triangles and the circles, but with so may pairs it is hard to follow. An additional panel, figure, or supplemental figure showing the FLS value would be helpful.}\\

We agree with the Reviewer that we should also use our data to visualize the differences in the sensitivity among phases. We have added additional panels for Fig. \ref{fig:preddy} and Fig. \ref{fig:preddy_sp}, and created an alternative version of Fig. \ref{fig:model} that displays the response differences between phases (gaps between the shapes) for the Supporting Information (Fig. \ref{fig:altview}) to aide with this interpretation (we can move this figure to the main text if requested). \\ 

\emph{For Fig 1, I might prefer presenting the panels in a different order: FHH, DSH, and the combination}\\

\noindent Agreed, we have made this change.\\

\emph{I’d like a bit more emphasis on results in the Abstract. Nothing too long, for example I recommend highlighting consistency of results across species.}\\

\noindnet We thank the Reviewer for this idea and have incorporated their suggestion at line \lineref{sults1}.

\emph{Line 266 and elsewhere. When discussing the photoperiod treatment, the language of ‘increasing’ photoperiod is a bit awkward or counterintuitive. In terms of departing from the historic natural conditions, the treatment (and possible expectation with climate change) is that photoperiod will 'decrease'. I understand the point from a statistical perspective, and I don’t dispute it. But because it is not intuitive, the reader might be helped with some guidance/reminders.}

\noindent We think this is a good point the Reviewer makes, and in our revised manuscript we tried to strike a better balance between consistency with the statistical orientation of our figures and the intuitive understanding many readers have about how photoperiod impacts phenology. We have removed references to ``increasing photoperiod'' throughout the manuscript, replacing it with simply ``longer'' or ``shorter'' photoperiod where appropriate (see lines \lineref{photo1}-\lineref{photo2}).\\


MINOR/OPTIONAL COMMENTS:
\emph{When I first read the title, I was not sure if 'Differences' referred to the different bud types or the different species. A slight re-wording might be in order.}\\


\noindent We have changed the title to more precisely reflect our study:\\
``Differences between flower and leaf phenological responses to environmental variation drive shifts in spring phenological sequences of temperate woody plants".\\

\emph{Line 63-67, the authors state that a decrease in FLS interphase could harm reproduction in wind-pollinated species. Is there any reason to believe that an increase could also be harmful?}

\noindent We thank the Reviewer for raising this interesting point. We think this is an important point to consider and have added a paragraph to the manuscript to discuss it in greater detail from lines \lineref{increase1}-\lineref{wind7}. \\

\emph{Line 125, change `with' to `will'}\\

\noindent We have made this change.\\


\emph{Line 133, 353, and elsewhere. The authors make the point that under the DSH, shifts in FLS could be quite local. I would expect that they would be more 'regional' than 'local'. The latter seems a too-narrow scale. Perhaps this perspective is shaped by the very flat, elevationally homogenous places I've lived.}\\

\noindent We agree with the Reviewer that it is difficult to clearly predict the geographic scale at which FLS differences will occur. We now qualify this in the manuscript by suggesting that there may be:\\

``strongly localized or regional effects of climate change on FLSs." (line \lineref{reg}).\\

\emph{Line 186, change `standardized' to `standardize'}\\

\noindent We made this change.\\


\emph{Line 224, the authors refer to a 5 deg C increase in temperature although their study design uses a 6 deg increase. Is this an error? Do they mean that climate projections predict a 5 deg increase for the study site?}\\

\noindent This was a typo in the original submission, which we apologize for and have fixed.\\

\emph{The authors do a good job of explaining how the implications of their findings may differ for wind-pollinated versus insect-pollinated species, and for flowering-first versus leafing-first species. I would like to see them (briefly) consider tree versus shrub.}\\

\noindent We thank the Reviewer for this point, and we now explicitly mention the possible differences between tree and shrubs in line \lineref{shrub}. We think this is an important comparison and designed our study in hopes to address it more fully. Due to low flowering of two species, however, our final analysis consisted of far more shrub species than trees (seven shrubs, two under story trees and only one canopy tree). Thus, further analysis of this difference is now beyond the scope of our data.\\ 

\emph{Line 281, change `showed' to `show'}\\

\noindent We have made this change.\\

\emph{Line 304-307, although a point is made later, I think it is worth mentioning here that both species shrubs, flowering-first, and wind-pollinated}\\

\noindent We have added this information in this line (line \lineref{minor2}).

\emph{Line 347, missing a comma after `temperatures' (optional)}\\

\noindent We added this comma.\\

\emph{Line 372-374, another supporting point may be that photoperiod had less of an effect}.\\

\noindent We think this is a strong point and have added this to the manuscript in line \lineref{weak}.\\


\emph{Line 413, delete 'which is of little relevance to abiotically pollinated taxa'}\\

\noindent We have removed this line.\\

\emph{Supplemental Methods Line 13, change the wording of ``In this scenario we let increased both chilling...''}\\

\noindent We have clarified this sentence (line \lineref{supple}). \\

\bibliography{..//..//sub_projs/refs/hyst_outline.bib} 
\end{document}