\documentclass[11 pt]{article}
\usepackage[margin=.85in]{geometry}
\usepackage{graphicx}
\usepackage{natbib}
\usepackage{gensymb}
%\begin{footnotesize}
%\address{1300 Centre Street \\ Boston, MA, 20131}
%\end{footnotesize}
\begin{document}
\bibliographystyle{..//..//sub_projs/refs/styles/besjournals.bst}
\def\labelitemi{--}
\parindent=24pt
\noindent\includegraphics[width=0.3\textwidth]{AA_logo.jpg}
\pagenumbering{gobble}

\noindent{Dear Dr. Fox,}\\
\vspace{1.5ex}

\noindent Please consider our revised manuscript ``Experimental designs for testing the interactive effects of temperature and light in ecology: the problem of periodicity" as a ``Commentary" in \textit{Functional Ecology}.\\

\noindent Experiments in controlled environments have been used for decades to quantify the effects of environmental stimuli on biological processes to the benefit of fundamental ecology and applied forecasting. %Yet, seemingly small choices about design can generate significant differences in experimental outcomes. 
Our submission details how a commonly used experimental design to estimate the effects of temperature and photoperiod on spring phenology results in the incorrect estimation of cue effects when the periodicity of light and temperature treatments are unintentionally covaried. We find that this problem is widespread in the ecological literature, and we use simulations, an algebraic solution and a comparative analysis of published studies to describe the scope of the problem. Importantly, we also provide guidance for dealing with this issue statistically, and offer alternative experimental designs to improve inference in studies that include periodicity of multiple variables.\\

\noindent Comments from the Associate Editor and two reviewers suggested our manuscript was timely and clear, and our analyses thorough and robust, but pointed out several areas for improvement regarding our proposed solutions to the issues we described.
While one reviewer felt our proposed modifications were too complex to be practically adopted, the other indicated more complex examples and solutions my be necessary for biological realism. %We feel that these contrasting points highlight the importance of the topic of our manuscript, and offered us the opportunity to address potential solutions to the problem we describe more broadly.\\
We believe there are a number of solutions to address the problem of periodicity that range from simple statistical corrections to elaborate experimental redesigns. Reconciling the contrasting feedback of the two reviewers has driven us to more comprehensively layout these options.\\% in our manuscript.\\%, which we feel has improved its usefulness and clarity for readers of \emph{Functional Ecology} and experimental ecologists in general.\\

\noindent Based on their comments, we have revised the structure of our ``Paths Forward" section to highlight the range of options, and emphasize that the exact choice a researcher makes in this area will depend on their study system, research question and resources. Additionally, %while trying to maintain simple examples throughout the text for clarity and broad applicability, 
we have created a new figure for the Supporting Information, made changes to the Abstract and main text all to more clearly relate the issues presented in our simple examples to larger, increasingly complex experiments. We feel that the editor's and reviewers' input has helped shape a new submission that is much improved, and we detail our specific changes in the following pages with reviewer comments in \emph{italics} and our responses in regular text.\\

\noindent The main text of this manuscript is 4,002 words in length and it contains four figures. It is co-authored by M. Donahue and E.M. Wolkovich, and is not under consideration elsewhere. We hope that you will find it suitable for publication in \textit{Functional Ecology}, and look forward to hearing from you.\\


\noindent Best,\\
\\



\noindent Daniel Buonaiuto

\end{document}
