\documentclass[11pt]{article}
%Required: You must have these
\usepackage{graphicx}
\usepackage{tabularx}
\usepackage{natbib}
\usepackage{array}
\usepackage{amsmath}
%\usepackage[backend=bibtex]{biblatex}
\setkeys{Gin}{width=0.8\textwidth}
%\setlength{\captionmargin}{30pt}
\setlength{\abovecaptionskip}{10pt}
\setlength{\belowcaptionskip}{10pt}
 \topmargin -1.5cm 
 \oddsidemargin -0.04cm 
 \evensidemargin -0.04cm 
 \textwidth 16.59cm
 \textheight 23.94cm 
 \parskip 7.2pt 
\renewcommand{\baselinestretch}{1.2} 	
\parindent 0pt

\begin{document}

\textbf{How experiments can better assess the influence of temperature and daylength on biological activity}

Changes in daylength (hours of light each day) and temperature are important environmental cues for major biological processes, such as spring leafout. To understand how exactly these cues alter biological activity, researchers have long used labs, greenhouses and indoor growth chamber experiments that alter the daylength and temperature organisms experience. One common way to make these experiments more realistic is to include daily temperature variation---with lower night- than day-time temperatures---just like in nature. However, in studies that are also interested in the effects of daylength, this means that longer daylength treatments also receive more warmth over a 24-hour period. This makes it impossible to separate the influence of temperature and daylength on the biological activity of interest.

Here, we examine a case study from experiments on how temperature and daylength affect the timing of budburst in trees and shrubs to illustrate how experiments that combine daylength treatments with different day and night temperatures cause problems for understanding these cues. Through multiple approaches, including theory and comparing different designs, we show that experiments of this kind will generally overestimate the influence of daylength and underestimate the importance of temperature. We also find that almost 40\% of published experiments that investigate the effect of daylength of the timing of spring budburst have this issue, which may partially explain why the influence of daylength on spring leafout is still debated.

We also show there are ways for researchers to avoid this problem. Options we present range from simple, statistical corrections to novel and complex experimental set-ups, but in the end, a better understanding of the effects of temperature and daylength on biological activity will contribute substantially to our ability the predict how organisms will respond to climate change.
 
 
Photo: An experiment to quantify the effects of temperature and daylength on the timing of spring leafout in growth chambers at the Arnold Arboretum in Boston, MA, USA. Photo credit: Tim Savas.
 
\end{document} 
 %However, becoming more aware of this potential experimental pitfall can allow researchers to improve experiments moving forward.
 
 %However, researcher must design these artificial environments, and need to balance the need to make experiments as realistic as possible with other limitations like technology, time, and space, and seemingly small differences in designs can have big effects on the results of a study. 
