\documentclass[11pt]{article}
%Required: You must have these
\usepackage{graphicx}
\usepackage{tabularx}
\usepackage{natbib}

\usepackage{array}
\usepackage{amsmath}
%\usepackage[backend=bibtex]{biblatex}
\setkeys{Gin}{width=0.8\textwidth}
%\setlength{\captionmargin}{30pt}
\setlength{\abovecaptionskip}{10pt}
\setlength{\belowcaptionskip}{10pt}
\topmargin -1.5cm 
\oddsidemargin -0.04cm 
\evensidemargin -0.04cm 
\textwidth 16.59cm
\textheight 23.94cm 
\parskip 7.2pt 
\renewcommand{\baselinestretch}{1.2} 	
\parindent 0pt


\bibliographystyle{..//refs/styles/besjournals.bst}
%\usepackage{xr-hyper}
\usepackage{hyperref}
\title{Supporting Information: Experimental designs for testing the interactive effects of temperature and light in biology(ecology) and the problem of periodicity }
\begin{document}
\maketitle
\subsection{Light and temperatre and spring phenology}
Here a very brief (one or two paragraph) overview of how light and temperature influence spring phenology. Keep it basic (Warming accelerate phenology, photoperiod might be a threshold), aknoledge chilling is important too, but our example won't really focus on it. Leave some open questions about their interactive nature.
\subsection{OSPREE table of interactive studies}
Maybe there is some limiting cues code that can help me with this?
\subsection{Math}
Maybe Lizzie (or Megan) can write this part since they were the ones who did this math?

% EMW: Putting this here for now, but can move some/all it to main text as needed. [] means I am not sure we need it, but I added it. 
[See comments in tex file.] \\
When experimental designs co-vary thermo- and photoperiodicity, they may incorrectly estimate effects of temperature, photoperiod and (for experiments with crossed designs) the interaction between photo and thermo-periodicity. While robustly testing the effect of this covariation requires additional experiments, you can solve for its potential effect based on the geometry of the experimental design given several assumptions.\\

If we assume that the effects of photoperiod and forcing (alone) are linear and additive (no FxP interaction), then the treatmeants and their results occupy a three-dimensional plane with temperature on the $x$ axis, photoperiod on the $y$ axis and the response on the $z$ axis. Given our assumptions we can solve for the expected response if temperature and photoperiod were uncoupled. \\

To show this, we follow the design of CITEFLYNN as an example. We used summed temperature (degree hours), to reflect the duration and intensity of temperature, on the $x$ axis, photoperiod (hours) on the $y$ axis and the response ($\delta$ days to budburst) on the $z$ axis. [The combination of the covaried treatments and responses produces a parallelogram in space, and we effectively solve for it as a rectangular---orthogonal---set of treatments.] Given the equation for a 3D plane and our knowledge of three sets of the coordinates:

\begin{align}
ax+by+cz+d & =0\\
200a + 8b + 0 + d &= 0\\
240a + 12b+4.5c + d &=0\\
320a + 8b + 9c+ d &=0
\end{align}

A, we can solve for the effects if we had been able to decouple the treatments. In this example, given that we know the $\delta$ days to budburst for 200 degree hours (temperature) at 8 hours of photoperiod, as well as 240 degree hours (temperature) at 12 hours of photoperiod, we essentially solve for the $\delta$ days to budburst given orthogonal treatments (e.g., 200 degree hours (temperature) at 12 hours of photoperiod; 240 degree hours (temperature) at 8 hours of photoperiod). \\
% Dan: Can you make the two figures on last page of PDF notes?
% I would put this worked math below in the supp ...

Algebraically, we simply solve for each unknown in turn. We begin by setting $z$ to zero:
\begin{align}
200a + 8b + 0 + d & = 0 \text{ and solving for d yields:} \\
-200a - 8b = & d  % setting c to 0
\end{align}
Now, we can solve for $c$ (in terms of $a$ and $b$) using our solution for $d$:
\begin{align}
240a+12b+4.5c+(-200a - 8b) &=0\\
-40/4.5a+4/4.5b & = c% simplified 
\end{align}
And similarly then solve for $a$ given our last set of coordinates and solutions for $d$ and $c$ (not shown). This yields:
\begin{align}
a & =\frac{1}{5}b\\
c & =-\frac{8}{3}b\\
d & =-48b\\
\end{align}
Putting these back into the equation for a plane yields an overall equation to estimate the response $z$ (days) for any combination of $x$ (temperature) and $y$ (photoperiod):
\begin{align}
\frac{1}{5}bx + by -\frac{8}{3}bz -48b&=0\\
\frac{3}{40}x + \frac{3}{8}y-18 & = z\\
\end{align}
Comparing:
\begin{align}
\frac{3}{40}*(200) + \frac{3}{8}*(8)-18 &=0\\ 
\frac{3}{40}*(200) + \frac{3}{8}*(12)-18 &=3\\ 
\end{align}
Shows that---given our assumptions---the estimated photoperiod effect that may be due to temperature is 3 days. 

% Also show forcing? I did not finish this ... 
% \frac{3}{40}*(360) + \frac{3}{8}*(8)-18 &=12 \text{ days}\\

\subsection{Modeling methods}
To do.\\
Also maybe a table of model comparison to go along with the figure.
\maketitle
\end{document}