\documentclass[11pt]{article}
%Required: You must have these
\usepackage{graphicx}
\usepackage{tabularx}
\usepackage{natbib}

\usepackage{array}
\usepackage{amsmath}
%\usepackage[backend=bibtex]{biblatex}
\setkeys{Gin}{width=0.8\textwidth}
%\setlength{\captionmargin}{30pt}
\setlength{\abovecaptionskip}{10pt}
\setlength{\belowcaptionskip}{10pt}
\topmargin -1.5cm 
\oddsidemargin -0.04cm 
\evensidemargin -0.04cm 
\textwidth 16.59cm
\textheight 23.94cm 
\parskip 7.2pt 
\renewcommand{\baselinestretch}{1.2} 	
\parindent 0pt


\bibliographystyle{..//refs/styles/besjournals.bst}
%\usepackage{xr-hyper}
\usepackage{hyperref}
\title{Supporting Information: Experimental designs for testing the interactive effects of temperature and light in biology(ecology) and the problem of periodicity }
\begin{document}
\maketitle
\subsection{Light and temperatre and spring phenology}
Here a very brief (one or two paragraph) overview of how light and temperature influence spring phenology. Keep it basic (Warming accelerate phenology, photoperiod might be a threshold), aknoledge chilling is important too, but our example won't really focus on it. Leave some open questions about their interactive nature.
\subsection{OSPREE table of interactive studies}
Maybe there is some limiting cues code that can help me with this?
\subsection{Math}
Maybe Lizzie (or Megan) can write this part since they were the ones who did this math?
\subsection{Modeling methods}
To do.\\
Also maybe a table of model comparison to go along with the figure.
\maketitle
\end{document}