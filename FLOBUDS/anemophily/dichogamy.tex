\documentclass[11pt]{article}
\renewcommand{\baselinestretch}{1.8}
\usepackage{textcomp}
\usepackage{fontenc}
\usepackage{graphicx}
\usepackage{caption} % for Fig. captions
\usepackage{gensymb} % for \degree
\usepackage{placeins} % for \images
\usepackage[margin=1in]{geometry} % to set margins
\usepackage{setspace}
\usepackage{lineno}
%\usepackage{cite}
\usepackage{amssymb} % for math symbols
\usepackage{amsmath} % for aligning equations
\usepackage[sort&compress]{natbib}
\bibliographystyle{refs/styles/gcb.bst}
\title{Caution to the wind: global change and reproductive uncertainty for wind-pollinated plants}
\begin{document}
\maketitle


\noindent The vast majority of terrestrial plants rely on mutulistic associations with animal pollinators for reproduction \citep{Giannini1994}. Biotic pollination is one of the many ecological processes being disrupted by global change \citep{Gerard:2020aa}. With growing evidence that extirpation of pollinators \citep{Burkle:2013aa}, phenological mismatches \citep{Memmott2007} and species invasions \citep{Dietzsch:2011aa} can dramatically reduce reproductive success of plants and threaten their persistence, pollinator services have become a high priority topic in global change research \citep{Dicks975,Kremen:2000aa}.\\

\noindent Often left out from the scientific conversations about global change and pollination are the estimated 10\%-20\% of terrestrial plants that are pollinated by the wind \citep{Ollerton:2011aa,Friedman:2009aa,Ackerman:2000aa}. While this may be a small percentage taxonomically, wind-pollinated species dominate the vast temperate and boreal regions of the globe \citep{Regal:1982aa} and are critical for human livelihood and ecological well being.\\

\noindent While pollinators are declining with global change, the wind remains. Despite predicted changes to wind patterns associated with climate change \citep{IPCC2013} we are aware of no scientifically credible studies that have predicted that these shifts will adversely affect wind-pollination. It is tempting to think that wind-pollinated taxa might be immune to the uncertain pollination future faced by their biotically-pollinated relatives. In fact, it is likely that wind-pollination evolved from biotically-pollinated ancestors for reproductive assurance in times when pollinator services were unreliable \citep{Friedman:2009aa}. Some have even suggested as pollinator services continue to decline with global change, wind-pollinated species may come to dominate many ecosystems \citep{Bond:1995aa,Hoiss:2013aa}.\\

\noindent However, the continued reproductive success of wind-pollinated species in an era of global climate change should not be taken for granted. While the literature pertaining to abiotic pollination is far less developed than for biotic pollination in general \citep{Friedman:2009aa}, we have identified several aspects of the pollination biology shared by many wind-pollinated species that may be vulnerable to disruptions from changing climate. We highlight some of these characteristics below, first detailing how climate change may affect these essential processes and then identifying critical research needs to better understand how these changing processes may impact the fitness of wind-pollinated taxa in the future.

\subsection*{Rain-scavenging}
\noindent Pollen release in wind pollinated species generally occurs during periods of low humidity when the chance of precipitation is low \citep{Niklas1985, Whitehead1969}. Rainfall events are extremely effective at removing airborne pollen \citep{Kluska:2020aa}. Through this phenomenon, known as raindrop-scavenging,  precipitation can virtually eliminate all pollen from the air that it encounters and the associated humidity can damage pollen grains through osmotic shock \citep{Niklas1985}. With the frequency and intensity of precipitation events changing with climate change \citep{IPCC2013}, it follows that changing precipitation patterns during the time of pollen release may effect the the amount and quality of airborne pollen available for pollination.\\

\noindent Several studies have found negative associations between precipitation and airborne pollen counts \citep{Grewling:2014aa,Gross2019,Pace:2018aa} and some have even found that shifts in recent counts systematically correlate with changes in precipitation patterns \citep{Zhang:2015, Bruffaerts:2018aa}. The strength of these trends seem to vary among species, locations, and elevations \citep{Knaap:2010aa,Pace:2018aa}. In the context of global change, these impacts may be completely negated by the increasing pollen counts and longer pollen seasons driven by warming \citep{Gross2019} and increased atmospheric carbon dioxide \citep{Ziska:2000aa}. In fact, most studies predict an increase in the duration and intensity of the pollen season with climate change \citep{Ziello:2012aa,Zhang:2015}, but it is unclear how these trends will affect pollination success. Further, while precipitation events can remove pollen from the air, more precipitation in general may increase the reproductive output of plants \citep{Fernandez-Martinez:2012aa}. The specific impact of precipitation shifts on wind pollination may depend more on shifts in the timing of precipitation events than absolute amount of precipitation. \\ %Additionally, some studies have found that large storm events can dramtically increase airbourne pollen loads \citep{} whether increased precipitation is driven by more frequent or larger events may matter for pollen availability.

\noindent All of the data we found about relationships between wind pollen and environmental variables comes from public health studies concerned not with how changes in pollen counts may effect pollination, but rather how these trends will impact seasonal allergies. The significance of the trends to the reproductive success of wind-pollinated species depends on to what degree they drive pollen limitation in these taxa. In general the prevalence of pollen limitation in wind pollinated species is debated \citep{Friedman:2009aa} though several studies find evidence for it \citep{Koenig:2012aa} and suggest it may increase with global change \citep{Koenig:2003aa,Knapp:2001aa}.\\

\noindent \textbf{Research needs:} In addition to more studies testing for pollen limitation in wind-pollinated taxa in general, research should also focus on assessing the role of variable environmental conditions in driving pollen limitation dynamics in these species, as has been done for some biotically-pollinated taxa \citep{Totland:1999aa}. Studies of this kind are critical to determine whether observed changes to airborne pollen dynamics are substantial enough to effect the pollination in wind pollinated species.\\

\subsection*{Hysteranthy}
\noindent In addition to meteorological conditions, there are biological factors that dictate the timing of pollen release as well. In wind-pollinated deciduous woody plants of the temperate regions, pollen release almost always occurs in the early spring before leaf development \citep{Whitehead1969}. While this phenological syndrome is known by many names in the literature (hysteranthy, proteranthy, protanthy, precocious flowering)\citep{Buonaiuto2020}, theory, modeling and empirical studies suggest that the flowering-first phenological sequence is critical for pollination efficiency and long-distance pollen transport in wind pollinated species \citep{Tauber1967,Nathan2005,Milleron2012}.\\

\noindent Recent work suggests that the duration of the flowering to leafing interphase is shifting with climate change in many temperate woody species, yet the direction and magnitude of these shifts vary among species and populations \citep{Buonaiuto2020,Buonaiuto2020b}. Theoretically, increases in the flowering to leaf out inter-phase should improve pollination success and enhance long distance gene flow while reductions of this interphase should have opposing, detrimental effects, but to our knowledge the relationship between hysteranthy and reproductive performance have not been empirically tested. Like rain-scavenging, the functional importance of hysteranthy shifts are contingent on increasing incidences of pollen limitation with reductions to the hysteranthous period. \\

\textbf{Research needs:} It is important that future research evaluate the performance implications of alterations to hysteranthy. Given that these phenological sequences appear to be relatively plastic under natural conditions, the regular inter-populational or inter-annual variability, or artificially induced variation in hysteranthy could be leveraged to test for associations with common reproductive performance metrics such as pollen capture or seed set. Further, increased pollen interception by advancing leafout may be driven by community canopy structure \citep{Khanduri:2019aa}, and the impact of advancing leaf out on pollination success should also be investigate at this scale.\\ 



\subsection*{Dichogamy}
\noindent Like many of their insect-pollinated relatives, female and male flowers of wind-pollinated species are often temporally separated \citep{Bertin:1993aa}. This phenological pattern, known as dichogamy, is considered to be a mechanism to promote out-crossing \citep{Bertin:1993ab}or reduce interference between male and female anatomy \citep{Lloyd:1986aa,Routley:2004aa}. Dichogamy can be highly plastic and determined by both genetics and the environment \citep{Friedman2011}. Though investigations into the proximate cues of dichogamous flowering are limited, several studies suggest that variation in temperature drives variation in dichogamy \citep{Gabriel:1986aa,Schaffer1994,Alexander2016}. Therefore, sustained shifts in temperature due to climate change may disrupted patterns of dichogamy.\\

\noindent Several studies from seed orchards have reported shifts in dichogamy associated with climate change \citep{Alizoti2010,Mutke:2005aa,Elkassaby1991}, yet the direction of these shifts vary. Because most of the evidence we found for correlations between dichogamy and the environment were from seed orchards which are usually more homogeneous than natural populations it is unclear how broadly shifts in dichogamy are occurring.\\

\noindent The function of dichogamy may vary among species. While usually toted as a mechanism to prevent selfing, dichogamy is also present in predominately selfing taxa \citep{Friedman:2009ab} as well species that are self-incompatible \citep{Lloyd:1986aa,Routley:2004aa}. While theoretically, increasing overlap of female and male flowering periods may have negative fitness consequences by promoting selfing and driving inbreeding depression, the severity of these impacts will depend on differences in species' mating systems.\\ 

\textbf{Research needs:} One of the major challenges to answer questions related to dichogamy and climate change is that the majority of phenological studies do not record the timing of male and female flowers separately, making it hard to establish baseline variability for a diversity of wind-pollinated species. Many common-use phenological scales do not even differentiate between these two sub-phases of flowering \citep[e.g. the BBCH scale,][]{Finn2007}. A major research priority must be to better characterize the temporal variation in male and female flowering, and identify how both external environmental cues and internal regulators structure these patterns.\\

\noindent As with hysteranthy, research must test for correlations between variation in dichogamy and variation in performance. For dichogamy, the natural place to start is to investigate how changes in dichogamy affect ratios of self to outcrossed pollen. Similar studies exists for biotically-pollinated plants \citep[e.g.][]{Koski:2018aa,Kalisz:2011aa}. Researchers should start by focusing on species that may be must vulnerable to the negative impacts of shifting dichogamy, those that are primarily out-crossing with no internal barriers to selfing.\\

\subsection*{Masting}

\noindent For many wind-pollinated species, inter-annual seed production is highly irregular, yet reproductive episodes tend to be synchronized over large areas \citep{Bogdziewicz2017}. This behavior, know as masting, is an important mechanism for recruitment, allowing offspring to escape seed predation in years of plenty \citep{Janzen:1971aa}. There is a large body of research exploring the proximate cues of mast seeding (reviewed in \citet{Kelly:2002aa,Pearse:2016aa}). Our purpose here is to emphasize that masting is strongly link to climate variation \citep{Bogdziewicz2017,Koenig:2015aa,Kelly:2002aa,McKone1998}, and therefore shifts in climate are likely to affect masting patterns.\\

\noindent Several recent studies have already linked disruptions in masting patterns to climate change \citep{Bogdziewicz:2020aa,Shibata2020}. Yet others predict that masting behavior may be resilient to increasing temperatures \citep{Kelly:2013aa}. Shifts in masting behavior may be dependent on mean shifts in spring temperature \citep{Schermer2020,Bogdziewicz:2018aa}, variation in temperature among years \citep{Kelly:2013aa} or shifts in precipitation patterns \citep{Perez-Ramos:2010aa}, and these proximate cues may be species-specific \citep{Pearse:2020aa}. Further, additional factors affected by global change, such as nutrient availability, may also constrain shifts in masting patterns \citep{Monks:2016aa}. While our understanding of these environmental controls is increasing rapidly, global patterns of seeds production remain largely unexplained \citep{Pearse:2020aa}.\\

\citebf{Research needs:} Research into the climatic drivers of masting will continue to improve our understanding on how climate change will affect the reproductive success of wind-pollinated species. Studies should continue with a focus on expanding the geographic and taxonomic scope of such inquiries \citep{Pearse:2020aa}.

\subsection*{Synthesis:} In the sections above we have briefly detailed four components of the wind-pollination syndrome that may be disrupted by global change resulting in fitness loss for wind-pollinated plants. We acknowledge that both the likelihood of these disruptions and their consequences remain highly uncertain, yet we feel our short treatment of this subject demonstrates that there is a need for a stronger research focus on the pollination biology of wind-pollinated plants in the context of global change. \\

\noindent While each of these factors may be more or less important to the reproductive success of wind-pollinated plants, they also may interact with each other, and with other drivers of global change, amplifying their impact. For example, mast fruiting may be strongly related to pollination efficiency \citep{Sork:1993aa} which, as we have shown above, may be compromised by shifts in pollen-scavenging, hysteranthy, and dichogamy. These effects may be of greater significance in highly fragmented or modified landscapes \citep{Koenig:2003aa}, and altered further by the drivers of global change that do not discriminate between pollinator syndromes like  disease and pest spread, invasive species, and altered disturbance regimes.\\

\noindent It is important to remember that the changes to the processes we discussed above could also result in an increase in reproductive fitness. Just as some biotically pollinated species may benefit from novel pollinator assemblages, we can also imagine a wind-pollinated species whose reproductive efficiency increases with shifts in pollen-scavenging, hysteranthy, dichogamy and masting. All we can say is that the reproductive fate of wind-pollinated species in an era of global change is anything but certain, and that more research is needed.\\

\bibliography{refs/dichigamy.bib} 


\end{document}