\documentclass[11pt]{article}
%Required: You must have these
\usepackage{graphicx}
\usepackage{tabularx}
\usepackage{natbib}
\usepackage{pdflscape}
\usepackage{array}
\usepackage{authblk}
\usepackage{gensymb}
\usepackage{amsmath}
%\usepackage[backend=bibtex]{biblatex}
\usepackage[small]{caption}

\setkeys{Gin}{width=0.8\textwidth}
\setlength{\captionmargin}{30pt}
\setlength{\abovecaptionskip}{10pt}
\setlength{\belowcaptionskip}{10pt}

 \topmargin -1.5cm 
 \oddsidemargin -0.04cm 
 \evensidemargin -0.04cm 
 \textwidth 16.59cm
 \textheight 21.94cm 
 \parskip 7.2pt 
\renewcommand{\baselinestretch}{1.8} 	
\parindent 0pt
\usepackage{setspace}
\usepackage{lineno}


\usepackage{xr-hyper}
\usepackage{hyperref}


\linenumbers
\title{Caution to the wind: global change and reproductive uncertainty for wind-pollinated plants}\\

\date{}
\author{D.M. Buonaiuto $^{1,2,a}$}
\begin{document}
\noindent Biotic pollination is one of the many ecological processes being disrupted by global change \citep{Gerard:2020aa} and the study of pollinator services has become a high priority topic in global change research \citep{Dicks975,Kremen:2000aa}. Yet, often left out from the scientific conversations about global change and pollination are the estimated 10-20\% of terrestrial plants that are pollinated by the wind \citep{Ollerton:2011aa,Friedman:2009aa,Ackerman:2000aa}. While this may be a small percentage taxonomically, wind-pollinated species dominate the vast temperate and boreal regions of the globe \citep{Regal:1982aa} and are critical for human livelihood and ecological well being.\\

\noindent While pollinators are declining with global change, the wind remains. %Despite predicted changes to wind patterns associated with climate change \citep{IPCC2013}, 
It is therefore tempting to think that wind-pollinated taxa might be immune to the uncertain pollination future faced by their biotically-pollinated relatives. In fact, it is likely that wind-pollination in angiosperms evolved from biotically-pollinated ancestors for reproductive assurance in times when pollinator services were unreliable \citep{Friedman:2009aa}. Some have even suggested as pollinator services continue to decline with global change, wind-pollinated species may come to dominate many ecosystems \citep{Bond:1995aa,Hoiss:2013aa}.\\

\noindent However, the continued reproductive success of wind-pollinated species in an era of global climate change should not be taken for granted. There are several aspects of the pollination biology of wind-pollinated species that may be vulnerable to disruptions from global change.  \\
%add a sentence here

\subsection*{Atmosphereic conditions}
Wind-pollinated species typically release their pollen on sunny, dry days when the likelihood of precipitation is low and wind turbulence is high \citep{}. Under such conditions, pollen clouds can be transported great distances from a paternal source plant, and remain airborne for several days \citep{}. Departures from these atmospheric conditions can significantly reduce the airborne pollen load and the distance pollen will travel from it source \citep{}. Both wind and and precipitation patterns are predicted to shift dramatically with climate change \citep{}, which is likely to affect the pollen availability and transport in the future.\\

\noindent Wind speeds dictate both pollen entrainment and capture \citep{} and wind direction determines the patterns of gene flow across the landscape \citep{}. One of the major ways that plants can survive a warming climate in situ is through genetic rescue, which is only possible for wind pollinated species if the wind direction is the same as requisit gene flow \citep{Kling2020}. 2 more sentences.

\noindent While wind draws pollen into the air column, precipitation can virtually eliminate all pollen from the air \citep{Niklas1985,Kluska:2020aa}. Several studies have found negative associations between precipitation and airborne pollen counts \citep{Grewling:2014aa,Gross2019,Pace:2018aa} and some have even found that shifts in recent counts systematically correlate with changes in precipitation regimes \citep{Zhang:2015, Bruffaerts:2018aa}.\\ 

\noindent Paragraph summing up this section.\\


\subsection*{Phenology}
\noindent Pollen from wind-pollenated taxa is short lived--- often only viable for a number of hours to days \citep{}. As such, the timing of flowering, or floral phenology, of wind-pollianted species has evolved to be highly synchronized \citep{}. Temperature is considered to be the most important driver of phenology \citep{}, and several recent studies have indicated that increasing spring temperature are disrupting phenological synchrony within and among populations \citep{}.\\ 

Increasing asynchrony in flowering may impact reproductive fitness by simply reducing the amount of viable pollen that arrives to receptive stigmas, but in the long term, it may also impact mating patterns and genetic structure. Like many of their insect-pollinated relatives, female and male flowers of wind-pollinated species are often temporally separated \citep{Bertin:1993aa}. This phenological pattern, known as dichogamy, is considered to be a mechanism to promote out-crossing \citep{Bertin:1993ab}. Several studies from seed orchards have reported shifts in dichogamy associated with climate change \citep{Alizoti2010,Mutke:2005aa,Elkassaby1991}. If such shifts are also widespread in wild populations, primarily out-crossing taxa may experience increased self-pollination and inbreeding depression, reducing the likelihood that populations will be able to adapt in the face of novel climate exposure with climate change.

\subsection*{Community structure}
\noindent Wind pollinated species tend to occur in open habitats with high densities of conspecifics \citep{Regal 1982}. These ecologcal conditions make for effect wind pollination by increasing wind speeds for transport \citep{}, reducing the amount of pollen intercepted by other structures and increasing likelihood pollen with come into contact with target stigmas \citep{}. This requirement for open habitats as an explaination as to why so many wind-pollinated trees of temperat forests flower prior to leafing out in the spring \citep{}. \\

\noindent Several drivers of global change are altering the environmental structure of many habitats in which wind-pollinated species are found. Altered distrurbance regimes, fire surpression, and novel climate conditions correlate with an increase in forest density, especially in the understory \citep{Dolanc:2014aa}. Invasive species tend to leaf-out earlier in the season that native ones so seasonal canopy fill is more rapid. Across large wind-pollinated grassland and remnant prairies fire supression and changing precicitation patterns has lead to ``shrubification".\\

Several studies indicate that wind-pollination success decreases with increase vegetation structure \citep{Mileron2012,Khanduri:2019aa}, these impacts need to be tested more broadly. 

\section{Impacts}
In the sections above have present evidence that many of the environmental and ecological condtions under which wind pollination evolved are currently changing. While there is strong theory to suggest that this changes will impact the pollination success of wind-pollinated taxa, when it comes to predicted their impact of the reproductive fitness of wind-pollinated plants, these observations are merely half the equation. Researcher should study the impacts. 

In our theoretical treatment above, we can broadly group the potential impacts of changing abiotic conditions, phenology and community structure into two categories---pollen limitation and gene flow (population structure?).

\subsection*{Pollen limitation}
\subsection*{Populations structure}

\section*{Conclusion}
Is important to remember that the changes to the ecological processes discussed above could also positively impact the reproductive fitness of some wind-pollinated species. Just as some biotically pollinated species may benefit from novel pollinator assemblages \citep{}, one can imagine many scenarios under which the pollination effeciency of a wind-pollinated species increase as global change opens forest canapies or shifts prevailing winds in a favorable direction for genetic rescue. I will also emphasize that there are plenty of other threats to wind-pollinated taxa from global change---pest and pathogens, land conversion, drought that do not discriminate among pollinator syndromes and are unlikely to affect the pollination success of wind-pollinated species but certainly threaten their persistance. Yet, as presented above, also true that due to changing abiotic, phenological and community patterns, even the reproductive fate of wind-pollinated species in an era of global change is anything but certain and the effects of global change on the pollination biology of wind-pollinated taxa merrit research attenton.\\


\subsection*{}



\end{document}