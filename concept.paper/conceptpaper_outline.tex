\documentclass{article}
\usepackage{Sweave}
\usepackage{float}
\usepackage{graphicx}
\usepackage{tabularx}
\usepackage{siunitx}
\usepackage{mdframed}
\usepackage{cite}
\usepackage{natbib}
\bibliographystyle{..//refs/styles/besjournals.bst}
\usepackage[small]{caption}
\setkeys{Gin}{width=0.8\textwidth}
\setlength{\captionmargin}{30pt}
\setlength{\abovecaptionskip}{0pt}
\setlength{\belowcaptionskip}{10pt}
\topmargin -1.5cm        
\oddsidemargin -0.04cm   
\evensidemargin -0.04cm
\textwidth 16.59cm
\textheight 21.94cm 
%&\pagestyle{empty} %comment if want page numbers
\parskip 0pt
\renewcommand{\baselinestretch}{1.5}
\parindent 20pt

\newmdenv[
  topline=true,
  bottomline=true,
  skipabove=\topsep,
  skipbelow=\topsep
]{siderules}
\usepackage{lineno}
\linenumbers
\begin{document}
\Sconcordance{concordance:conceptpaper_outline.tex:conceptpaper_outline.Rnw:%
1 23 1 50 0}

\title{Concept paper outline}
\author{Daniel Buonaiuto, Lizzie Wolkovich, Ignacio Morales Castilla?}
\maketitle{}
\indent Green is the color of spring \citep{Schwartz1998}, but any keen observer walking the Eastern deciduous forests of North America early in the season would readily notice that it is often the subtle reds and yellows of emerging tree flowers that are the first harbingers of the season. Why do some tree species seasonally flower before leafing out? What benefit might a species gain by engaging in the costly process of reproduction at a time when the current year's photosynthesis unavailable to them, and stored energy is at a seasonal low \citep*{Hoch2003,Pfanz2002}? This trait, known as hysteranthy, is a feature common to deciduous forests, and is apparent in many commercially and ecologically important woody plant species. This flowering pattern has long been noted by botanists \citep{Rathcke1985} and several hypotheses have been put forth suggesting that this temporal trait is critical for the reproductive success of these species, but there has been little empirical investigation into the origins or significance of this pattern. Further, while the study of phenology, the timing of seasonal life cycle events, has received increased attention in the past decades for it link to anthropogenic climate change, floral and leaf phenology have long been treated as disparate processes, and have been rarely observed in tandem \citep{Wolkovich2014}. As a result, though hysteranthy is reported to be common in temperate deciduous forests, it is difficult to evaluate the prevelance of hysteranthy, or even to conclusively classify hysteranthous species.\\
\indent Interactions between temperature and day length are the dominant cues for both floral and foliar phenology in trees \citep{Forrest2010}. Significant shifts in phenology due to anthropogenic climate change have already been observed, \citep{Menzel2006}, but there is little baseline data for evaluating if and how hysteranthy has been altered \citep{Lechowicz1995}. This gap in the literature is particularly alarming, as seasonal temperatures are projected to continue to change dramatically as a result of human activity \citep{Raftery2017}. If hysteranthy is indeed affected by climate change, climate induced alteractions to this phenological pattern could have negative reproductive, and ultimately demographic, consequences for many important woody species.\\
\indent In this paper we attempt to lay a groundwork for a thorough investigation of hysteranthous flowering. More specifically, in the following sections we will:
\begin{itemize}
\item Present and evaluate the current hypotheses on these origins and significance of hysteranthy. 
\item Develop an empirical definition and framework for identifying hysteranthous species.
\item Characterize the prevalence of hysteranthy in Eastern North American temperate forests, investigate the phylogenetic signal of the trait, and identify biological traits associated with hysteranthy.
\item  Discuss the connection between hysteranthy and reproductive uncertainty in the context of climate change, and present future research directions.
\end{itemize}
\section*{A History of Hysteranthous Hypotheses}
\indent Descriptions of hysteranthous flowering in trees seems to have entered the scientific literature in the mid 1890's \citep{Robertson1895}. However, several terms to describe this phenological pattern have been used variably and imprecisely \citep{Dyer1953}, making it difficult to trace the study of this trait through the literature. Hysteranthy, succinctly defined as: producing leaves after the flowers have formed, comes from the Greek \textit{husteros} (after) and \textit{anthos} (flower) \citep{Heinig1899}. The very same phenological pattern can also be described by its linguistic opposite proteranthy or protanthy, coming from the Greek prefix \textit{pro} (before), defined as: flowering before the foliage leaves appear \citep{Heinig1899}. Some have attempted to differentiate between hysteranthy and proteranthy, defining hysteranthy more broadly as flowering anytime during the leafless season and ascribing to proteranthy a temporal component where flowering occurs seasonally \textit{prior} to budburst \citep{Lamont2011}.  A third synonym, precocious flowering, comes from the Latin \textit{praecox}, meaning premature\citep{Heinig1899}. However, precocious flowering is more widely used to describe species or individuals that flower early in their ontogeny \citep*{Seleznyova2008,Pharis1965}, and using it to describe leafless flowering produces unnecessary confusion. Additionally, many sources describing hysteranthous flowering do not use any of these terms but rather rely on verbal descriptions of the phenological pattern. We suggest adopting the general term hysteranthy, eliminating confusing synonyms and allowing for comparison with other systems in which flowering in the leafless season is common such as the dry deciduous tropics \cite{Franklin2016} and Mediterranean geophyte communities\citep{Marques2012}.\\
\indent Despite the infrequent and ambiguous descriptions of hysteranthy in the literature, several hypotheses for the origins and significance of the this phenological trait have emerged. These hypotheses can be broadly classified into two categories: functional and physiological. Functional hypotheses posit that the hysteranthous pattern confers, in and of itself, an adaptive advantage on a species, while physiological hypotheses posits that the hysteranthous pattern emerges due to physiological constraint within a species.\\
\subsection*{Functional hypotheses}
\indent Perhaps the most common explanation for the seemingly high rate of occurrence of hysteranthy in temperate deciduous species is that this phenological pattern is an adaptation critical for wind pollination, with leafless flowering allowing for more efficient pollen dispersal and transfer \citep{Whitehead1969,Rathcke1985, Spurr1980}. While we are unaware of any studies that have tested this wind pollination efficiency hypotheses directly, there are several studies that provide strong, suggestive support. Simplistically, pollen dispersal can be described a function of the terminal velocity of pollen grains and the wind intensity \citep*{Niklas1985,Whitehead1969}. It has been shown that wind velocities in forests are considerably higher in the leafless season than when a canopy is full \citep*{Brown1969,Whitehead1969}. and that vegetation structure and canopy closure reduce particle diffusion through a forest\citep{Brown1969}. These finding indicate that flowering during the leafless period would increase the possibility of long distance pollen dispersal.\\
\indent Other studies have shown that there is significant filtration of pollen by leaves \citep*{Milleron2012, Tauber1967}. A particularly relevant study Tauber \citeyear{Tauber1967}, demonstrated high rates of pollen transfer in the trunk space of forest canopies and quantified the amount of pollen impacted on non-floral structures such as branches and leaves. In this study, the author reports pollen counts on a single bare twig, and, on 20 twigs with leaves of a grey willow (\testit{Salix cinerea.}), with 40,000 grains impacted on the bare branch, and 1,687,600 grains on the 20 branches with leaves \citep{Tauber1967}. Simple arithmetic allows us to estimate that a single branch with leaves would be expected to intercept ~84,400 pollen grains, more than double than what was impacted on the bare branch. This finding suggests that flowering during the leafless season significantly reduces the amount of pollen filtration by non-floral plant tissue. Reduced pollen filtration by non-floral structures would decrease the likelihood these taxa would be pollen limited. Pollen limitation has been show to be more common in trees than other plant taxa \citep{Larson2000}, but studies of pollen limitation in wind pollinated taxa are limited \citep{Knight2005}. As far back as Darwin, it has been doctrinal that wind pollinated taxa produce an over abundance of pollen \citep{Friedman2009}, but recently, several studies have suggested the pollen and pollen dispersal limitation in wind pollinated taxa may be more prevalent than once thought \citep{Koenig2003}. Research is beginning to address the relationship between phenology, pollen limitation and reproductive output in wind pollinated species \citep*{Koenig2012, Koenig2015,Bogdziewicz2017}, but these studies tend to focus on the absolute timing and duration of the flowering season, rather than the timing of flowering relative to leaf phenology. A comparative study of pollen dispersal distances, or the frequency of pollen limitation in hysteranthous and non-hysteranthous taxa would be instructive, but such studies are conceptually challenging because it is difficult to control for other differences in pollination syndrome between taxa. One possibility would be to utilize the interanual variation in hysteranthy and test whether pollen dispersal distances or the degree pollen limitation change in association with the regular variability in the degree of hysteranthy. \\
\indent While our discussion of hysteranthy will be primarily focus on the temperate deciduous forests, hysteranthous flowering is also reported to be common in the dry-deciduous tropics \citep*{Janzen1967,Franklin2016}. In this system where the vast majority of woody plants species are biotically pollinated, a similar, pollination efficiency hypothesis has emerged, stating that hysteranthous flowering is an adaptation for increased pollinator visibility \citep{Janzen1967}. To our knowledge there have been no direct test of this hypothesis. One study by Gunatilleke and Gunatilleke \citeyear{Guantilleke1984} compared the floral of biology of three species of Dogwoods( genus: \textit{Cornus}), and found that the hysteranthous Cornelian cherry (\textit{Cornus mas}) allotted less investment to their floral display and attractant than two other \textit{Cornus} species that flowered with or after leafing out, suggesting increased visibility in the leafless season compensated for this reduced investment. One possible approach to testing the pollination efficiency hypothesis would be to follow this comparative morphology approach for a broader range of hysteranthous and non-hysteranthous species. Another option would be to perform y-maze pollinator choice trials \citep{Giurfa1997} between closely related hysteranthous and non-hysteranthous taxa, or remove the leaves from non-hysteranthous species and compare their visitation rates to unaltered controls.\\
\indent A third functional hypothesis which we refer to as the "differential selection hypothesis" comes out of an application of life history theory. Phenological plasticity allows organisms to match life cycle events to the appropriate environmental conditions. For species in the temperate zones, the optimal timing for spring phenological events, such as flowering and leafout, is the product of a trade off between seasonal advance to maximize the length of the growing season, and delay to minimize exposure to last season frost events \citep{Kramer1995}. One possible explanation for hysteranthous flowering patterns is that these selection pressures operate upon foliar and floral phenology with different strength. It has been shown that long lived perennials such at trees and shrubs invest more heavily in growth and survival (foliar resources), rather than reproduction (floral resources) \citep{Franco2004}. It follows that for such organisms, floral tissues would be more expendable than leaf tissue. Frost damage to developing leaves has been shown to reduce gross primary productive by up 14 percent \citep{Hufkins2012}, while it is unlikely that sporadic pollination droughts from losing a cohort of flowers would make a significant difference in the lifetime fitness of long lived organisms such trees \citep{Knight2005}. This difference in value of these tissues could enable some species to employ a riskier strategy with floral phenology than leaf phenology. For these species, the benefits of early flowering, such as pollination efficiency or increasing fruit development and dispersal time \citep{Primack1987}, out weigh the risk of late season frost exposure. This is less true for the more critical leaf tissue, and thus a more conservative, delayed phenological strategy is employed \citep{Lenz2013}. This hypothesis posits that it is this differential selection pressure on flower and leaf timing produces the hysteranthous pattern.\\
\indent For all of these functional hypotheses, species must exhibit physiological independence between flowering and leafing, which is certainly not the case in all temperate woody species, such as species with floral phenologies constrained by the requirement to build flower tissue from the current years photosynthate, or species with flower buds contained within leaf buds. But there is also the possibility that hysteranthy itself is the product of biological constraints.\\
\subsection*{Physiological hypotheses}
\indent The predominant physiological hypothesis suggests that hysteranthous flowering is an adaptive resource partitioning strategy in areas prone to water stress. In such environments, species cannot maintain hydration in their flowers while transpiration is occurring through their leaves, so flowering is temporally allocated to the leafless season. This hypothesis is generally presented as an alternative to the insect visibility hypothesis of the dry tropics \citep{Franklin2016}. To our knowledge, it has not been widely discussion in the context of temperate forests but may explain hysteranthous flowering in temperate, insect pollinated taxa of more tropical origins.\\%should I say something about this being a big flower thing?%
\indent It is also possible that hysteranthy is a highly conserved trait, and the preponderance of this phenological pattern in the temperate zone is a product of phylogenetic representation of the region rather than an adaptive quality to the trait. In this paper, we explore the phylogenetic signal of hysteranthy in the eastern temperate forests of North America, but more work should be done to understand the distribution and evolutionary history of hysteranthy globally.\\
\indent None of these hypotheses are necessarily mutually exclusive, it is is certainly possible that hysteranthy has arisen multiple times in different selection environments. We should not presume that a single "just so" story could explain the adaptive significance of this complex phenological pattern.\\
\section*{Towards an Empirical Definition of Hysteranthy}
\indent Given the lack of explicit research attention in the literature, the most detailed descriptions of hysteranthous flowering come from regional Flora, botanical guidebooks and species monographs. In these sources, hysteranthy information is generally given as verbal description such as \textit{"Flowers: Before the leaves."} or \textit{"Flowers: When leaves are half grown."}\citep{Barnes2014}. %%% Add more sources?
 These kinds of qualitative verbal descriptors are inherently ambiguous and incompatible with our current, more quantitative observation standards such as the the BBCH scale \citep{Finn2007} displayed in the supplement. Does the description "Flowers before leaves"  mean that a plant's flower buds burst before the leaf buds? (bbch 55 before 09)? Does it mean that at least one flower is open before the first leaf begins to expand (bbch 60 before 15)? Does it mean peak flowering occurs before most of a tree's leaves are full size (bbch 65 before 19)? The answer to this question would radically change which species are categorized as hysteranthous. For example, using phenological observations from Harvard Forest from 1990-2015 \citep{OKeefe2015} we see that if our criteria for hysteranthous classification is flower budburst before leaf budburst, only three species in the community could be classified as hysteranthous, while if we use the criteria of flowers open before leave expansion reaches 75 percent of full size, most of the species community would be considered to be hysteranthous (see figure 1).  We suggest that an appropriate empirical definition of hysteranthy is largely dependent on which hypotheses are of interest. For the pollination efficiency hypotheses, we suggest that species that have open flowers during the early part of leaf expansion (bbch 60-65 before 15 or 17) should be considered hysteranthous. If researchers are more interested in the physiological hypotheses developed for the dry tropics, a more conservation definition, flowers between leaf drop and new leaf budburst (bbch 55-65 after and 97 before 09) would be more appropriate.%% Should I talk about continous evaluation rather than catagorical here?
 Because the focus of this paper is on temperate forests communities, we have elected to primarily adopt a functional definition of hysteranthy.\\

\section*{Predicting hysteranthy}
\indent While direct tests of each of the hysteranthy hypotheses should still be pursued, we can deduce much about the strength of the hypotheses by examining the relationship between hysteranthy and other relevant plant traits. We used published descriptions of hysteranthous species to model the association between hysteranthy and several other biological and phenological traits pertinent to the functional hypotheses of hysteranthy.\\
\indent For our analysis, we obtained species level descriptions of floral-foliate sequences from the regional guidebook Michigan Trees \citep{Barnes2004} and its companion volume Michigan Shrubs and Vines \citep{Barnes2016}, hereafter: MTSV. We investigated several other floras and monographs for inclusion in our analysis, but we found no other with comparably high levels of completeness of phenological descriptions. The complete list of sources we evaluated can be found in the Supplement. While MTSV describes woody plants found in Michigan, these communities bear a strong resemblance to forest communities of the Northeastern United States in generally, and can serve as a reasonable model for the whole region. \\
\indent We coded hysteranthy as a binary trait based on verbal phenological descriptions. In keeping with our functional definition of hysteranthy, Entries \textit{"flowers: before the leaves"}, \textit{"flowers: before or with leaves"} and textit{"flowers: with leaves"} were coded as hysteranthous while entries \textit{"flowers with or after leaves"} and \texit{"flowering after leaf development"} were coded as non-hysteranthous. Using the same data source, we obtained descriptions of several other traits that we determined to be biologically relevant to the hysteranthy hypotheses including pollination syndrome, maximum height, shade tolerance, and flowering and fruiting phenology. We coded pollination syndrome as binary trait (wind or animal pollinated). We also condensed verbal descriptions of shade tolerance to binary, collapsing descriptions "moderately, or medium shade tolerant", "tolerant" and "very tolerant" to "tolerant". In the text, flowering and fruiting phenology are described by a range of months. For both phenological entries, we calculated the average of the time span described, and coded it numerically in our dataset. In total, 194 woody species were included in our analysis. \\
\indent To investigated the phylogenetic signal of hysteranthy and control for phylogenetic structure in our dataset, we used a published angiosperm phylogenetic tree \citep{Zanne2013} pruned to match the species list from the MTSV data. Species found in the trait dataset but not in the original phylogenetic tree were added to the pruned tree at the genus level root. In total 32 species were added to the generic roots. To assess the phylogenetic structure in the trait of hysteranthy, we used Caper packaged \citep{Orme2013} to calculate a phylogenetic D statistic. To test the hypothesizes regarding the trait associations of hysteranthy, we used phylogenetic generalized linear modeling framework \citep{Ives2010} to build a logistical regression model corrected for phylogenetic structure using the R package phylolm \citep{Ho2014}.The model was run with 50 bootstrapped re-sampling iterations for each dataset. Continuous predictors were centered and re-scaled by subtracting the mean and dividing by two standard deviations to allow for a reasonable comparison of effect sizes between the binary and continuous predictors in this model \citep{Gelman2007}.\\
\indent To illustrate that the our analysis is sensitive to how hysteranthy is defined, we also built a model using a physiological definition of hysteranthy in which only the descriptor "flowers: before leaves" was coded as hysteranthous and all other descriptors were coded as non-hysteranthous. The results from this model can be found in the Supplement.\\
\indent Our primary analysis suggested that 101 out of 194 species should be classified as hysteranthous (see figure 2). We found that the phylogenetic signal for hysteranthy was low- the D statistic for the trait in the MTSV data was 0.06. Average timing of flowering was the strongest predictor of hysteranthy, with the likelihood of hysteranthy increasing substantially with earlier flowering. Pollination syndrome also had a strong effect, with the likelihood of hysteranthy increasing in wind pollinated taxa (see figure 3). None of the other predictors included in the model has substantial effects.\\
\indent One challenge to interpreting these results is that since hysteranthous species must flower before their leaves, they can never flower late in the season. This begs the question, do hysteranthous species indeed flower earlier than early flowering non-hysteranthous species? To address this question we re-ran our model on a restricted dataset which only included species that flowered between mid-March and mid-May. We found that even among early flowering species only, the likelihood of hysteranthy still increased substantially with earlier flowering (see figure 3b).\\ 
\indent The large effect size of pollination syndrome on the likelihood of hysteranthy gives credence to wind pollination efficiency hypothesis, and supports the long held observation that hysteranthy is associated with this pollination syndrome in the temperate zone. The substantial effect of earlier seasonal flowering on the likelihood of hysteranthy observed in both our full and restricted data set provide support for the differential selection hypothesis in demonstrating that the earliest flowering species are indeed hysteranthous.  Using average predictive comparisons, we find that with all other traits equal, a species that is wind pollinated is 38 percent more likely to be hysteranthous than an insect pollinated species and a species flowering in April is 22 percent more likely to be hysteranthous than one flowering in May. These findings suggest that hysteranthy is indeed associated with wind pollination and/or extremely early flowering in the temperate deciduous forest of eastern North America. %average predictive comparison are with uncentered predictors. ask Lizzie about this%

\section*{Hysteranthy and Climate Change}
\indent The pollination efficiency hypotheses of hysteranthy that seems to predominate the literature and find support in our model suggest that the leafless period of flowering is critical for the reproductive success of hysteranthous species. With many reported cases of phenological shifts in plants due anthropogenic climate changes, it is certainly possible that hysteranthous flowering pattern may be altered by changing seasonal conditions. Any substantial shifts in timing or duration of the hysteranthous period could have significant effects on the reproductive success of these species. As stated above, foliar and floral phenology both respond to complex interactions between cold winter vernalization temperatures, warm spring forcing temperatures, and day length. However, we have little understanding of the comparative strength of the cues on different phenophases such as flowering and leafing within one organism. Several studies have found temperature to be the driving phenological cue for leaf out and classify many woody species as "photoperiod insensitive" with regard to their foliar phenology \citep{Basler2012}. At the same time, much of the classic work with regards to flowering suggests that the importance of photoperiod (long day/short day) in determining flowering phenology {Glover2007}. With global climate change, both winter and spring temperature are projected to be warmer on average while photoperiod will remain unchanged. If floral and foliate phenologies respond with proportionate sensitivity to these cues, we many observe overall shifts on phenology but the relative timing between the phenophases would be maintained (figure 3, Scenario 1). However,if flowering and leafing are deferentially sensitive to these cues, we would expect that the timing relative to the other many shift. This could result in an extension or contraction (figure 3, Scenario 2a,2b) of the hysteranthous period, or even a loss of it entirely (figure 3, Scenario 3). A contraction or loss of the hysteranthous period could result increase pollen limitation or restrict pollen dispersal, endangering the long term  demographic viability of hysteranthous species.\\
\indent Climate change is also likely to affect the life history trade off inherent to the differential selection hypothesis, which also found support in our analysis. It has been shown that warming spring temperatures tend advance flowering, though this effect may be muted by inverse effect of warmer winters \citep{Cook2012}. Advancing phenology and changing spring climate patterns could push hysteranthous flowering deeper into the unstable climate period of late winter and early spring, and may increase the risk of exposure to late season frost. If the return interval of late season frost damage become more frequent or severe, the reward of early flowering in frost free years may no longer offset the detrimental effects of years with frost episodes. This may give more conservative, later flowering species a demographic advantage over the hysteranthous opportunists.\\
\indent With the current existing body of research, we cannot assess the likelihood and impact of these scenarios. Given the theorized importance of hysteranthous flowering for the reproductive success of many important temperate forest trees, and the potential for the disruption to this pattern due to global climate change, it is clear that future phenological and climate change research must pay attention to hysteranthous flowering and the inter-phenophase constraints in general. Below, we outline several research directions that should be pursued to better understand the role of this phenological trait.
\section*{Future Directions}

\indent \textbf{Patterns of hysteranthy.} Because long term data records that report both floral and foliate phenology from the same individuals are relatively rare \citep{Wolkovich2014} our analysis relied on verbal descriptions of floral-foliate sequences to classify woody plant species which are by nature imprecise. From such data, we cannot know if differences between descriptions found is separate sources are a reflection of observer bias stemming from authors different metrics for defining hysteranthy (see section II), or reflect true temporal or population differences in degree of hysteranthy. Future phenological observations should make a strong effort incorporate both floral and foliate observations, and hysteranthy should evaluated as a continuous description rather than a catagorical trait. For example, rather than hyster such as "flowers before leaves" in the future, we hope that descriptions could read "over a ten year period, flowers opened on average 12 days before leaves expanded". These more precise descriptions would allow researchers to address many important questions about hysteranthy such as: What is the inter-annual reaction norm of the degree of hysteranthy in individuals? Are there significant population differences in hysteranthy, and do any geographic patterns emerge? Such studies would clarify the extent of the hysteranthous flowering, and may better reveal the selective forces structuring the floral-foliate phenological sequence in woody plants.\\
\indent \textbf{Direct tests for functional hypotheses.}The results of our study provide suggestive evidence supporting multiple hypothesis regarding the adaptive benefit of hysteranthous flowering.  In our discussion of the hypotheses in section 1, we suggested several research approaches that would allow for these hypotheses to be tested more directly. Such explicit tests of these hypotheses should be undertaken by scientists.\\
\indent \textbf{Independence and Constraint between phenophases.} The looming uncertainty about how global climate change may alter the hysteranthous phenological syndrome is underlain by a more fundamental question: to what degree are floral and foliate phenophases constrained by, or independent of each other? Are floral and foliate phenological responses differentially sensitive to different environmental cues? Observational studies, correlating variability in hysteranthy with environmental cues, and experimental studies in which cues are directly manipulated would both be useful to address the questions.\\
\textbf{Hysteranthy and Masting.} It has been well documented that reproduction in long lived organisms is highly variable. Significant inter-annual variation in reproductive investment and output \citep{Bogdziewicz2017} is well documented in the literature, and this masting is especially prevalent in wind-pollinated taxa \citep{Kelly2001}. How does the variability in hysteranthy interact with this variability in reproductive output? Does the pollination gain suggested in the pollination efficiency hypotheses contribute to mast years? Observational studies should investigate the association between mast years and hysteranthy to help refine our understanding of how phenology may influence this important ecological phenomenon.\\
\indent \textbf{Hysteranthy in Other Systems.} Our analysis found that hysteranthous flowering is indeed widespread in the temperature forests of the Eastern United States. However, hysteranthy has been reported in other ecosystems, and should be thoroughly investigated in other regions and habitats. In particular, attention should be paid to the dry deciduous forest of Central America where hysteranthy has been reported to be common \citep{Janzen1967}, but hysteranthy should be explored in other deciduous ecosystems as well.\\
\section*{Summary}
\indent Hysteranthy, is an understudied and poorly defined phenological trait that may be critical for the reproductive success of many important woody plant species. In our analysis of ~200 woody plant species of the Eastern United States, we found that hysteranthy is common, lacks phylgenetic structure, and is associated with a wind pollination syndrome and extremely early flowering. If hysteranthy is indeed essential for reproductive success of many species, shifted in the hysteranthous period caused by global climate change could have negative impacts on species' fitness, and as such, it is essential that future phenological research investigate this trait more thoroughly.
\section*{Figures}

\bibliography{..//refs/hyster.manuscript}



\end{document}
