\documentclass[12pt]{article}
\usepackage[]{graphicx}
\usepackage[]{color}
\usepackage{alltt}
\usepackage[top=1.00in, bottom=1.0in, left=1.1in, right=1.1in]{geometry}
\renewcommand{\baselinestretch}{1.1}
\usepackage{graphicx}
\usepackage{natbib}
\usepackage{amsmath}
\bibliographystyle{..//refs/styles/besjournals.bst}
\def\labelitemi{--}
\parindent=24pt
\usepackage{lineno}
\linenumbers

\begin{document}
\section*{Introduction}
\indent \indent Phenology, the timing of seasonal life cycle events, allows organisms to synchronize important life history transitions with optimum environmental conditions \citep{Forrest2010}, and is a critical component of ecosystem structure and function \citep{Cleland2007,Piao2007}. Recent work in woody plant phenology has shown that it is not only individual phenological stages that affect these processes, but also the relationships between them \citep{Ettinger2018}.\\

\indent One phenological relationship that has long received scientific interest \citep[see][]{Robertson1895}, and recently, increased attention in the literature \citep{Savage2019, Gougherty2018} is the flower-leaf phenological sequence (FLS) of deciduous woody plants. In a typical model of plant life-history, vegetative growth precedes reproduction. However, for many species in the forests of Eastern North America, it is not the green tips of new shoots that mark the commencement of the growing season, but the subtle reds and yellows of their flowers. This flowering-first FLS is common in these regions, and its prevalence suggests that this FLS has adaptive significance \citep{Rathcke_1985}.\\ 

% EMW: Can you add some numbers? For example, after ' as a result of climate change' you could at 'at a rate of up to XX for some species' or you could say how much the rate differs between species? Two-fold?
\indent Understanding this phenological pattern is particularly necessary and timely now because anthropogenic climate change is altering FLSs (Fig. \ref{fig:Figure 1}). Long-term observations show the number of days between flowering and leafout is increasing as a result of climate change, but the rate of change differs among species (Fig. \ref{fig:Figure 1}).  If FLSs are indeed an important component of woody plant fitness, this inter-specific variation will exacerbate fitness differences between species, influencing which species will persist under altered climate conditions. These long-term data showing shifts in FLSs over time also highlights how variable---within species---FLS are. \\

\indent Despite recent advances in understanding the physiology and evolution of FLS \citep{Gougherty2018,Savage2019}, most research has ignored this variability---potentially styming efforts to predict how FLS patterns will respond to climate change. While some authors present general correlations between flowering and leafing phenology \citep{Lechowicz_1995, Ettinger2018}, fine-scale FLS variability has never been evaluated. We suggest that characterizing FLS variation among individuals and populations will not only improve our ability to predict how FLS patterns will change in the future, but also allow for a more biologically relevant evaluation of the current FLS hypotheses, revealing avenues for future, direct hypothesis testing.\\

\indent Here we 1) Review the hypotheses of woody plant FLSs and their respective predictions, 2) Evaluate variation in FLSs, and explore how FLS variation within species, populations and individuals alters the predictions of the hypotheses, 3) Show how the incorporation of variation reveals consistencies and anomalies in support for FLS hypotheses using several case studies from temperate forests, and 4) make recommendations to improve future study of FLSs. 

\section*{Hypotheses of FLS}
\subsubsection*{ Wind pollination}
\indent\indent The most prevalent FLS hypothesis suggests that hysteranthy is an adaptation critical for effective wind pollination, with leafless flowering allowing for more efficient pollen dispersal and transfer \citep{Whitehead1969, Spurr1980,Friedman2009}. The primary evidence for this hypotheses comes from pollen diffusion studies \citep[e.g., particle movement through closed and open canopies][]{Niklas1985,Nathan2005, Milleron2012} and suggest pollen movement is indeed incombered by the maturing leaves, but exactly at what stage in the canopy delopment this effect becomes substantial has not be addressed.

\subsubsection*{Water dynamics}
\indent\indent Another hypothesis, emerging from the dry-deciduous tropics where flowering during the leafless season is also common \citep{Janzen1967}, suggests that flowering before leaf development is an adaptation to reduce water stress associated with maintaining floral hydration while leaves are transpiring \citep{Franklin2016}. We are unaware of any studies that have mechanistically evaluated the water dynamics hypothesis, though observations of flowering in the dry tropics suggest that the timing of flowering in hysteranthous taxa is associated with a plant water status recovery due to leaf drop \citep{Borchert1983,Reich1984}. Only recently has it even been suggested that this hypothesis might be relevant in the temperate zone as well, as springs in the temperate zone are generally wet to the point that do limit biological activity \citep{Gougherty2018}.\\ % EMW: Need citation for last point.
\subsubsection*{Early flowering}
\indent\indent A third possibility is that the flowering-first FLS is a physiological byproduct of selection for early flowering \citep{Primack1987}. Within this framework, there is no advantage to a species being hysteranthous vs. seranthous, as long as the absolute flowering time is the same. We are aware of no direct tests that have tried and distinguish selection for hysteranthy from selection for early flowering, but \citet{Primack1987} notes that hysteranthous, wind-pollinated species tend to also have large seed mass, and lack primary seed dormancy for germination, traits associated with early flowering in general. This raises the distinct possibility that hysteranthy may simply be one component of a larger suite of early flowering traits. Recent work from \citet{Savage2019} has demonstrated that spring flower phenology is less constrained by prior phenological events than leaf phenology, which would allow selection to drive flowering into the early season, producing the hysteranthous FLS. This might explain why hysteranthous species tend to be the earliest species to flower (Fig. \ref{fig:Figure 2}). 

\subsubsection*{Phylogenetics} 
\indent\indent Finally, it is also possible that FLSs are highly conserved traits, and the preponderance of hysteranthy in the temperate zone is a product of phylogenetic representation of the region rather than an adaptive aspect of the trait. In a recent analysis, \citet{Gougherty2018} found strong phylgenetic clustering.

Despite decades of inquiry, only limited progress has been made towards any consensus for these hypotheses.. Many studies only test a single hypothesis, making comparison between them difficult. In contrast, studies testing multiple hypotheses have generally found support for more than one evolutionary driver of hysteranthy\citep[see][]{Bolmgren2003,Gougherty2018}.

TRANSITION-Framework, how FLS are defined hold us back because, as we will suggest is disconnected romt eh biology underlying these hypotheses

\section*{ Difficulties in defining FLS}
% EMW: Much better! Though I still adjusted it and made some notes. I also wondered if this section might fit well within a box if the journal takes boxes....
\indent\indent Flower-leaf sequences have traditionally been classified into qualitative categories that are almost always defined at the species level. The terms hysteranthy, protanthy, proteranthy or precocious flowering describe species that produce flowers before their leaves. Synanthy describes species whose flowering period overlaps their leaf development and seranthy describes species whose flowers open after their leaves emerge \citep{Lamont2011, Heinig1899}. But applying these conceptual categories to real phenological sequences is not always so straight-forward.\\

\indent Both reproductive and vegetative phenological sequences consist of multiple sub-stages, and this introduces significant ambiguity into how we interpret qualitative FLS descriptions. Consider a species with the following FLS:\\
\begin{center}
\textbf{flower budburst}\rightarrow \textbf{leaf budburst}\rightarrow \textbf{first flowers open} \rightarrow \textbf{leafout} \rightarrow \textbf{peak flowering} \rightarrow \textbf{end of leaf expansion}\\
\end{center}

% EMW: Dan, please check my edits at the end of paragraph (would bre great to give example if possible) -- I think what 'different sources' are may not be clear so added some lead-up for readers. 
\noindent Phenological observers could justifiably classify this species as: 1) Hysteranthous because flower budburst proceeds leaf budburst, 2) Synanthous because flowers open during the time between leaf budburst and leafout or 3) Seranthous because peak flowering occurs after leafout. This problem extends beyond this simple example to real datasets, such as the long term phenological records from Harvard Forest in Petersham, MA \citep{OKeefe2015} where the same ambiguities exist (Fig. \ref{fig:Figure 2}). Not surprisingly then, different sources (e.g., floras) may classify the same species as synathous or hysteranthous. For example, we compared species-level FLS descriptions in two of the most comprehensive records of FLS, Michigan Trees and its companion volume Michigan Shrubs and Vines (MTSV) \citep{Barnes2004,Barnes2016} with The USFS Silvics manual volume II \citep{Burns1990}. Of these 49 amount of species 30\% were classified differently. Such different classifications could reflect interesting temporal or geographic variability in FLSs, but---given current definitions---they could equally be an artifact of observer decision-making.\\

\indent Given that the most complete FLS datasets currently come from regional guide books and floras like the ones mentioned above, these categories were most likely originally described to aid with plant identification rather than to describe functional biological processes like those underpinning the hypotheses may suggest different boundries. For example, the wind-pollination hypothesis hinges on the fact that leaves create a substantial physical disruption to pollen transfer, a premise that we would not necessarily expect to be true for the early stages of leaf expansion when tiny leaf primordia would have little impact on environmental structure. Rather we expect that trees that flower during the early stages of leaf expansion would gain similar mechanical advantage to those who complete their flowering before any leaf activity. Alternatively, because transpiration intensifies as soon as leaves begin to expand in th spring \citep{Breda1996,Wang2018}, the water dynamics hypothesis would assert there is a significant cost to maintaining floral structures during any stage of leaf activity and only species whose flowering occurs before any leaf expansion would gain a drought advantage. It follows a statistical relation between FLS and pollination sydrome is more likely if hysteranthy and synanthy are collapsed into a single catagory, and a relationship between FLS and water dynamics would be more likely when synanthy and seranthy are combined. Such categorization can often introduce biases in analyses (% EMW: I would cite the PNAS correction by Gelman below. ) 
and in this case, a reseacher looking for an association between FLS and pollination syndrome would be more likely to find it given one catagorization scheme over the other.  
\indent In the case of FLS, any statistical relationship between FLS and other traits predicted by the hypotheses is biased by the subjectivity of the original observer, the modeler, and the possibility that the associations we are testing may not reflect the biology processes that shape FLS. To illustrate this, we compared trait associations with hysteranthy between the the MTSV and USFS datasets. To address how FLS definitions may bias certain hypotheses we applied two alternative FLS classification schemes; physiological hysteranthy, which allowed for no overlap between floral and leaf phenophases, and functional hysteranthy, which allowed for a degree of overlap. We found systematic differences between the datasets which differed in how FLS was catagorized and species included in each. As predicted, we also detected catagorization bias with catagorization that matched the hypothesized mechanism being better. Thus a researcher using our MTSV dataset with functional FLS definitions would find strong support for the wind pollination hypothesis but one using USFS physiological data would not. \\

Catagorization obscures a second reality about FLS that may be the key.
\indent We find that there is substantial intra-specific differences in FLS, and this variation has become even more obvious as climate changes (Fig. \ref{fig:Figure 1}).Yet, FLS categories are always applied at the species level, and intra-specific variation has never been broadly assessed \citep{Gougherty2018}. Intra-specific variation is the engine of natural selection, and if it is substantial in FLS patterns, we can infer much about the origins of this trait as well as its trajectory as the climate changes. 

\section*{Variation in FLS}
 \indent\indent We investigated individual FLS variation in both the PEP725 \citep{PEP725}  Harvard Forest data \citep{OKeefe2015}, and found that the time between flowering and leaf activity varied by as much as several weeks for individual species between years and sites. This variability is lost completely in the classic framework of categorization. For example, for  \textit{Q. rubra}, a species classically listed as flowering and leafing in synanthy, there are some years in which flower budburst is more than a week before leaf budburst, and other years in which leaf buds burst weeks prior to floral budburst (Fig. \ref{fig: Figure 3}). 

 \indent Given the variability of FLSs at the individual and population level, it is clear that considering FLS variability at only higher taxonomic levels obscures important realities about the biology of this phenological trait. Below, we discuss how the observed variation below the species level may alter the existing FLS hypotheses.

\subsection*{How FLS variation alters predictions}
\subsubsection*{Wind pollination} 
\indent\indent  Pollination syndrome is generally treated as a species-level trait, considered to be fairly immutable across ecological time and space. Because of this, we would not expect significant variation in FLS across populations or individuals because we would not expect variation in pollination syndrome. However, as discussed above, a tree with no overlap between flowering and leafing phenology does not necessarily gain a significant pollen transfer advantage over an individual with some overlap. The pollination efficiency advantage from flowering-first should diminish as the canopy fills in, but  we do not know at what point during leaf expansion pollination would become significantly encumbered. It is possible that interannual and population-level variation in the timing between flowering and leaf out for hysteranthous and synanthous individuals could maintain a wind pollination advantage, as long as the overlap did not cross a certain unknown threshold. Therefore, based on the wind pollination efficiency hypothesis, we would not expect high levels of population or individual variation in FLS, but the detection of some FLS variability at these levels does not inherently challenge this hypothesis.
\subsubsection*{Water dynamics} 
\indent\indent If FLS's are driven by water dynamics, we would expect there to be significant population-level variation in FLSs. Populations growing in drier habitats should flower earlier relative to their leaf activity than their counterparts growing in wetter areas that experience weaker selection for minimizing phenological overlap. Therefore, increased time between flowering and leafing should be negatively correlated with average soil moisture. Water availability may also drive interannual FLS variation, with drought years increasing hysteranthy, and wetter years permitting more FLS overlap. 
\subsubsection*{Early flowering} 
\indent\indent This hypothesis predicts some variation on the population level based on local adaptation.  We would expect populations in which selection for earlier phenology is stronger, perhaps those in regions with shorter growing seasons, to flower earlier relative to their leaf development.  At the individual level, FLS variability could be driven by interannual variability in spring conditions. Both flowering and leaf phenology are strongly cued to temperature and photoperiod \citep{Flynn2018,Rathcke_1985}, but with leaf phenology constrained by xylem activity and flowering phenology relatively independent of it, we would expect a more sensitive response to environment in flowering time resulting in FLS variation. This hypothesis predicts that early flowering years or populations should be associated with an increase in the time between flowering and leafing for hysteranthous species. It also predicts a tighter temporal correlation between flowering and leafing for seranthous species or those with mixed buds in which flower timing is constrained by leaf budburst.
\subsubsection*{Phylogenetics} 
\indent\indent With the lack of treatment of intra-specific FLS variability in the literature, we have no strong basis for asserting whether the apparent variability in FLSs is a product of genetic or environmental controls. If there is a strong genetic component to FLS as has been show for other phenophases \citep{Wilczek2010}, some population-level variation could be driven by reproductive isolation. With strong genetic control of FLS, we might also see consistent genotypic differences in FLS among individuals within a population, but would not predict high levels of interannual variation.\\

\indent In all of these cases, variability in FLS below the species-level was not addressed. Yet, there are datasets widely available that allow for testing these several hysteranthy hypotheses concurrently, and at multiple taxonomic levels. To address this gap, we supplement our literature review with several analyses.

\indentThe Harvard Forest data set (HF) contains quantitative flowering and leaf phenology measurements for individuals of 24 woody species over a 15 year period \citep{OKeefe2015}. From the Pan European Phenological Database (PEP725) \citep{PEP725} we obtained spatially and temporally explicit, quantitative flowering and leaf phenology for four common European tree species. The HF data are temporally explicit, allowing for both inter- and intra-specific FLS comparisons. The PEP725 data is species-limited, and allows us to evaluate FLSs only at the intra-specific level, but permits us to address variability in individuals over time and among population.\\

\indent We used our intra-specific datasets to compare to test some of the predictions about intra-specific variability in the water dynamics and early flower hypotheses. %Contrary to our prediction, we found that dry years correlate with a decrease in time between flowering and leafing for hysteranthous species, largely due to delayed flowering. Taking this out because I am not confident the analysis is good.
When we examined the relationship between 30 year soil moisture records \citep{DWD} and population level variation in FLS timing across Germany, we found a weak negative association between average soil moisture levels and time between flowering and leafing as predicted by the water dynamics hypothesis. However, when we incorporated other predictors, such as flowering time into our analysis, the association disappeared (Fig. \ref{fig:Figure 4}, PEP725 estimates). This suggests that FLS variation at this scale may be primarily driven by flowering time rather than water availability. \\ 

% EMW: How many species in this analysis?
\indent In accordance with our predictions for the early flowering hypothesis, we found that for hysteranthous species, FLS variation is much more tightly correlated with variation in flowering timing than in leafing timing, but this contrast is far less stark in seranthous \textit{Aesculus hippocastum} (table \ref{tab:Table S2}). Though our intra-specific data set is species limited, plasticity in the first phenophase of the season (flowering for hysteranthous species and leafing for seranthous species) appears to drive variability in FLSs, but this observation should be tested more rigorously and explicitly in future work. While the inter- and intra-specific case studies are not perfectly comparable (i.e., the wind pollination hypothesis cannot be evaluated on the intra-specific level), the general insights from our intra-specific studies supports the relationships found in the inter-specific case studies and provide novel, higher resolution insights of their own.

\subsection*{Future}
\indent\indent Each of our case studies provided its own insights into the nature of the relationship between FLS variation and the FLS hypotheses for woody species. For MTSV and USFS, we found that the strength of each predictor's effect varied depending on how the FLSs were defined. From the HF study, we found that re-defining  continuous FLS as binary masked important species level variation in trait associations and from PEP725, we identified a new hypothesis for the physiology behind FLS; that FLS variation is generally driven by variation in the first phenophase of the sequence. However, considering the results of the cased studies together yielded a more comprehensive picture of where our understanding of this phenological trait is currently, and where it needs to go. Below we highlight five characteristics of FLS that we suggest should be incorporated into future research. %EMW: Why should they be incorporated?

\subsubsection*{Multiple hypotheses explain FLSs}

\indent\indent Our results underscore other lines of evidence that show multiple hypotheses should be starting point for all future FLS research. While there is certainly value to broad taxonomic studies, and future large-scale analyses should continue, the consistent support for multiple hypotheses shows there are limits to the utility of these kinds of studies. We suggest that it is better to explore the evolutionary dynamics of hysteranthy with a more mechanistic approach, which may mean utilizing a more taxonomically-restricted focus. The significance of interaction terms in some of our models suggest that a promising option is to look within the hypotheses to address sub-grouping of taxa in which overlap between hypotheses could be controlled. For example, we know that wind-pollination efficiency is not driving hysteranthous flowering among biotically-pollinated taxa, so if we consider this group of species alone, we may be able to detect stronger signals from other traits that support other competing hypotheses. Incorporating a more explicit phylo-biogeographic approach would be instructive at this level; if there are phylo-geographic commonalities between the few biotically-pollinated hysteranthous species in Eastern flora, we might better understand the function of FLS variation in these species by investigating FLS variation in their sister-taxa in their regions of origin. Even with focused work on sub-groupings of species, interspecific trait-association models can only can take us so far. \\

% EMW: Need more citations below, stuff by Violle and Craine maybe? The question of what is the 'right' trait is very well trodden ground. 
\indent As in most other areas of plant biology examining traits, research is hampered by the difficulty of knowing which are the right traits. For example, we used minimum precipitation across a species' range, one of the only available quantitative drought metrics at the scale of large inter-specific models, to represent the water dynamics hypothesis but we have no way of knowing for certain that this is really a good proxy for drought tolerance. Further, species evolve a suite of traits for any function, and unmeasured traits might bias our results \citep{Davies2019}. For example, wind-pollinated species could compensate for pollen intercepted by a synanthous or seranthous FLS by over-producing pollen or through self-pollination. To really understand FLS across large taxonomic space, one would have to compare species across an unfeasibly large, N-dimensional trait space, suggesting we will need to utilize other, complementary approaches, detailed below. % EMW: Nice transition.

%EMW: Below is generally very nice!
\subsubsection*{Intra-specific variation in FLS}
\indent\indent In this paper, we have shown that FLSs can be highly variable at the intra-specific level. This variation can be leveraged through carefully designed research to overcome many of the limitations of larger trait-correlation models. Unlike with inter-specific approaches, focusing on FLS variation within species holds most other traits relatively equal, avoiding the problem of tradeoffs with latent unmeasured traits. Evolutionary theory traditionally predicts that intra-specific variation should follow the same trends as inter-specific variation, and consistent agreement between inter- and intra-specific, as we found in our analysis, will help narrow in on certain hypotheses.\\ % 'Evolutionary theory predicts that intra-specific variation should follow the same trends as inter-specific variation' ... needs citation, this is true GIVEN assumptions, so you need a citation to show what you mean. 
\subsubsection*{The FLS is a quantitative trait}
\indent \indent Treating FLS observations as continuous variables is a far more accurate and useful way to describe phenological sequence data. Our modeling work shows that this is an important step towards reducing observer bias and revealing important inter-specific differences that are masked by categorization. Quantitative measure of phenology \citep[e.g. the BBCH scale,][]{Finn2007}, standardize data across time and space, observer, and analyst. Adopting such measurements in the study of phenological sequences would allow for FLS patterns to be compared across larger temporal, geographic and taxonomic scales, giving researchers more power to accurately address questions about FLS variation.
\subsubsection*{FLS and fitness}
\indent\indent While trait associations point to past selection, fitness is the driver of trait evolution, and at the core of each FLS hypothesis is a fitness prediction. By utilizing intra-specific comparisons and continuous measurements of FLS, we can move beyond trait associations and test the fitness consequences of FLS variation. \\
\indent Variability in hysteranthy should lead to variability into fitness outcome at the intra-specific level. For example, the wind pollination hypothesis predicts that with all else equal, years with increased time between flowering and leafing should correlate with more pollination success. The water dynamics hypothesis suggests hysteranthous populations with a consistently larger time between flowering and leafing should better tolerate drought. These predictions could be directly assessed through well-designed experiments and field studies.\\
\subsubsection*{FLS and physiology} 
\indent\indent Decades of research suggests that both floral and vegetative phenological events are cued by temperature and photoperiod \citep{Forrest2010, Flynn2018}, suggesting they are under shared genetic and physiological control. But to yield the FLS variation seen in nature, there must be systematic differences in reproductive and vegetative phenological responses to the environment. Researchers can use intra-specific variation in FLS to identify which cues dominate each phenological process and better understand the underlying genetic and physiological constraints that structure phenological sequences.\\

\noindent \emph{Conclusions:} Our proposed framework provides a path to understand the drivers of FLSs in woody plants. Through examining FLS variation in more targeted taxonomic assemblages and using quantitative data with mechanistic metrics, we can refine the existing FLS hypotheses and better comprehend the causes and consequences of FLS variation at multiple taxonomic scales. This is an essential step towards a more complete understanding of the fundamental biology of temperate woody plants, and for predicting the fate of these species as global climate continues to change.

\end{document}