\documentclass[10.5pt,a4paper]{article}
\usepackage[top=.75in, bottom=.75in, left=.7in, right=0.7in]{geometry}
\usepackage{graphicx}
\usepackage{natbib}
\usepackage{gensymb}
\begin{footnotesize}
\address{1300 Centre Street \\ Boston, MA, 20131}
\end{footnotesize}
\begin{document}
\bibliographystyle{..//refs/styles/nature.bst}
\def\labelitemi{--}
\parindent=24pt
\includegraphics[width=0.3\textwidth]{/Users/danielbuonaiuto/Desktop/arb_logo.png}
\pagenumbering{gobble}
\opening{Dear Dr. Hetherington,}
\par We propose a ``Viewpoint" about the drivers of flower-leaf phenological sequences (FLSs) in decidouous woody plants. Evolutionary theory predicts that both flowering and leaf phenology are critical fitness componets of woody plants, and a century of empirical research supports this assertion \citep{}. In recent decades, this theory has been extended to suggest that it is not only these indivudal phenophases but also the relationship between them that determines woody plant fitness \citep{}. Many deciduous woody plants flower before leafing, yet sustained research efforts have yet to yield a well-supported explanation for this. These unresolved hypotheses are critically important now as climate change is shifting FLS---with the sequences of some species being reversed with warming. Our ``Viewpoint" shows how progress in this area has been stalled by our conceptual framework for FLSs; we detail a new approach built on continuous measures of FLS and intra-specific and within-individual-level variation to rapidly advance progress.\\

\noindent \emph{What hypotheses or questions does this work address?}\\
Studies have variously suggested that flowering before leafing may be an adaptation for wind-pollination \citep{}, for reducing water stress \citep{}, or to facilitate extreme early season flowering \citep{}. Studies that directly compare these hypotheses, however, are rare, and those that do, tend to find support for more than one \citep{}. While FLS patterns are usually treated as qualitative descriptors at the species level, for example, \textit{``flowers emerge before leaves"}, we argue that a novel approach focusing on intra-specific FLS variation and quantitative inter-specific comparisons is necessary to accurately evaluate these hypotheses. \\

\noindent \emph{How does this work advance our current understanding of plant science?}\\
In our ``Viewpoint", we would 1) Review the hypotheses of woody plant FLSs 2) Comprehensively evaluate FLS variation across and within species, and explore how this variation alters the predictions of the hypotheses, 3) Evalate the FLS hypotheses using several case studies from temperate forests, and 4)
make recommendations for future study of FLSs [to some important aim/goal]. \\
Through this novel approach, we show that 1) There are high levels of both inter- and intra- specific in FLS that cannot be accomidated in the current FLS framework and that this obscures our ability to effectively test and differentate between hypotheses, 2) This variation provides novel insights about the nature of FLS and reveals consistencies and anomalies in support for FLS hypotheses, and 3) leveraging intra-specific varation in phenological research provides an avenue forward to advance our understanding of FLS.\\

\noindent \emph{Why is this work important and timely?}\\
Adopting our new framework for understanding FLS is key to predicting how species will impacted by global climate change. From long term data we can see that climate change is alterning FLS patterns, but that these effects vary across speicies and populations. If, as suggested, optimum FLS patterning is an important component of fitness, this differential FLS sensitivity to climate change may influence which species will persist under altered climate conditions.


\end{document}