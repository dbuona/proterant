%Scratch from NEw PHyt manuscript
%old abstract
%old abstract Phenology is a major component of an organism's fitness. While individual phenological events affect fitness, growing evidence suggests that the relationship between events may be equally or more important. This may explain why deciduous woody plants exhibit considerable variation in the order of reproductive and vegetative events, or flower-leaf sequences (FLSs). Research suggests that FLSs are adaptive, with several competing hypotheses to explain their function. Here, we present a new framework for the study of FLSs based on FLS variation at multiple taxonomic scales and demonstrate how it can help reconcile the existing hypotheses using case studies from temperate forests. Using this framework, we found concurrent support for multiple hypotheses that reflect the complicated history of migration and community assembly in the temperate zone. We highlight how adopting a quantitative, intra-specific approach generates new hypotheses and avenues for evaluating fitness consequences of FLS variation and provides cascading benefits to improving predictions of how climate change will alter FLSs and thereby re-shape plant communities and ecosystems.
% EMW (2Mar): Are we under a word limit here that makes this abstract feel squished? I feel like we lost some of the good bits from our submitted proposal. Below adds them back in, see what you think edit as needed. Basically my big concern with the above is that 'Here, we present a new framework for the study of FLSs based on FLS variation at multiple taxonomic scales and demonstrate how it can help reconcile the existing hypotheses using detailed case studies from temperate forests.' is pretty verbose and the former abstract was not. Can we simplify? Also keep wording consistent -- if you are offering a 'new framework' then the next sentence should be 'using this framework' or such ... 
%DB 4Mar. Below this switch works for me. I made some edits to more accuraretly reflect what we actually do in the paper

% EMW (2Mar): check my edits above, not clear what 'high-resolution dynamics of this variability' really means/refers to
% EMW (2Mar): Also, I think you need to pick between 'novel avenues' and 'direct hypothesis testing' %DB i chose novel avenues  
% EMW (2Mar): Could the last sentence be: There is no expectation that the FLS hypotheses must be mutually exclusive. Indeed, understanding the relative importance of each one and the relationships between them may provide the most useful path forward, if they can be robustly compared. DB YES!

% EMW (2Mar): Check edits above.


% EMW (2Mar): What about one section? Inter- and intra-specific variation in the current framework sure

%One of the challenges of inter-specific comparisons that may be mitigated by an intra-specific approach is that species evolve a suite of traits for any function, and unmeasured traits might bias results \citep{Davies2019}. For example, wind-pollinated species could compensate for pollen intercepted by a seranthous FLS by over-producing pollen or through self-pollination, and omitting such trade-offs from analyses would bias inter-specific comparisons. Focusing on FLS variation within species holds most other traits relatively equal,  

%For example, there is some evidence that hysteranthous species invest less in floral resources than seranthous sister taxa \citep{Gunatilleke1984}, suggesting that perhaps hysteranthous flowering increases visability to pollinators \citep{Janzen1967}. 
%\noindent  It is possible that hysteranthy in these taxa is maintained because there is no strong selection against it,  Hysteranthous flowering may expand species' temporal niches and partition inter-specific competition for pollinators, or it may be a useful tool for attracting them \citep{Janzen1967}. For example, there is some evidence that hysteranthous species invest less in floral resources than seranthous sister taxa \citep{Gunatilleke1984}, suggesting that perhaps their flowers are easier for pollinators to locate. While , they are readily testable under the new framework.\\

%\noindent Given the the emergence of support for the water dynamics hypothesis under the modified FLS framework, we also tested one of the major intra-specific predictions of the water dynamics hypothesis. The hypothesis predicts that drier years should be associated with a increased time between phenophases. As predicted, we found a negative association between FLS variation and annual precipitation at Harvard Forest. The estimated effect of precipitation, however, was quite small (change of approximately 100 centimeters in annual precipitation required to produce a day difference in FLS), suggesting annual precipitation is not a major contributor to the observed inter-annual FLS variation.

%\noindent These result suggest that even if water dynamics shaped FLS in some temperate taxa on the evolutionary time-scale, it does not appear to be a major driver of FLS variation on an ecological one. Given this, \\


%<<Code_chunk_Minimal_example3, results='asis',Fig=FALSE, echo=FALSE,message=FALSE,warning=FALSE,out.width='6.5in'>>=
%rm(list=ls()) 
%options(stringsAsFactors = FALSE)
%#setwd("~/Documents/git/proterant/sub_proj")%


%library(ggtree,quietly=TRUE)
%library(gridExtra,quietly=TRUE)

%mich.data<-read.csv("..//MTSV_USFS/michdata_final.csv")
%mich.tre<-read.tree("..//MTSV_USFS/michtre_final.tre")

%silv.data<-read.csv("..//MTSV_USFS/silvdata_final.csv")
%silv.tre<-read.tree("..//MTSV_USFS/silvtre_final.tre")


%mich.data$synth<-mich.data$pro2+mich.data$pro3
%mich.data$synth2[which(mich.data$synth==0)] <- "always seranthous"
%mich.data$synth2[which(mich.data$synth==1)] <- "transitional"
%mich.data$synth2[which(mich.data$synth==2)] <- "always hysteranthous"

tr <-mich.tre

par(mar=c(2,0.5,2,0))

dd <- data.frame(taxa  = mich.data$name, hysteranthy = mich.data$synth2 )

p<-ggtree(tr,layout="circular")
p <- p %<+% dd +geom_tippoint(aes(color=hysteranthy,shape=hysteranthy),size=2)
p<-p+theme(legend.position = "none")



silv.data$synth<-silv.data$pro2+silv.data$pro3
silv.data$synth2[which(silv.data$synth==0)] <- "always seranthous"
silv.data$synth2[which(silv.data$synth==1)] <- "transitional"
silv.data$synth2[which(silv.data$synth==2)] <- "always hysteranthous"

#### make a tree for svics

tr1 <-silv.tre

dd1 <- data.frame(taxa  = silv.data$name, hysteranthy = silv.data$synth2 )


q<-ggtree(tr1,layout="circular")
q <- q %<+% dd1 +geom_tippoint(aes(color=hysteranthy,shape=hysteranthy),size=2)

q<-q+theme(legend.position="bottom")


ggpubr::ggarrange(p,q,ncol=2,legend="bottom",common.legend = TRUE,labels = c("a)","b)"), vjust = 10)

@