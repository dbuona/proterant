
% Straight up stealing preamble from Eli Holmes 
%%%%%%%%%%%%%%%%%%%%%%%%%%%%%%%%%%%%%%START PREAMBLE THAT IS THE SAME FOR ALL EXAMPLES
\documentclass{article}[11pt]
%Required: You must have these
\usepackage{alltt}
\usepackage[margin=1in]{geometry}
\usepackage{graphicx}
\usepackage{natbib}
\usepackage{gensymb}
%\begin{footnotesize}
%\address{1300 Centre Street \\ Boston, MA, 20131}
%\end{footnotesize}
\IfFileExists{upquote.sty}{\usepackage{upquote}}{}
\begin{document}
\bibliographystyle{..//..//refs/styles/newphyto.bst}
\def\labelitemi{--}
\parindent=24pt
\noindent\includegraphics[width=0.2\textwidth]{/Users/danielbuonaiuto/Desktop/arb_logo.png}
\pagenumbering{gobble}

%Optional: I like fancy headers
%\usepackage{fancyhdr}
%\pagestyle{fancy}
%\fancyhead[LO]{How do climate change experiments actually change climate}
%\fancyhead[RO]{2016}
 
%%%%%%%%%%%%%%%%%%%%%%%%%%%%%%%%%%%%%%END PREAMBLE THAT IS THE SAME FOR ALL EXAMPLES

}

% line numbers for letter
%\usepackage{lineno}


%Start of the document
\begin{document}
\vspace{2.5ex}

\noindent Dear Dr. Vamosi,\\ %do you address this to the handling editor?

\vspace{1.5ex}

\noindent Please consider our revised manuscript ``Reconciling competing hypotheses regarding flower-leaf sequences in temperate forests for fundamental and global change biology" as a Viewpoint article in \emph{New Phytologist}.\\

\noindent Deciduous woody plants of the temperate zone exhibit considerable variation in the order of reproductive and vegetative events, or flower-leaf sequences (FLSs). Several long-standing hypotheses suggest FLSs are under strong selection and critical to fitness, yet research efforts have yet to yield a consistent, well-supported explanation for this variation. We argue that advances may have stalled because FLSs are typically classified into categorical descriptors that do not reflect the underlying biology of the FLS hypotheses. We present and test a new conceptual framework for the study of FLSs built on 1) quantitative measures of FLSs and 2) observations of FLSs below the species level. Our treatment of this topic charts a path forward to better understand both the evolutionary origins of FLS variation and the consequences of observed FLS shifts associated with climate change.\\

\noindent Comments from three reviewers suggested our manuscript was on a timely and important topic and generally well written, but pointed out many areas for improvement. Based on their comments, we have revised the structure of the manuscript to highlight that the two components of our framework, quantitative FLS measurements and intra-specific observations, offer complementary but different contributions to the study of FLSs. As the need to make this distinction stronger was identified by all three reviewers we have further provided additional analyses that demonstrate these differences in a revised version of Figure 4.\\

\noindent Reviewers also raised important questions about the sensitivity of our analyses to the use of different flower and leaf phenophases (Reviewer 1) and functional traits representing the FLS hypotheses (Reviewer 2) in our models. To address these concerns, we now provide additional analyses using two different measurements of FLSs (the time between leaf and flower budburst and the time between flowering opening and 75\% leaf expansion) in the Supporting Information and four alternative functional traits to represent the FLS hypotheses (Fig. S7 and Fig. S5 respectively), which find our models to be robust to these alternative specifications.\\

\noindent We feel that the reviewers' input has helped shape a new submission that is much improved, and we detail our specific changes in the following pages with reviewer comments in \emph{italics} and our responses in regular text.\\

\noindent We have worked to address Reviewer 2's concerns regarding our manuscript length. We have cut much original text, but have also added substantial text and references to address reviewer concerns; as a result our current submission is slightly more concise with approximately 3,300 words with a 192 word summary, and four figures (well under the 6,500 word maximum for Viewpoints). We can work to cut text or references further if requested but believe the current submission is the best balance between length and clarity. It is not under consideration anywhere else. We hope that you will find it suitable for publication in \emph{New Phytologist}, and look forward to hearing from you.\\\\\\

\noindent Best,\\
\\\\\\\\







\noindent Daniel Buonaiuto




\end{document}
