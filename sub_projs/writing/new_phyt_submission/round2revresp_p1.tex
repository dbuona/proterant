
% Straight up stealing preamble from Eli Holmes 
%%%%%%%%%%%%%%%%%%%%%%%%%%%%%%%%%%%%%%START PREAMBLE THAT IS THE SAME FOR ALL EXAMPLES
\documentclass{article}[11pt]
%Required: You must have these
\usepackage{alltt}
\usepackage[margin=1in]{geometry}
\usepackage{graphicx}
\usepackage{natbib}
\usepackage{gensymb}
%\begin{footnotesize}
%\address{1300 Centre Street \\ Boston, MA, 20131}
%\end{footnotesize}
\IfFileExists{upquote.sty}{\usepackage{upquote}}{}
\begin{document}
\bibliographystyle{..//..//refs/styles/newphyto.bst}
\def\labelitemi{--}
\parindent=24pt
\noindent\includegraphics[width=0.2\textwidth]{/Users/danielbuonaiuto/Desktop/arb_logo.png}
\pagenumbering{gobble}


%%%%%%%%%%%%%%%%%%%%%%%%%%%%%%%%%%%%%%END PREAMBLE THAT IS THE SAME FOR ALL EXAMPLES
% line numbers for letter
%\usepackage{lineno}


%Start of the document
%\begin{document}
\vspace{2.5ex}

\noindent Dear Dr. Vamosi,\\ 
  
  \vspace{1.5ex}

\noindent We hope that you will consider the revisions to our manuscript ``Reconciling competing hypotheses regarding flower-leaf sequences in temperate forests for fundamental and global change biology" for publication as a Viewpoint article in \emph{New Phytologist}.\\

\noindent Deciduous woody plants of the temperate zone exhibit considerable variation in the order of reproductive and vegetative events, or flower-leaf sequences (FLSs). Several long-standing hypotheses suggest FLSs are under strong selection and critical to fitness, yet research efforts have yet to yield a consistent, well-supported explanation for this variation. We argue that advances may have stalled because FLSs are typically classified into categorical descriptors that do not reflect the underlying biology of the FLS hypotheses. We present and test a new conceptual framework for the study of FLSs built on 1) quantitative measures of FLSs and 2) observations of FLSs below the species level. Our treatment of this topic charts a path forward to better understand both the evolutionary origins of FLS variation and the consequences of observed FLS shifts associated with climate change.\\

\noindent We are pleased that two of the reviewers found previous revisions to the manuscript satisfactory and appreciate both Reviewer 1's acknowledgement that the manuscript and its analyses are much improved as well as their continued input towards its refinement.\\ 

\noindent Reviewer 1 asked us to elaborate on the sensitivity of our models to the use of different phenophases, which we have done both in the main text of the manuscript as well as in the Supporting Information. They also asked us to clarify our usage of drought metrics to represent the water limitation hypothesis. We explain our decision-making process for choosing metrics to represent this hypothesis at length below and have added an additional analysis using a robust aridity index to represent this hypothesis. We are pleased to report that the inferences from our models are robust to the use of any one of the four alternative predictors we now provide to represent the water limitation hypothesis. We are grateful to the reviewer for pressing us to think harder about this hypothesis.\\

\noindent As with our previous re-submission, we feel that the reviewer's input has helped shape a new submission that is much improved, and we detail our specific changes in the following pages with reviewer comments in \emph{italics} and our responses in regular text.\\

\noindent We hope that you will find it suitable for publication in \emph{New Phytologist}, and look forward to hearing from you.\\\\\\

\noindent Best,\\
\\\\\\\\







\noindent Daniel Buonaiuto

\end{document}