\documentclass{article}[11pt]
%Required: You must have these
\usepackage{graphicx}
\usepackage{tabularx}
\usepackage{natbib}

\usepackage{array}
\usepackage{amsmath}
%\usepackage[backend=bibtex]{biblatex}
\setkeys{Gin}{width=0.8\textwidth}
%\setlength{\captionmargin}{30pt}
\setlength{\abovecaptionskip}{10pt}
\setlength{\belowcaptionskip}{10pt}
\topmargin -1.5cm 
\oddsidemargin -0.04cm 
\evensidemargin -0.04cm 
\textwidth 16.59cm
\textheight 23.94cm 
\parskip 7.2pt 
\renewcommand{\baselinestretch}{1.2} 	
\parindent 0pt

\bibliographystyle{..//..//refs/styles/besjournals.bst}
\usepackage{xr-hyper}
\usepackage{hyperref}
\externaldocument{suppliment}


\usepackage{lineo}
\linenumbers
\title{Better title needed: Flower-leaf phenological sequences in the American Plums--unraveling the mystery of hysteranthous flowering in insect-pollinated species}

\usepackage{Sweave}
\begin{document}
\input{outline-concordance}
\maketitle


\section*{Introduction}
%<<label=numbers, echo=FALSE, results=hide, message=FALSE>>=
%rm(list=ls()) 
%options(stringsAsFactors = FALSE)
%require(brms,quietly = TRUE)
%setwd("~/Documents/git/proterant/investment/Input")
%load("pcerasus.Rda")
%doyeff<-round(fixef(mod.ord.scale.phlyo)[6],digits=2)
%doyints<-round(fixef(mod.ord.scale.phlyo)[6,3:4],digits=3)
\noindent Woody perennials have a unique ability among plants to seasonally begin reproduction prior to vegetative growth. This flowering-first phenological sequence known as hysteranthy, proteranthy or precocious flowering is particularly common in temperate, deciduous forests around the globe \citep{Rathcke_1985}. A number of studies suggest that this flower-leaf sequences (FLSs) are under selection, and that hysteranthy has functional significance \citep{Gougherty2018,Buonaiuto2020,Guo2014}.

\noindent The most common, and well-tested explanation for the evolution of hysteranthy in temperate forests is that it is adaptive for wind-pollination, as leafless canopies increase wind speeds for pollen transport and reduce the likelihood of pollen interception by vegetation \citep{Whitehead1969,Niklas1985}. In the dry-deciduous tropics of South and Central America, hysteranthy is also common \citep{Rathcke_1985,Franklin2016}, and is regarded as an important adaptation to alleviate water stress by partitioning the hydraulic demand of flowers and leaves across the season \citep{Gougherty2018,Franklin2016,Borchert1983,Reich1984}.

However, these explanations do not address the widespread prevalence of hysteranthy in biotically-pollinated taxa found in temperate regions that are rarely water-limited in the early season during which flowering and leafing occur \citep{Polgar2011}. This number is not trivial; a recent analysis found that approximately 20\% of the hysteranthy species in the moist, Eastern Temperate Forests of North America are biotically pollinated \citep{Buonaiuto2020}. With mounting evidence anthropogenic climate change is driving shifts in flower-leaf sequences \citep{Ma2020:aa}, expanding our understanding of the functional significance of hysteranthy to included these groups is vital to forecasting the demography and performance of forest communities in an era of global climate change.

Despite the fact that  hysteranthous flowering in biotically-pollinated taxa violate (better word), the conventional explanation for this phenological syndrome, several alternative hypotheses to the wind pollination hypothesis have been put forth and may help explain the function of hysteranthy in biotically-pollinated species. Below we review them, and their predictions of trait associations.

\subsection*{Hypotheses of Hysteranthous flowering in biotically pollited taxa}
 
\underline{Water limitation hypothesis:} Despite being considered a ``wet" biome, there is still considerable variation in water availability in space and time within temperate regions of the globe. It is possible that the function of hysteranthous flowering in thes regions parallels that in the dry tropics---partitioning hydraulic demand across the season to allow hysteranthous species to tolerate increased aridity. If this is the case, we would expect to find hysteranthous taxa in locations that are, on average, drier than their non-hysteranthous kin.

\underline{Freeze tolerance hypothesis:} There is a demonstrated physiological relationship between drought and freeze tolerance, and it has been suggested that adaptations to drought allowed plants to expand their ranges higher latitudes of the Northern Hemisphere \citep{}. It is possible that hysteranthy contributed to this adaptation, though the mechanisms by which hysteranthous flowering may contribute to cold tolerance has not been investigated. One possibility is that for long lived organisms like woody plants, occasional frost damage to flowers has less of an impact on lifetime fitness, than damage to leaves (say better, I think I have old writing that might say this better). With this hypothesis, we would expect hysteranthous species to be found at colder sites than related non-hysteranthous ones. (Drop this if I don't have readily available data to test it.)

\underline{Insect-visibility hypothesis:} Hysteranthous flowers are visually conspicuous in the landscape. It is possible that like in wind pollinated taxa, hysteranthy in biotically pollinated taxa is an adaptation for pollination efficiency as a flowering-first species are easier for insects pollinators to locate \citep{}. This hypothesis predicts that flower displays will differ in size between flowering- first and leafing-first species. Though the direction is unclear. 1) Hysteranthy may be associated with smaller flowers. Since they are easier to see, there is weaker selection on large floral display. 2) Hysteranthy may be associated with bigger flowers. Because these species are going all in on visual displays, big flower might be additive to the benefits of hysteranthy. A second complicating factor is that there is likely to be associates between flower size and hysteranthy even if pollinator visibility doesn't matter due to developmental constraints. For example it requires more time and energy to produce big displays, so non-hysteranthous species that flower later in the season, after leaves emerge to gather energy are can produce bigger dispalys than early flowering hysteranthous species. (Could also move some of these nuances and contradictions for the discussion)

\underline{Phenological niche extension:} Species that flower before their leaves inheirantly flower early in the season. It is possible that hysteranthy flower is simply a by-product of selection for early flowering.
``Recent work from \citet{Savage2019} demonstrated that spring flower phenology is less constrained by prior phenological events than leaf phenology, which would allow selection to drive flowering into the early season, producing the flowering-first FLS\linelabel{small6.r1}. With this hypothesis there is no specific advantage to a species flowering before or after leafing; all that matters is its absolute flowering time." (quotes indicate self plagiarism and needs to be re-written).

\underline{Fruit maturaturion hypothesis:} Like the phenological niche hypothesis describe above, there are several aspects of reproductive development that suggest hysteranthy is a by-product for early flowering, driven by development constraints. Hysteranthy may be common in large fruited species that require lots of time to mature their fruits. Alternatively, its may be common in small, early fruiting species that have evolved dispersal syndromes (wind dispersal, non-dormant seeds) that reuqire dispersal early in the season. In either case, we  should expect fruit size to associate with hysteranthy.

Of course none of the hypotheses are mutually exclusive. One challenge is the same traits correlation could be driven by different mechanisms (ie small flower could be insect-visibility, developmental constraint, aridity tolerance or all of the above). Yet despite this, we should still investigate these associations. Why? because this will help us narrow our study and better understand this trait as a whole. Or because that's just what science is.

\noindent A second challenge for robust testing of hysteranthy hypotheses is that most characterizations of flower-leaf phenological sequences are based on expert-opinion verbal descriptions (e.g. ``flowers before leaves" or ``flower before/with leaves"), which make comparisons across taxa, time and space difficult and sensitive to observer bias  \citep[see;][]{Buonaiuto2020}. This problem can be overcome by adopting standardized quantitative measures of plant phenology for observational studies and applying them to historic data records. Herbarium records are an excellent source of data that can be leveraged for quantitative phenological measurements \citep{Willis2017}, but have not be used widely to investigate variability of flower-leaf sequences variation among and within species.

The American plums (\textit{Prunus} subspp. \textit{prunus} sect. \textit{prunocerasus}) offer potential for a higher resolution investigation of drivers of hysteranthous flowering in taxa that don't fit the bill. (Better topic sentence needed.) The 16 species that make up the section are distributed across the temperate zone of North America and, like the genus \textit{Prunus} at large, are all insect-polliated, yet show pronounced inter-specific variation in flower-leaf sequences.  Species in this section are well represented in herbaria records (Fig. \ref{fig:mappy}), making them a tractable group to measure and assess variation in flower-leaf sequences as well as other ecological and morphological characteristics that may explain the evolution of this variation (eww this paragraph needs help). 

\noindent In this study,we used herbaria records to to quantify flower-leaf sequence patterns in the American plums, (subspecies  \textit{Prunus}, sect. \textit{prunocerasus}). We then evaluated the association between hysteranthy and several ecological and morphological traits to interrogate the functional hypotheses for hysteranthous flowering described above. We then compare our findings in this clade to associations between hysteranthy and traits in the larger genus obtained from published accounts in flora to better understand how these dynamics vary over taxonmic scales. Our findings both clarify the hypothesized function of flower-leaf sequence variation in biotically-pollinated taxa, and offer insights into how shifting flower-leaf sequences may impact species demography and distributions as climate continues to change.

%However, there are two major methodological challenges to testing the hydraulic demand hypothesis: First, characteristics like aridity tolerance, are the emergent product of a suite of biological traits \citep{Simova:2017vk}. Thus, when analyzing selective drivers of any particular trait at large taxonomic scales, unmeasured trait differences may obscure the estimated effects of the trait of interest, biasing results. This is a common problem in trait-based ecology, and one of the most promising solutions for understanding the functional significance of hysteranthy in woody plants is through character deconstruction \citep{Terribile2009}; comparing flower-leaf sequences variation for only a subset of taxa of shared phylogenetic and morphological character.   

%\noindent A second challenge for robust testing of hysteranthy hypotheses is that most characterizations of flower-leaf phenological sequences are based on expert-opinion verbal descriptions (e.g. ``flowers before leaves" or ``flower before/with leaves"), which make comparisons across taxa, time and space difficult and sensitive to observer bias  \citep[see;][]{Buonaiuto2020}. This problem can be overcome by adopting standardized quantitative measures of plant phenology for observational studies and applying them to historic data records. Herbarium records are an excellent source of data that can be leveraged for quantitative phenological measurements \citep{Willis2017}, but have not be used widely to investigate variability of flower-leaf sequences variation among and within species.

%Despite the pervasiveness of this phenological syndrome, direct tests of the function of hysteranthy in biotically pollinated taxa are rare for temperate forest species.
%but

%therefore


%. and in these ecosystems, flowering is associated with a recovery in plant water status due to leaf drop \citep{Borchert1983,Reich1984}. By temporally separating leaf and flower activity, woody plants can partition their hydraulic demand across the season, alleviating water stress \citep{Gougherty2018,Franklin2016}. 


%However, this hypothesis fails to address the prevalence of hysteranthous taxa that are biotically-pollinated. Approximately 30\% of woody plant species of Eastern temperate forests of North America flower before leafing out, and of these, approximately 20\% are biotically pollinated  \citep{Buonaiuto2020}. Despite the pervasiveness of this phenological syndrome, direct tests of the function of hysteranthy in biotically pollinated taxa are rare for temperate forest species.

%Yet looking to other biomes in which hysteranthous flowering is also common offers important insights regarding the function of hysteranthy in temperate, biotically-pollinated taxa. In the dry-deciduous tropics of South and Central America, flowering during the leafless period is also common \citep{Rathcke_1985,Franklin2016}. In these ecosystems, flowering is associated with a recovery in plant water status due to leaf drop \citep{Borchert1983,Reich1984}. By temporally separating leaf and flower activity, woody plants can partition their hydraulic demand across the season, alleviating water stress \citep{Gougherty2018,Franklin2016}. These physiological observations suggest that hysteranthous flowering may be an adaptation to arid environments.

%It is unclear whether this hydraulic demand hypothesis (also refered to as the water dynamic hypothesis \citep{Gougherty2018} or water limitation hypothesis \citep{Buonaiuto2020}) is relevant in the temperate zone where forests are rarely water-limited in the early season during which flowering and leafing occur \citep{Polgar2011}.%, and evidence of links between hysteranthy and aridity tolerance is mixed \citep{Gougherty2018,Buonaiuto2020} 
%Yet the hypothesis yields several predictions to evaluate whether hysteranthy serves to increase aridity tolerance in temperate flora:
%\begin{enumerate}
%\item Hysteranthous taxa should be found in dryer habitats compared to closely related, non-hysteranthous species.
%\item Hysteranthy may be linked to other reproductive traits associated with dry environments such as reduced flower and fruit size \citep{Herrera:2009aa,Liu:2013ua}.
%\item Additionally, flower-leaf sequences can be highly variable within individuals across time \citep{Buonaiuto2020}, and if hysteranthy contributes to aridity tolerance, climate variability may be positively correlated with variability in hysteranthy.
%\end{enumerate}
%\noindent With mounting evidence anthropogenic climate change is both driving shifts in flower-leaf sequences \citep{Ma2020:aa} and changing geographic patterns of water availability \citep{Overpeck11856}, understanding the functional significance of hysteranthy is vital to forecasting the demography and performance of forest communities in an era of global climate change. However, there are two major methodological challenges to testing the hydraulic demand hypothesis:

%\noindent First, characteristics like aridity tolerance, are the emergent product of a suite of biological traits \citep{Simova:2017vk}. Thus, when analyzing selective drivers of any particular trait at large taxonomic scales, unmeasured trait differences may obscure the estimated effects of the trait of interest, biasing results. This is a common problem in trait-based ecology, and one of the most promising solutions for understanding the functional significance of hysteranthy in woody plants is through character deconstruction \citep{Terribile2009}; comparing flower-leaf sequences variation for only a subset of taxa of shared phylogenetic and morphological character.   

%\noindent A second challenge for robust testing of hysteranthy hypotheses is that most characterizations of flower-leaf phenological sequences are based on expert-opinion verbal descriptions (e.g. ``flowers before leaves" or ``flower before/with leaves"), which make comparisons across taxa, time and space difficult and sensitive to observer bias  \citep[see;][]{Buonaiuto2020}. This problem can be overcome by adopting standardized quantitative measures of plant phenology for observational studies and applying them to historic data records. Herbarium records are an excellent source of data that can be leveraged for quantitative phenological measurements \citep{Willis2017}, but have not be used widely to investigate variability of flower-leaf sequences variation among and within species.

%\noindent In this study,we used herbaria records to to quantify flower-leaf sequence patterns in the American plums, (subspecies  \textit{prunus}, sect. \textit{prunocerasus}. We then evaluated the association between hysteranthy and several ecological and morphological traits to test the predictions of the hydraulic demand hypothesis of hysteranthy. Our findings both clarify the hypothesized function of flower-leaf sequence variation in biotically-pollinated taxa, and offer insights into how flower-leaf sequences may impact species distributions as climate continues to change.


%<<>>=
%mich.data<-read.csv("..//..//sub_projs/MTSV_USFS/michdata_final.csv")
%table(mich.data$pro2,mich.data$Pollination)

%table(mich.data$pro)
%42/147
%table(mich.data$pro2)
%82/147



\section*{Methods}
%\subsection{Study system}
%\noindent The genus \texit{Prunus} comprises approximately 200 species distributed across the globe \citep{Chin:2014wu}, Within the genus, The American plums (\textit{Prunus} subspp. \textit{prunus} sect. \textit{prunocerasus}) offer potential for a higher resolution investigation of drivers of hysteranthous flowering. The 16 species that make up the section are distributed across North America and, like the genus \textit{Prunus} at large, show pronounced inter-specific variation in flower-leaf sequences. While within the larger genus species can be separated into three distinct morphological clades by inflorescence architecture (solitary, corymbose or racemose) all members of the section share solitary inflorescences \citep{Shaw:2004aa} allowing for refined character deconstruction. Species in this section are well represented in herbaria records (Fig. \ref{fig:mappy}), making them a tractable group to measure and assess variation in flower-leaf sequences as well as other ecological and morphological characteristics related to the hydraulic demand hypothesis. 

\subsection{Quantifying flower-leaf sequence variation}
We obtained digital herbarium specimens for all member of the section \textit{Prunocerasus} from the Consortium of Midwest Herbaria Database. To quantify the flower-leaf sequence sequence variation within and across species we randomly sample 200 specimens for each species and scored the phenological development of flower and leaves using a modified BBCH scale for woody plants \citep{Finn2007}. In total, we evaluated the phenology of 2521 specimens, but only specimens with visible flower were included in this analysis (n=1009). We reconstructed the phylogenetic relationships among species in this group based on the tree topology in \citet{Shaw:2004aa}. Following the methods of \citet{Granfen1989} we computed branch lengths for this phylogeny by assigning each node a height and computing the distance between upper and lower nodes using the R package ``ape" \citep{}.

\textit{Need to write this part more professionally}. To compute a phylogentic signal for flower-leaf sequence variation, we calculated the mean of the log(vegetative BBCH observed during flowering) for each species and calculated Bloomberg's K using the function phylosig \citep{}. 

To quantify FLS variation, we fit an ordinal, hierarchical, Bayesian, phylogenetic mixed model \citep{Garamszegi2014} to assess the likelihood an individual would be at any given vegetative BBCH phase while flowering. Because we expect that hysteranthy is more likely to occur earlier in the flowering period and species differ in their flowering periods, we included the day of the observation as a varying slope, main effect in the model and species and phylogeny as random effects. The model is written below:\\

$logit(P(Y \leq j)) &= \beta_{[j]sp[i]}+ \beta_{[j]sp[i]}+ \beta_{day of year[sp[i]]}*X_1+ \epsilon_{i}$\\
  
   \epsilon_i & \sim N(0,\sigma^2_y) \\ 
   
   where Y is the ordinal outcome (leaf stage) and j is the number of categories (1,2,...6). $P(Y \leq j))$ is the probability of Y less than of equal to a category j&=1,...j&-1.  In this varying slope and intercept model, $\beta_{[j]}$ describes an intercept for each category [1,2,...6], while slope \$beta_{day of year[sp[i]]}$ is constant across categories. 
  
  \noindent The influence of the phylogeny $\alpha_{phylo}$ was modeled as follows:\\
  \alpha_{sp} & \sim N(\mu_{\alpha}, COR[\sigma^2_{phylo}]) \\
  
  \noindent THe $\alpha$ for species effects independent of the phylogeny was modeled as follows:\\
  \alpha_{sp} & \sim N(\mu_{\alpha}, \sigma^2_{species}) \\

 We fit the model in the R package ``brms" \citep{Burkner2018} using weakly informative priors, and ran the model on four chains with a warmup of 3,000 iterations and 4,000 sampling iterations for a total of 4,000 sampling iterations. Model fit was assessed with Rhats <1.01 and high effective sample sizes and no divergent transitions.
 
Because the day of observation strongly influenced the BBCH stage likelihood, quantifying flower-leaf sequences among species was intractable without accounting for this temporal trend. To address this issues, we used our model to predict the likelihood each species would be observed at a given vegetative BBCH stage during flowering at the 0\%, 25\% 50\% and 75\% quantiles of their flowering period. We then developed a flower-leaf sequence index, by assigning a numerical score to each species per seasonal quantile, and summing over the full flowering season. In each seasonal quantile, species received a 1 if more that 50\% of their probability distribution occurred at BBCH 0 and BBCH 09 and a 0 if not. These values were summed across the season generating an index from 0 (never hysteranthous) to 4 (hysteranthous through late season (Q75)), where 1&= hysteranthous at start of season, 2&= hysteranthous through early season  (Q25) and 3 &= hysteranthous  through mid season (Q50). We also used two alternative indexing schemes ($>$25\% of the probability distribution occured at BBCH 0 and $>$40\% of the probability distribution occurred at BBCH 0 and BBCH 09).

\subsection{Evaluating hysternanthy hypotheses}

To test the predictions of the hypotheses of hysteranthy we obtained data on petal length, fruit diameter and directly from herbarium specimens and characterized the aridity of the sites specimens were collected from using the Palmer Modified Drought Index (PDSI).

\noindent For our morphological measurements, we sampled an additional 321 specimens measured the petal length of up to 10 randomly selected petals per specimen (n=2757) using ImageJ image processing software. We also used ImageJ to measure the diameter of fruits on an additional 316 specimens, measuring up to 5 fruit per specimen (n=224).
We computed the average Palmer Modified Drought Index score from 1900-2017 for every \textit{Prunocerasus} specimen in the database (n=2305) from the North America Drought Atlas \citep{Cook2004}.

We than used Bayesian phylogenetic mixed models to test the relationship between flower-leaf sequence index scores and each of the variables. In these models, we included species and phylogeny as the random effect. 

The model structure is written below: 

  y_i &= \alpha_{ind/sp[i]} +\alpha_{phylo[i]} + \beta_{hyst.index}*X_{hyst.index} + \epsilon_i\\
  
  \epsilon_i & \sim N(0,\sigma^2_y) \\ %Check this
  
  \noindent The effect of the phylogeny was model as above.% and here, the individual effects within species were modeled:\\
  %alpha_{ind/sp} & \sim N(\mu_{\alpha}, \sigma^2_{ind/sp}) \\
  
Like above, we fit these models in the R package ``brms" \citep{Burkner2018} using weakly informative priors, and ran the model on four chains with a warmup of 3,500 iterations and 4,500 sampling iterations for a total of 4,000 sampling iterations. Model fit was assessed with Rhats <1.01 and high effective sample sizes and no divergent transitions. We also ran each model using our two alternative FLS indexing approaches to make sure our particular indexing approach was not influencing our results. Though these alternative classification scheme did change the hysteranthy index score for some species (Fig. \ref{fig:plums}), they changes did not substantially impact the interence from our models (see Tab. \ref{tab:modput} for comparisons).

%\noindent Because our data dependent and independent were collect we employed a sequential modeling approach to first estimate the mean and standard error of the posterior distribution of trait values for each species then model the relationship between these estimate and the likelihood of hysteranthy using Bayesian measurement error models. This approach propagates the error in the initial estimates of trait values into the our final model, yielding a more accurate evaluation than using mean trait values alone \citep{}. For each parameter of interest, we ran Bayesian phylogenetic mixed-effects models with our measured traits as the response variable and species as the random effect. For traits like flower petal length and fruit diameters than included multiple measurements per specimen, we included specimen ID as an additional random effect. The model structure is written below:

%\noindent Then using each of these trait mean estimates as predictors, we modeled their associations with flower leaf sequences OF using a repeat measure phylogenetic mixed ordinal regression in brms \citep{}. Because we found the three predictors of interested to be highly colinear (pariwise correlations >0.5), we ran one regression model per predictor trait to avoid skewing our model inference due to multi-collinearity \citep{MacElreath, Nations}.

%For all models in the sequences,

\section*{Results}
\subsection*{Quantifying flower leaf sequences in the American plums}
We found substantial inter-specific differences in flower-leaf sequences within the American plums (Fig. \ref{fig:ordinals}, \ref{fig:plums}). The phylogenetic signal was relatively weak (Phylogenetic signal K : 0.28), and flower-leaf sequences patterns were strongly dependent on the day of observations, with observations later in the the flowering season of each species decreasing the likely hood of finding flowers open during early vegetative BBCH phases ($\beta_{doy}$ 0.03, $CI_{50}$ [0.02,0.03] ). Based on our flower leaf sequence index, two species (\textit{P. umbellata},\text{P. mexicana}) were likely to be hysteranthous regardless of the time of observation and three species (\textit{P. rivularis},\texit{P. subcordata}, and \textit{P. texana}) were always most likely to flower after level expansion began (Fig. \ref{fig:phylo2}). All other species displayed intermediate phenotypes with five species mostly likely to hysteranthous at the start of the season (\textit{P. alleghaniensis},\texit{P. americana},\textit{P. hortulana},\textit{P. munsoniana} and \texit{P. nigra}), one species through early season (\textit{P gracilis}) and two species through mid season (\textit{P. angustifolia}, \textit{P. maritima}) (Fig \ref{fig:phylo2}).

\subsection*{Associations between hysteranthy and environmental and morphological traits}
We found a negative association between flower-leaf sequence index and mean pdsi ($\beta$: -0.03 ,$CI_{50}$[-0.05,  0.02] ,Fig. \ref{fig:prunes}a.), suggesting that species that displayed hysteranthous flowering later into their flowering season were found in dryer locations. 

We found a negative association between flower-leaf sequence index and both  petal length and fruit diameter (-.21, $CI_{50}$"[-0.38 -0.04],-1.40, $CI_{50}$[-1.97 -0.82] respectively), though the relationship between FLS index and fruit size was much stronger (Fig. \ref{fig:prunes}b.,c.).

\section*{Discussion}

Our analyses suggest that within the American plums, hysteranthous taxa occur in more arid environments and are associated with drought-tolerant reproductive traits like reduced flower and fruit size. These associations support the hydraulic demand hypothesis of hysteranthous flowering. These results indicate that even though water limitation less common during the flowering season in temperate trees, the temporal segregation of flowering and leaf phenology can impact whole plant-water status later in the season.

Studies that have compared the transpiration rates among flowers and leaves that occur simultaneously provide insights to the potential importance of this seasonal partitioning for maintaining water status. These studies report floral transpiration rates of flowers can range from 20\%-60\% of that of leaves under comparable conditions \citep{Whiley:1988uf,Roddy:2012wn}. This additional hydraulic demand can drive loss of stomatal conductance and  decrease photosynthetic rates \citep{Galen:1999vr}. A recent study \citet{Liu:2017wg} comparing hydraulic properties of flowers and leaves in two hysteranthous tree species (\textit{Magnolia spp}), found that sap flow to flowers (a measure of water movement) was 22-55\% that of leaves. When considering species in or study specifically, the xylem conductivity of spring floral branches of \textit{Prunus americana} is reported to be ~20\% of summer foliage branches \citep{McMann:2022ww}. Taken together, magnitude of water loss through floral organs in these physiological measurements demonstrate an underlying mechanism for the macro-ecological patterns we observed in our data.

Our finding that smaller flowers were associated with extended hysteranthy in the Amerian plums may be surprising when viewed in the context of a classic ecological tradeoff. It is well established that larger flowers demand more resources to maintain turgor and reproductive function than smaller ones\citep{Galen:1999vr,Lambrecht:2007ur}, therefore one might expect that hysteranthous flowering serves to compensate for maintaining larger flowers and in dry environments. The fact that we observed a negative association between the degree of hysternathy and flower size suggests rather, that hysteranthy might be part of a suite of traits that operate to increase the aridity tolerance of a species.

The negative relationship between hysteranthy and flower size we observed is also expected when considered in the context of resource allocation. A negative trade-off for the benefits gained by hysteranthous flowering is that hysteranthous woody plants much begin their reproductive investment from stored carbon alone; at the time of their annual cycle when their stored reserved are likely at their lowest \citep{}. The association between smaller flower and hysteranthous flowering has been observed in other clades \citep{}, though to our knowledge has not been investigated on the context of hydraulic costs or aridity tolerance. It it, however, clear that hysteranthous species have evolved specialized mechanisms for mobilizing water and carbohydrates early in the season to accommodate this resource partitioning strategy \citep{}. Sum up this paragraph.

Of course, selection on both phenology and floral traits is driven by a number of other factors than just plant hydraulics and resource allocation. The support we found for the hydraic demand hypothesis does not rule out other eco-evo drivers shaping the flower-leaf sequences of insect-pollinated. In fact, the relationship we observed between hysteranthy with flowering and fruit size could also be evidence for alternative hypotheses for FLS.

Pollinator attraction is major selective force on both floral phenology and morphology \citep{} and it has been alternatively suggested that hysteranthous flowering is an adaptation to increase the visibility of flowers to visually-foraging pollinators \citep{}. To our knowledge this hypothesis has not bee widely tested though there is evidence that background contrast does impact pollinators ability to locate flowers \citep{}. This hypothesis and the hydraulic demand hypothesis that we tested may be related, with loss of pollinator visibility that is associated with accompany the reduction in flower size \citep{} due to aridity compensated for by hysteranthous flowering. While we cannot make this mechanistic link from our analysis, our findings that hysteranthy is associated with aridity and reduced flower size set up intriguing follow up work to elucidate the ecological and evolutionary links between floral morphology and function, aridity and hysteranthous flowering.

In this study we intentionally chose to analyze a small, and morphologically restricted taxonomic clade in order to reduce the impact of unmeasured biological variation on our traits of interest. Our findings compliment and clarify previous analyses performed at large taxonomic scales which suggested that aridity may be a more important driver in biotically pollinated taxa than wind pollinated ones \citep{}. It was interesting we found a relatively weak influence of phylogeny at our restricted scale, while it appears to be quite strong at larger ones 
\citep{}. \textit{Jonathan is there something interesting we can say about this in a line or two?}

To compliment the work that has been done at both very fine and course taxonomic resolution, it would be useful to explore the drivers of hysteranthy at an intermediate scale to better understand if the associations we observed in the American plums emerged at the genus or family level, or in other clades with strong intra-generic flower-leaf sequence variation like Rhododendron, Magnolia, Acer, Cornus. 

Flower-leaf sequences patterns can vary substantially on an interannual basis with populations and even individuals \citep{}. It has been suggested that this variation is a product of differential sensitivity to temperature and light cues between flowers and leaves \citep{}, but if water limitation drives the evolution of hysteranthous flowering at the species level, it is possible that water availability may influence the plasticity of flower-leaf sequence on a seasonal timescale. Executing experimental or observational studies about this (say better) is an important step of for understanding the significance to the water dynamics hypothesis and for predict how woody plant phenological sequences may shift with climate change as local patterns of both temperature and water availability continue to change in the coming decades.




\bibliography{..//..//..//sub_projs/refs/hyst_outline.bib} 

\newpage
\section*{Figures}
    \begin{figure}[h!]
    \centering
 \includegraphics[width=\textwidth]{..//..//Plots/Prunus-Map-raster-plasma.jpeg}
    \caption{This is a map of all the herbaria records of our focal clade. Maybe better in the supplement }
    \label{fig:mappy}
\end{figure}

\begin{figure}[h!]
    \centering
 \includegraphics[width=\textwidth]{..//..//Plots/ord_quants_exmpsps.jpeg}
    \caption{Predicted likelihood that a species would be in flower during each vegetative BBCH phase for five example species in the American plums. Points are the mean likelihood and bar the 95\% uncertainty intervals. Species were classified as hysteranthous if greater than 50\% probability flowering occured in BBCH 0 and BBCH 09 (colors) for each part of the flowering seaion.
  See Fig. \ref{fig:plums} for all species and alternative hysteranthy classification schemes. }
    \label{fig:ordinals}
\end{figure}



\begin{figure}[h!]
    \centering
 \includegraphics[width=.6\textwidth]{..//..//Plots/phylosig2.jpeg}
    \caption{Phylogenetic relationships amoung the American plums and the duration of their flowering perion they are hysterathous. These categorizations are based on ordinal phylogenetics mixed models. Tree topology is from Shaw.}
    \label{fig:phylo2}
\end{figure}


\begin{figure}[h!]
    \centering
 \includegraphics[width=\textwidth]{..//..//Plots/dataplots.jpeg}
    \caption{Relationships between the duration of hysteranthy across the flowering period and environmental and biological traits based on Bayesian phylogenetic mixed models.}
    \label{fig:prunes}
\end{figure}


\begin{figure}[h!]
    \centering
 \includegraphics[width=\textwidth]{..//..//Plots/fullprunus_4manu.jpeg}
    \caption{}
    \label{fig:genus}
\end{figure}

%\begin{figure}[h!]
 %   \centering
 %\includegraphics[width=\textwidth]{..//..//Plots/droughtstuff.jpg}
  %  \caption{Hysteranthy more likely in drought years.}
   % \label{fig:plastic}
%\end{figure}

\end{document}
