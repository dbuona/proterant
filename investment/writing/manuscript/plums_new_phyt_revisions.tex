\documentclass{article}[12pt]
%Required: You must have these
\usepackage{graphicx}
\usepackage{tabularx}
\usepackage{natbib}
\usepackage{caption}
\usepackage{subcaption}
\usepackage{array}
\usepackage{amsmath}
%\usepackage[backend=bibtex]{biblatex}
\setkeys{Gin}{width=0.8\textwidth}
%\setlength{\captionmargin}{30pt}
\setlength{\abovecaptionskip}{10pt}
\setlength{\belowcaptionskip}{10pt}
\topmargin -1.5cm 
\oddsidemargin -0.04cm 
\evensidemargin -0.04cm 
\textwidth 16.59cm
\textheight 23.94cm 
\parskip 7.2pt 
\renewcommand{\baselinestretch}{1.5} 	
\parindent 0pt

\bibliographystyle{refs/styles/newphyto.bst}
\usepackage{xr-hyper}
\usepackage{hyperref}
\externaldocument{suppliment}


\usepackage{lineo}
\linenumbers
\title{Ecological drivers of flower-leaf sequences: aridity and pollination success select for flowering-first in the American Plums}

\author{D.M. Buonaiuto $^{1,2,3,a}$, T.J. Davies $^{4,5}$, S. Collins $^{4}$ \& E.M. Wolkovich$^{2,3,4}$}
\date{}
\usepackage{Sweave}
\begin{document}
\input{plums_new_phyt_revisions-concordance}
\maketitle
\noindent \emph{Author affiliations:}\\
\noindent $^1$Department of Environmental Conservation, University of Massachusetts, Amherst, Massachusetts, USA. ORCID: 0000-0003-4022-2591\\
\noindent 
$^2$Arnold Arboretum of Harvard University, Boston, Massachusetts, USA.\\
$^3$Department of Organismic and Evolutionary Biology, Harvard University, Cambridge, Massachusetts, USA \\
$^4$Forest \& Conservation Sciences, Faculty of Forestry, University of British Columbia, Vancouver, British Columbia, Canada\\
$^5$ Department of Botany, University of British Columbia, Vancouver, British Columbia, Canada\\
$^a$Corresponding author: 617.823.0687; dbuonaiuto@umass.edu\\

\pagebreak

\section*{Summary} %lang for new phyt\
\begin{itemize}
\item Many trees in temperate forests produce flowers before their leaves emerge. This flower-leaf phenological sequence, known as hysteranthy, is generally described as an adaptation for wind-pollination---which does not explain why it is also common in biotically-pollinated taxa. 

\item In this study, we quantified flower-leaf sequence variation in the American plums (\emph{Prunus}, subspp. \emph{Prunus} sect. \emph{Prunocerasus}), a clade of insect-pollinated species, using herbaria specimens and Bayesian hierarchical modeling. With these observations, we tested common hypotheses for the evolution of hysteranthy by modeling the associations between hysteranthy and related traits. To better understand how these phenology-trait associations were sensitive to taxonomic scale and flower-leaf sequence classification, we extended these analyses to the more inclusive genus \emph{Prunus}. 

\item In both groups hysteranthy was associated with aridity and smaller floral displays. These findings indicate that hysteranthy may function to temporally partition hydraulic demand and reduce water stress, or increase pollinator visibility and reduce selective pressure on flower size.

\item Our study provides insights into the function of flower-leaf sequences in biotically-pollinated species.  %paves the way for a continued research agenda regarding 
%function of flower-leaf sequences. 
Our findings that hysteranthy is linked to aridity tolerance and pollination success %increases the urgency for understanding phenological sequences as global change continues to disrupt pollinator services and impact environmental variability. 
highlights the importance of phenological sequences in global change research as climate shifts continue to disrupt pollinator services and impact environmental variability.\\
\end{itemize}

Keywords: Deciduous forests, Flower-leaf sequences, Hysteranthy, Phenology, Plant hydraulics, Pollination, Phylogeny

\pagebreak
\section*{Introduction}
%<<label=numbers, echo=FALSE, results=hide, message=FALSE>>=
%rm(list=ls()) 
%options(stringsAsFactors = FALSE)
%require(brms,quietly = TRUE)
%setwd("~/Documents/git/proterant/investment/Input")
%load("pcerasus.Rda")
%doyeff<-round(fixef(mod.ord.scale.phlyo)[6],digits=2)
%doyints<-round(fixef(mod.ord.scale.phlyo)[6,3:4],digits=3)
\noindent Woody perennials\linelabel{uniq1} are among the small subset of plant types with the unique ability to seasonally begin reproduction prior to vegetative growth. This flowering-first phenological sequence, known as hysteranthy, proteranthy or precocious flowering, is apparent in temperate deciduous forests around the globe \citep{Rathcke_1985}. A number of studies suggest that this flower-leaf sequence is under selection, and that hysteranthy has functional significance \citep{Gougherty2018,Buonaiuto2020,Guo2014}, but the importance of variation in flower-leaf sequences for maintaining fitness may vary across functional types and evolutionary clades within the temperate forest biome. With mounting evidence that anthropogenic climate change is driving shifts in flower-leaf sequences \citep{Ma:2021tf,Wang:2022wt}, expanding our understanding of the adaptive benefit of hysteranthy may be important to forecasting the demography and performance of forest communities.

\noindent The most common, and well-tested explanation for the evolution of hysteranthy in temperate forests is that it is adaptive for wind-pollination, as leafless canopies increase wind speeds for pollen transport and reduce the likelihood of pollen interception by vegetation \citep{Whitehead1969,Niklas1985}. However, this explanation does not address the widespread prevalence of hysteranthy in biotically-pollinated taxa found in temperate regions. This number is not trivial; a recent analysis found that approximately 20\% of the hysteranthy species in Eastern Temperate Forests of North America are biotically-pollinated \citep{Buonaiuto2020}. 

Several alternative hypotheses have been put forward to explain the advantage of hysteranthy in biotically-pollinated species, but they have not been widely evaluated in the literature. Below, we briefly review these hypotheses and their predictions, and then test these predictions using the American plums (\textit{Prunus} subspp. \textit{Prunus} sect. \textit{Prunocerasus}), a widespread clade with high variability in flower-leaf sequences, as a case-study. Our treatment here both clarifies the hypothesized function of flower-leaf sequence variation in biotically-pollinated taxa, and offers insights into how shifting flower-leaf sequences may impact species demography and distributions as climate continues to change.

\subsection*{Hypotheses of hysteranthous flowering in biotically-pollinated taxa}

\underline{Water limitation hypothesis:} In the dry-deciduous tropics of South and Central America, hysteranthy is common \citep{Rathcke_1985,Franklin2016}, and is regarded as an important adaptation to alleviate water stress by partitioning the hydraulic demand of flowers and leaves across the season \citep{Gougherty2018,Franklin2016,Borchert1983,Reich1984}. Under this hypothesis, the function of hysteranthous flowering in temperate regions parallels that in the dry tropics---partitioning hydraulic demand across the season to allow hysteranthous species to tolerate increased aridity. While temperate forests are rarely water-limited in the early season during which flowering and leafing occur \citep{Polgar2011}, there is still considerable variation in water availability in space and time within temperate regions of the globe.  With this hypothesis, we would expect to find hysteranthous taxa in locations that are, on average, drier than their non-hysteranthous relatives.

\underline{Insect visibility hypothesis:} Hysteranthous flowers are visually conspicuous in the landscape. Thus, as in wind-pollinated taxa, hysteranthy in biotically-pollinated taxa may be an adaptation for pollination efficiency as flowering-first species are easier for insect pollinators to locate \citep{Janzen1967}. A challenge\linelabel{caveat1} to evaluating this hypothesis is that correlated selection between flower-leaf sequences and pollinator visibility could have either a positive or negative relationship depending on the pollinator environment. In one scenario hysteranthy may be associated with smaller floral displays, because flowers are not obscured by leaves, they are easier to see, and there is weaker selection for increasing floral display size. However, in environments where plants are more often pollen limited, selection may favor both hysteranthy and increased floral display size in augment attraction to visual\linelabel{caveat2} pollinators.

%\underline{Fruit maturaturion hypothesis:} There are several aspects of reproductive development that suggest hysteranthy is a by-product of developmental constraints related to fruit maturation. Hysteranthy may be common in large fruited species that require lots of time to mature their fruits, or in small, early fruiting species that have evolved dispersal syndromes (wind dispersal, non-dormant seeds) that require dispersal early in the season \citep{Primack1987}. In either case, we should expect fruit size to associate with hysteranthy, although the sign of the correlation differs.

Alternative to these functional hypotheses is the assertion that hysteranthous flowering is simply a by-product of selection for early flowering. Species that flower before their leaves inherently flower early in the season. For\linelabel{null1} example, fruit development or dispersal constraints may drive early flowering \citep{Primack1987} and because spring flower phenology is less constrained by prior phenological events than leaf phenology \citep{Savage2019,Ettinger2018}, this selection for early flowering could incidentally produce the hysteranthous phenological sequence. Here, there is no specific adaptive advantage to hysteranthy;  selection is not operating on the relative timing of flower and leaf emergence, but rather the absolute flowering time alone. Rejection of the above hypotheses might provide support to this null explanation. 

\noindent A significant challenge for robust testing of hysteranthy hypotheses is that most characterizations of flower-leaf phenological sequences are based on expert-opinion verbal descriptions (e.g. ``flowers before leaves" or ``flower before/with leaves"), which make comparisons across taxa, time and space difficult and sensitive to observer bias  \citep[see;][]{Buonaiuto2020}. This problem can be overcome by adopting standardized quantitative measures of plant phenology for observational studies and applying them to historic data records. Herbarium records are an excellent source of data that can be leveraged for quantitative phenological measurements \citep{Willis2017}, but have not been used widely to investigate variability of flower-leaf sequences among and within species.

\noindent The American plums offer potential for a high resolution investigation of drivers of hysteranthous flowering in taxa that are not easily explained by the dominant wind-pollination hypothesis. The 16 species that make up the section are distributed across the temperate zone of North America and, like the genus \textit{Prunus} at large, are all insect-pollinated, yet show pronounced inter-specific variation in flower-leaf sequences. Usefully, species in this section are well represented in herbaria records (Fig. \ref{fig:phylo2}a), making them a tractable group to measure and assess variation in flower-leaf sequences.

\noindent To interrogate the functional hypotheses for hysteranthous flowering described above, we used herbaria records to to quantify both within- and across- species level variation in flower-leaf sequences of the American plums. Then we combined environmental attributes, biological traits and phylogenetic data in statistical models to evaluate whether the observed associations between flower-leaf sequence variation and morphological and environmental traits match the predicted associations of the hysteranthy hypotheses. Finally, we compared our findings in this clade to patterns observed in larger genus \emph{Prunus} to better understand whether these phenology-trait associations were sensitive to taxonomic scale and flower-leaf sequence classification.


\section*{Materials and Methods}
\subsection*{Quantifying flower-leaf sequence variation}  

We obtained digital herbarium specimens for all members of the section \textit{Prunocerasus} from the Consortium of Midwest Herbaria (CMH) Database \citep{CMH}. \linelabel{range1}Specimen\linelabel{range1} collection dates ranged from 1844-2020, with the majority collected between\linelabel{range2} 1950-2000. To quantify flower-leaf sequence variation within and across species we randomly sampled 200 specimens for each species and scored the phenological development of flowers and leaves using a modified BBCH scale for woody plants \citep{Finn2007}, which\linelabel{bbch1} is designed to evaluate vegetative and reproductive phenological progress using a standardized quantitative index. For\linelabel{samps1} species with less than 200 specimens in the collection, we included all available specimens. In total, we evaluated the phenology of 2521 specimens, but only specimens with visible flowers were included in this analysis. We also assessed and removed outliers of flowering observations visually, and by excluding observations that were beyond three standard deviations of the median flowering time for each species (n=9). Our final analyses included 1000 specimens (see \ref{tab:samps} for number of\linelabel{samps2} observations/species). We reconstructed the phylogenetic relationships among species in this group based on the tree topology in \citet{Shaw:2004aa}. We inferred branch lengths following the method of \citet{Granfen1989} in which node heights are estimated in proportion to number of subtending taxa using the R package ``ape" \citep{Paradis2019}.

To quantify flower-leaf sequence variation, we fit an ordinal, hierarchical, Bayesian phylogenetic mixed model \citep{Garamszegi2014} to assess the likelihood an individual would be at any given vegetative BBCH phase while flowering. Our model predicted leaf stage ($y_i$, ordinal, with six categories) as a function of species and additional phylogenetic effects. Because hysteranthy co-varies with flowering time (i.e., flowering first species will generally flower earlier than other species, on average), and collection\linelabel{bias1} dates are not evenly distributed across the flowering season (see Fig. \ref{fig:bias}), we included day of observation as an additional predictor. Additionally, because\linelabel{hinge1} it is possible that climate change has affected the interval between flowering and leafout over the course of our time series, we included the year of collection of each specimen as a co-variate. Because the concern for including this co-variate was related to shifting baselines due to climate change, we parameterized \emph{year} as hinge variable, using 1980 as a break point following standard conventions for modeling the effects of climate\linelabel{hinge2} change \citep{IPCC2013,Buonaiuto2020,Kharouba2018}.

The model is written below:\\

y_i = \left\{ \begin{array}{lll}
1 & if & z_i < 0\\ 
2 & if & z_i  \in (0,c_{2})\\ 
3 & if & z_i \in (c_{2},c_{3})\\ 
4 & if & z_i \in (c_{3},c_{4})\\ 
5 & if & z_i \in (c_{4},c_{5})\\ 
6 & if & z_i > c_{5}\\ 
\end{array}\right.
\\


$z_i  &= \alpha+ \alpha_{phylo}+ \alpha_{sp}+ \beta_{\text{day of year[sp]}}*X_{\text{day of year}}+
\beta_{\text{year}}*X_{\text{year}}+
\epsilon_i$\\
  
   $\epsilon_i & \sim logistic(0,1)$ \\ 
   
where $y_i$ is the ordinal outcome (leaf stage; as 1,2,...6 categories). $c_{2...5}$ are the estimated cutpoints between leaf stages on the logit scale. $z_i$ is the linear component of the underlying latent variable model.  
% $P(Y \leq j))$ is the probability of $Y$ less than or equal to a category $j&=1...j&-1$. 
$\alpha$ describes an intercept for each category [1,2,...6], while slope ($\beta_{\text{day of year}}$) is constant across cutpoints, but varies among $species$. 
  
  \noindent The influence of the phylogeny ($\alpha_{phylo}$) was modeled as:\\
  %\alpha_{phylo} & \sim N(\mu_{\alpha}, COR[\sigma^2_{phylo}]) \\
  \alpha_{phylo} & \sim N(0, COR[\sigma^2_{phylo}]) \\
  
  \noindent The $\alpha$ for species effects independent of the phylogeny was modeled as:\\
  %\alpha_{sp} & \sim N(\mu_{\alpha}, \sigma^2_{species}) \\
 \alpha_{sp} & \sim N(0, \sigma^2_{species}) \\

We used\linelabel{conty1} our model to predict the likelihood each species would be observed at a given vegetative BBCH stage during flowering for each day of the flowering period of each species. For each day of the flowering season, we summed the predicted likelihood that species would be at BBCH 0  (``bud closed"), BBCH 07/09 (``bud break") or BBCH 11 (``start of leaf unfolding) vs. BBCH 15 (``leaf unfolding"),BBCH 17(``most leaves unfolded"), BBCH 19 (``leaf expansion complete") to quantify the likelihood as species would be be hysteranthous or non-hysteranthy respectively on each day of the the season. We used these estimate to  developed a flower-leaf sequence index by summing the likelihood of hysteranthy vs. non-hysteranthy across the flowering period of each species, with 0 being never hysteranthous and 1 being always hysteranthous\linelabel{conty2}. ADD USING POSTERIOR DRAWS

To\linelabel{noday} better understand how within season dynamics affected our inference, we also repeated this procedure without including day of season as a predictor. This version of the model did not substantially change the our inference about the relationships between flower-leaf sequence variation and the trait representing the main hysteranthy hypotheses, which is available in our Supporting\linelabel{noday2} Information (Fig. \ref{})
%We then developed a flower-leaf sequence index, by summing over the full flowering season. In each seasonal quantile, species received a ``1" if more than 50\% of their probability distribution occurred at the two earliest stages of vegetative phenology---BBCH 0 (``bud closed") and BBCH 09 (``bud break")---and a ``0" if not. We summed these values across the season, generating an index from 0 (never hysteranthous) to 4 (hysteranthous through late season (Q75)), where 1&= hysteranthous at start of season, 2&= hysteranthous through early season  (Q25) and 3 &= hysteranthous through mid season (Q50). We also used two alternative indexing schemes ($>$25\% of the probability distribution occurred at BBCH 0 and $>$40\% of the probability distribution occurred at BBCH 0 and BBCH 09) to make sure our result were robust across multiple cutoffs.

\subsection*{Evaluating hysteranthy hypotheses}

To test the hypotheses of hysteranthy, we obtained data on petal length and fruit diameter directly from herbarium specimens. To assess aridity tolerance, we computed the average Palmer\linelabel{pdsi1} Modified Drought Index score (hereafter: PMDI) from 1900-2017, obtained from the \citet{NOAA}, for every \textit{Prunocerasus} specimen in the database(n=2305). PMDI is a standardize index that integrates temperature and precipitation data to estimate relative dryness in time and\linelabel{pdsi2} space \citep{Heim:2002uw}. For any specimens that lacked accurate geo-location information, we extracted PMDI values at the county centroid of the herbaria specimen. 

\noindent For our morphological measurements, we sampled an additional 321 specimens and measured the petal length of up to 10 randomly selected petals per specimen (n=2757) using ImageJ image processing software.% We also used ImageJ to measure the diameter of fruits on an additional 316 specimens, measuring up to 5 fruit per specimen (n=224).

Because\linelabel{just1} our all of our measurements were on different individuals, with different sample sizes we executed two different modeling approaches in order to test the relationship between flower-leaf sequence index scores, aridity tolerance and floral\linelabel{just2} displays.

First we computed species-levels means of PDSI and petal length and used a Beta regression to evaluate the relationship between flower-leaf sequences, PDSI, petal length and their interaction.

The model structure is: 

\textbf{Write it}

The advantage of this approach is that this model structure allowed us to assess the additive and interactive effects of PDSI and petal size on flower-leaf sequences. By using means trait values, this approach cannot incorporate within species variation in these trait/environmental predictors or account for their phylogenetic structure. Because of this we as modeled the relationship between flower-leaf sequences index values and PDSI, and between flower-leaf sequences index values and petal size separately. Because single-predictor regressions can be formulated with either variable as the dependent one (say better and cite),  this allowed us to both account for variation in within species PDSI and petal lengths and account for the phylogenetic structure of these variables as well. (Should everything below go to the supplement?)

 In these models, we modeled species and phylogeny as above. 

The model structure is: 

  $y_{trait} &= \alpha +\alpha_{sp} +\alpha_{phylo} + \beta_{hyst.index}*X_{hyst.index} + \epsilon$\\
  
  \epsilon & \sim N(0,\sigma^2_y) \\ %Check this
  
  where $y_{trait}$ is observed trait values (PDSI or petal length), and the slope $\beta_{\text{hyst.index}$ describes the relationship between extended hysteranthy (higher hysteranthy index value) and the trait of interest. $\alpha$ describes a grand intercept, and $\alpha_{sp}$ and $\alpha_{phylo}$ describe the species and phylogenetic effects respectively. 
  
%We also ran each model using our two alternative flower-leaf sequence indexing approaches to make sure our results were robust to choice of index. Though these alternative classification schemes did change the hysteranthy index score for some species (Fig. \ref{fig:plums}), they did not substantially impact the inference from our models (see Tab. \ref{tab:modput} for comparisons).
  
\subsection*{Hysteranthy in the larger genus \textit{Prunus}}

%emw -- check edits to first sentence
To better understand how the patterns we identified in \textit{Pruncerasus} scaled to a larger more inclusive group and across coarser taxonomic resolution and flower-leaf sequence classification we also evaluated the relationship between hysteranthous flowering and hypothesis-related traits in all \textit{Prunus} species native to, or established in, North America. For this analysis, we obtained categorical descriptions of flower-leaf sequences and mean estimates of the number of flowers per inflorescence as a proxy for floral investment from the Flora of North America \citep{Rohrer_1993}.  We extracted PDSI values for all herbaria observation of those species in the Consortium of Midwest Herbaria database (n=23,272) as described above.

 
To account for the influence of evolutionary relationships among species, we reconstructed the phylogenetic relationships in the genus based on the tree topology in \citet{Chin:2014wu}. As above, we computed branch lengths with the R package ``ape" \citep{Paradis2019}. 

We standardized the units of all predictors through z-scoring \citep{Gelman2007} to make their effect size estimates for the following model structure directly comparable to each other:

y_i = \left\{ \begin{array}{lll}
1 & if & z_i < 0\\ 
2 & if & z_i  \in (0,c_{2})\\ 
3 & if & z_i \in (c_{2},c_{3})\\ 
4 & if & z_i> c_{3}\\ 
\end{array}\right.
\\


z_i &= \alpha+ \alpha_{phylo}+ \beta_{PDSI}*X_{PDSI}+\beta_{\text{floral investment}}*X_{flowers/inflorescence}+\beta{PDSI x floral investment}*X_{PDSI x floral investment}+\epsilon_i$\\ % is this right?
  
   \epsilon_i & \sim logistic(0,1)) \\ 
  
where $y_i$ is the ordinal outcome of flower-leaf sequence category (``flowers before leaves", ``flowers before/with leaves", ``flowers with leaves" and ``flowers after leaves") and $c_{2...3}$ are the estimated cutpoints between categories on the logit scale. As above, $z_i$ is the linear component of the underlying latent variable model. $\alpha$ describes a grand intercept, and we modeled the influence of phylogeny ($\alpha_{phylo}$) as above. Note\linelabel{catergories} that this model includes four ordinal categories while our model of the American Plums clade included 6, due to the underlying structure of the data.


\subsection*{Model runs} 
 
We fit models in the R package ``brms" \citep{Burkner2018} using weakly informative priors, and four chains.
For the ``Quantifying flower-leaf sequence variation" and ``Evaluating hysteranthy hypotheses" we ran the models with a warm-up of 3000, and 3500 iterations, and 4000, and 4500 sampling iterations respectively, for a total of 4000 sampling iterations across all chains. For the ``Hysteranthy in the larger genus \textit{Prunus}" model, we used a warm up of 6,000 iterations and 8,000 sampling iterations for a total of 8,000 sampling iterations to maximize the effective sampling size. Model fits was assessed with  $\hat{R}$ <1.01, high effective sample sizes and no divergent transitions. We provide mean estimates and 50\% uncertainty intervals in the text with alternate intervals in figures and the Supporting Information. 

\section*{Results}
\subsection*{Quantifying flower leaf sequences in the American plums}
Across all species of the American Plums, day of observation had a strong association with the hysteranthy likelihood (Fig. \ref{fig:ordinals},a). Year of observation did not substantially impact the likelihood of hysteranthy in this taxonomic group (Fig. \ref{fig:ordinals},a).

We found substantial inter-specific differences in flower-leaf sequences within the American plums (Fig. \ref{fig:ordinals}. b), with likelihood of hysteranthy of hysteranthy across the season ranging from 0.16 (\emph{P. subcordata}) to .85 (\emph{P. mexicana}). Several species (\emph{P. mexicana}, \textit{P. umbellata},\textit{P. angustifolia},\textit{P. maritima} and \textit{P. gracilis}) were most likely to be hysteranthous for all---or most---of their flower period, while others (\textit{P. americana},\textit{P. munsoniana}, \textit{P. alleghaniensis}, \textit{P. nigra},\textit{P. hortulana},\textit{P. texana},\textit{P. rivularis}) hysteranthous flowering was only likely in the early part of their flowering session. One species, \emph{P. subcordata}, was unlikely to be hysteranthous at any point in its flowering period (Fig. \ref{fig:ordinals}. b).

\subsection*{Associations between hysteranthy and environmental and morphological traits}
In the American plums clade, increased likelihood of hysteranthy was negatively associated with PDSI and petal length  without a substantial interaction between them (put beta here, Fig. \label{fig:prunes}), indicating that hysteranthy species are more likely be be found have smaller flower and be found in drier localities.

For the larger genus \emph{Prunus}, there was a negative association between hysteranthy and PDSI and number of flowers per inflorescence, as well as a substantial supra-additive interaction between them (betas,  Fig. \ref{fig:genus}). 

\section*{Discussion}
In this\linelabel{style1} study we show that hysteranthous flowering can be linked to both aridity tolerance and pollination success.% through the predictions of the water limitation and insect visibility hypotheses, clarifying something important

%highlighting the urgency for advancing our understanding of phenological sequences as human-caused global change continues to disrupt pollinator services and impact environmental variability. 

%\subsection*{Hysteranthy hypotheses} % emw -- good first sentence below, flagging for future cover letters
Using North American \textit{Prunus} species as a case study, our analyses indicate that flower-leaf sequences are under selection by biological and environmental drivers, and that variation in these patterns across species may reflect adaptive tradeoffs between\linelabel{tradeoffz} the timing of investment in reproduction relative to the timing of resumption of carbon gain through leafout, and these other aspects of plant performance. We found that hysteranthous flowering is associated with smaller floral displays and increased aridity in both the American plums, and more broadly across \emph{Prunus} native or established in North America. The relationships between hysteranthy and aridity, and hysteranthy and floral display size support the predictions of the water limitation hypothesis and the insect visibility\linelabel{style2} hypothesis, respectively. 

While, above in our introduction to the insect visibility hypothesis, we highlighted that floral display size could either be positively or negatively associated with hysteranthy depending on the pollination environment, we found that for both the taxonomic scale we investigated, hysteranthy was associated with smaller flowers. This suggests that it is possible that the increased visibility of hysteranthous flower may reduce selection pressure on flower display size, a finding that supports exisiting evidence form both comparative anatomy studies in plants \citep{Gunatilleke1984} and studies about pollinator foraging behavior \citep{Forrest:2009aa,Rivest:2017aa}.


The supra-additive interactions terms between PDSI and floral display size we found highlight that the water limitation hypothesis and insect visibility hypothesis are not mutually exclusive, and could be related. Selection on floral size represents a classic evolutionary tradeoff where larger floral displays may generally be more effective for attracting pollinators but demand more resources, including water, to maintain turgor and reproductive function than smaller ones \citep{Galen:1999vr,Lambrecht:2007ur}. With this trade-off, reproductive displays are often small in harsher environments \citep{Teixido:2016aa,Lambrecht:2013aa}, and hysteranthy could represent a compensatory mechanism that both reduces hydraulic demand while increasing pollination efficiency in these environments.

It is not\linelabel{floflo1} surprising that the coefficient estimates for floral display size and their interaction term with PDSI were higher in the larger genus \emph{Prunus} analyses than that of the American Plums. All species in the American plums clade have solitary flowers, and the variation in size was highly constrained. By, contrast, our analysis of the larger genus \emph{emph} included species that range from having solitary flowers to ~100 flowers per inflorescence, representing substantial more variation in both floral investment and in hydraulic demand. This suggests that the tradeoff between the correlative selection between flower-leaf sequences and the floral traits may be more pronounced at coarser taxonomic resolutions, when underlying trait variation is\linelabel{floflo2} greater.

Despite these differences among taxonomic resolutions, aridity was associated with hysteranthy in both groups. Studies that have compared the transpiration rates among flowers and leaves provide insights to the potential importance of this seasonal partitioning for maintaining water status. Measurements of water movement (transpiration rates, sap flow, hydraulic conductivity) to flowers range from 20\%-60\% of that of leaves under comparable conditions \citep{Whiley:1988uf,Roddy:2012wn,Liu:2017wg,McMann:2022ww}. This level of additional hydraulic demand can drive loss of stomatal conductance and decrease photosynthetic rates \citep{Galen:1999vr}.
 
 % emw -- does PDSI have a unit? ... also check my edits
Despite this evidence that hysteranthy can reduced hydraulic demand in dry environments, hysteranthous species in the American plum clade are not found in extremely arid locations (PDSI values typically range from -4 to 4, however our analyses found mean values ranging from -0.5 to 0.2 for species classified as hysteranthous through mid-season or through late season). This contrasts with hysteranthous species in the dry tropics where this phenological syndrome allows them to tolerate more extreme aridity \citep{Franklin2016}. But the flower-leaf sequences of the hysteranthous species in our study were markedly different from patterns of hysteranthy in these dry-tropics where the water limitation hypothesis was initially proposed. While flowering can precede leafout by as much several weeks for species in the American plums, the process of fruit development, which is also water intensive, occurs when leaves are present. By contrast, in the dry tropics hysteranthous flowering is initiated at the time of leaf drop \citep{Borchert1983,Franklin2016}; thus, the full reproductive cycle occurs in the leafless period. The comparatively small window of leafless reproductive development in our temperate clade, may in part, explain why the association we observed between hysteranthy and aridity in our study was relatively weak with high residual variance. Our results suggest that hysteranthy may allow temperate species to occupy marginally drier environments than non-hysteranthous species, but may not facilitate species' persistence under extreme aridity. 

\subsection*{Inter-and intra-specific variation in flower-leaf sequences} %emw -- also great first sentence below
We developed a novel approach to assessing flower-leaf sequences that scales from quantitative, individual-level observations to species-level characterizations that were based on empirical likelihood estimates. With this approach, we were able to---for the first time---quantitatively assess intermediate cases of hysteranthy (such as those that are typically described as ``flowers before/with leaves"). Previous studies of hysteranthous flowering have either excluded these cases from their analyses  \citep[e.g.;][]{Gougherty2018} or binned them with the well defined cases \citep[e.g.;][]{Buonaiuto2020}. We found that many of American plum species expressed this intermediate flower-leaf sequence, and our classifications broadly matched previous species-level analyses in this group by \citet{Shaw:2004aa}. By estimating the likelihood of hysteranthy across the growing season with Bayesian methods, our approach identified substantial differences in flower-leaf sequences among these intermediate cases (Fig. \ref{fig:ordinals}, Fig. \ref{fig:plums}), which allowed us to assess the trait associations with this phenotype.

Our quantitative analysis of the American plums clade revealed that flower-leaf sequences---often described as a species-level trait---are highly variable within species (Fig. \ref{fig:ordinals}, Fig. \ref{fig:plums}). For almost all members of the clade, the day of phenological observation was a strong predictor of the likelihood that flowers would be visible before the emergence of leaves. In many cases there was high likelihood that individuals of a species may be observed at different vegetative stages during flowering (Fig. \ref{fig:ordinals}, \ref{fig:plums}). This variation could either suggest high levels of local adaptation in flower-leaf sequences or, alternatively, high levels of plasticity through which flower-leaf sequences respond to interannual variation in environmental conditions. Because\linelabel{lim1} our study was based on herbaria records collected on different individuals across space and time without repeat sampling, we were not able to robustly how much flower-leaf sequences vary within vs. among species. However this would be an important next step for understanding how the environment and species interactions have shaped these phenological\linelabel{lim2} patterns.

%For example---in a given population---flower-leaf sequences may respond to interannual variation in precipitation with increased temporal separation between flowers and leaves in drier years. While our data did not have the temporal resolution to address this question, the high levels of within-species variation we observed raise important questions about environmental drivers of flower-leaf sequences operating on both the macro-evolutionary scale we investigated here and on individual physiological responses to environmental change. 

By scoring these individual, quantitative observations as ordinal response categories with our hysteranthy index, we were able to contrast our findings to those derived from categorical, species-level characterizations based on expert opinion. The coherence between our individual based observational approach for the American plum clade and the top-down, categorical classification across \emph{Prunus} is an encouraging demonstration that the expert opinion-based data can still offer useful insights into the drivers of hysteranthous flowering when higher-resolution data is not available. 

\subsection*{Future directions}

In this study, we focused on a well-studied, and economically important clade of morphologically similar species, that allowed us to control for unmeasured biological variation. Our case-study provides a road map for evaluating the role of hysteranthy in temperate biotically-pollinated plant taxa (other groups with high interspecific flower-leaf sequence variation include \emph{Magnolia}, \emph{Rhododendron}, \emph{Acer} and \emph{Cornus}), and more broadly across taxa and biomes.

Combining the observational approach with novel experiments could further advance our collective understanding of the adaptive significance of flower-leaf sequences. To test the water-limitation hypothesis, researchers could plant sister-taxa with contrasting flower-leaf sequences in common environments across a gradient of aridity, and evaluate their performance. To test the insect visibility hypothesis, researchers should also consider hysteranthy---and phenology in general---in the more general framework of tradeoffs in pollination biology. The tradeoff between phenology and pollination investment should not only consider flower size, but also the number of flowers, nectar and pollen reward investment, volatiles between related hysteranthous and non-hysteranthous taxa. Findings that hysteranthous species invest fewer resources into these other pollinator attraction traits than non-hysteranthous relatives would support the insect visibility hypothesis. For a simple experiment to test the pollinator visibility hypothesis, researchers could force hysteranthy/non-hysteranthy phenotypes for the same genotype using environmental cues, and systematically release pollinators to observe their preferences, search times and foraging behavior. If pollinators are more readily drawn to the hysteranthous individuals, it would suggest that hysteranthy may be an adaptive trait for pollinator attraction. %emw -- I don't love the phrasing of 'The expectation here is that hysteranthous species would invest fewer resources into these other pollinator attraction traits than non-hysteranthous relatives.' .... but failed to fix it quickly. Maybe something like 'Support for [the insect visibility hypothesis?] hypothesis would find that hysteranthous species would invest fewer resources into these other pollinator attraction traits than non-hysteranthous relatives.' Or 'Findings that hysteranthous species invest fewer resources into these other pollinator attraction traits than non-hysteranthous relatives would support ....'

With a better mechanistic understanding of the relationship between flower-leaf sequences and ecological performance in hand, researchers could then use experiments to assess how differences in floral and leaf physiological responses to temperature variation may alter the adaptive benefits of flower-leaf sequences with climate change. The measurement and modeling approaches we developed in our observational study can be readily implemented to analyze data from such experimental settings, presenting an important opportunity to unite observations of broad ecological patterns with targeted experimental manipulations to better understand both the evolutionary past and ecological future of flower-leaf sequences.


\section*{Competing Interests:}
The authors declare no conflict of interest.

\section*{Author contributions}
DMB, and EMW conceived of the manuscript; DMB and SC collected the data; DMB led the statistical analyses with TJD and EMW; DMB led the writing of the manuscript. All authors contributed to writing and gave approval for the submission.

\section*{Data Availability}
The phenology and trait data collected for this study will be made available and archived at KNB: The Knowledge Network for Biocomplexity (https://knb.ecoinformatics.org/) at the time of publication.


\bibliography{refs/hyst_outline.bib} 

\newpage
\section*{Figures}


\begin{figure}[h!]
  \centering
 \includegraphics[width=\textwidth]{..//..//Plots/fig1_new.jpg}
    \caption{Geographic distribution and taxonomic relationships among the American plums. a) Maps the localities of all the herbaria records used in this study. b) Depicts phylogenetic relationships among the American plums and the duration of their flowering period they are hysteranthous. These categorizations are based on ordinal phylogenetic mixed models. Tree topology is from \citet{Shaw:2004aa}}
    \label{fig:phylo2}
\end{figure}



\begin{figure}[h!]
    \centering
 \includegraphics[width=\textwidth]{..//..//Plots/whatReviwerswant/sps_preds.jpeg}
    \caption{Predicted likelihood that a species would be in flower during each vegetative BBCH phase for five example species in the American plums. Points are the mean likelihood while bars represent 95\% uncertainty intervals. Species were classified as hysteranthous if greater than 50\% probability flowering occurred in BBCH 0 and BBCH 09 (colors) for each part of the flowering season.
  See Fig. \ref{fig:plums} for all species and alternative hysteranthy classification schemes. }
    \label{fig:ordinals}
\end{figure}




\begin{figure}[h!]
    \centering
 \includegraphics[width=\textwidth]{..//..//Plots/whatReviwerswant/hypoth_preds.jpeg}
    \caption{Relationships between the duration of hysteranthy across the flowering period and environmental and biological traits based on Bayesian phylogenetic mixed models. a) b) and c) depict the relationships between the duration of hysteranthy and mean PDSI, fruit diameter, and petal length respectively. Solid lines indicate the mean posterior estimate and shaded areas 4000 draws from the posterior distrubtion as a display of uncertainty. The points are jittered along the x-axis only for visibility.}
    \label{fig:prunes}
\end{figure}


\begin{figure}[h!]
    \centering
 \includegraphics[width=\textwidth]{..//..//Plots/whatReviwerswant/fullprunus_4manu.jpeg} %emwmar9 --mv fruit to supp?
    \caption{Relationships between the likelihood of hysteranthy and environmental and biological traits in the genus \emph{Prunus} based on Bayesian phylogenetic mixed models. Panel a) shows the estimated effect size of each predictor with negative values indicating an increased likelihood of hysteranthy. Points indicate the mean posterior estimate for each predictor, and thick and thin bars the 50\% and 97.5\% uncertainty intervals respectively. We also show the full posterior distribution as an additional more of uncertainty, Panel b), c) and d) show the marginal effect of mean PDSI, inflorescence size and fruit size respectively, on the likelihood that of each flower-leaf sequence category. Solid lines indicate the mean likelihood and shaded areas the 50\% uncertainty intervals.} %emw -- Units? Even if you add to methods, could be good to mention in captions and in first instance in results text
    \label{fig:genus}
\end{figure}


\end{document}
