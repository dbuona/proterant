\documentclass[11pt]{article}
\renewcommand{\baselinestretch}{1.8}
\usepackage{textcomp}
\usepackage{fontenc}
\usepackage{graphicx}
\usepackage{caption} % for Fig. captions
\usepackage{gensymb} % for \degree
\usepackage{placeins} % for \images
\usepackage[margin=1in]{geometry} % to set margins
\usepackage{setspace}
\usepackage{lineno}
%\usepackage{cite}
\usepackage{amssymb} % for math symbols
\usepackage{amsmath} % for aligning equations
\usepackage[sort&compress]{natbib}
\bibliographystyle{..//refs/styles/besjournals.bst}
%\usepackage{xr-hyper}

% line numbers for letter
% line numbers for letter


\title{GSAS Merit Fellowship: Statement of Purpose}
\date{November 2021}
\author{Daniel Buonaiuto\\
Dept. of Organismic and Evolutionary Biology}
\begin{document}
\maketitle
\section*{Background}
In a typical model of plant development, vegetative growth precedes reproduction. However, for as many as 30\% of trees and shrubs of eastern North America, it is not the green tips of leaves that mark the start of the growing season, but the colors of flowers \citep{Buonaiuto2020}. Why do so many species in these regions produce their flowers before their leaves; during the time of year when risk of frost damage is the highest \citep{Augspurger:2013aa} and their energy reserves are at their lowest \citep{Primack1987}? \\

Several hypotheses have been put forth in the literature to explain this risky behavior. Flowering during the leafless season may allow for efficient pollen transfer in wind-pollinated species \citep{Rathcke_1985} or reduce drought stress by partitioning the evaporative demand of flowers and leaves across the seasons \citep{Gougherty2018}. Flowering-first may also simply be a byproduct of extremely early flowering \citep{Primack1987} and variation in FLS patterns among species could also be more strongly determined be developmental or evolutionary constraints than ecological function \citep{Gougherty2018,Diggle1995}.\\

In a previous project my colleagues and I used statistical modeling to compare and evaluate these hypotheses \citep{Buonaiuto2020}. Across all of our models, we consistently found a strong association between wind-pollination and the flowering-first FLS. We also found that early flowering and phylogeny, the shared evolutionary lineages among taxa, strongly influence this trait. This support for multiple hypotheses suggests that flowering-first FLS’s may have evolved multiple times under different environmental conditions and raises new questions about the functional significance of this trait.\\

In particular, the strong support for the wind-pollination hypothesis from our models offers little information about the functional significance of flowering-first in the many insect-pollinated species share this FLS. To fill in this gap, I intend to undertake a new project that will explore the functional ecology of flowering-first FLSs in insect-pollinated taxa by exploring associations between FLSs, species distributions and functional traits using a large data set of herbaria specimens. I present the details of this project below.

\section*{Project Proposal: The functional ecology of flower-leaf sequences in the American plums}
%A more nuanced understanding about the function of FLS variation may result from pattern deconstruction (i.e. grouping of species according to subclades or other commonalities; \citep{Terribile2009)}. 
The American plums (\textit{Prunus} subsp. \textit{Prunus} sect. \textit{Prunocerasus}) are an ideal group to use to address my research question. The section consists of 16 species that are distributed across North America and vary in their FLSs \citep{Shaw:2004aa}. The Harvard herbaria houses 3,134 digitized specimens of these species and the Consortium of Midwest Herbaria collection contains an additional 13,869. Leveraging these 17000+ specimens, I will test the major hypotheses of the function of FLS variation in insect-pollinated taxa by testing for correlations between FLS patterns and the functional traits and ecological requirements of these species.
\subsection*{Part I: Quantification of FLS variation in \textit{Prunocerasus}}
A major challenge to study FLS variation is that most data sources describe FLS patterns using broad verbal descriptions ( e.g. ``flowers before leaves”). These categories are imprecise and, unsurprisingly, different sources classify the same species differently \citep{Buonaiuto2020}. While it’s possible that these differences could reflect temporal or geographic variability in FLSs, they could equally be the product of observer bias. It is important to apply a more rigorous quantitative method for describing FLSs.\\

To characterize FLS patterns among \textit{Prunocerasus} species, I will score the vegetative and flowering stages of 500-1000 herbarium specimens per species using the BBCH scale \citep{Finn2007}, a standardized method for robustly evaluating the timing of seasonal events in plants. From these data, I will be able to use Bayesian generalized linear models to predict a likelihood that a given species flowers before its leaves emerge, providing the first quantitative estimates of FLS variation for these species, and demonstrating a robust method to do so for other taxa. Using this method, I will also be able to quantify variation in FLS patterns within species and test for any temporal or environmental covariates that may drive this variation.

\subsection*{Part II: Assessing FLS hypotheses}
Each of the dominant hypotheses of the functional ecology of FLS variation makes a prediction about the environmental characteristics or biological traits that should be associated with flowering-first species.\\

The \textbf{water limitation hypothesis} \citep{Gougherty2018} predicts that the geographic ranges of leafing-first species will be more limited by water availability than flowering-first species. To test this hypothesis, I have obtained measurements of the historic drought conditions (based on the Palmer Drought Severity Index \citep{Dai:2004aa}) at the collection site of each herbarium specimen and will test for differences in aridity tolerance across geographic ranges of predominantly flowering-first and leafing-first taxa.\\

The \textbf{early flowering hypothesis} suggests that flowering-first is an adaptation that allows for extended fruit development time \citep{Primack1987}, and predicts that flowering-first species will have larger fruits than leafing-first species. To test this hypothesis, I will measure the diameter of fruits present on herbarium specimens and test for differences between predominantly flowering-first and leafing-first taxa.\\

The \textbf{insect-visibility hypothesis} suggest that flowering-first species are easier for insect pollinators to locate \citep{Janzen1967} and predicts that flower displays will differ in size between flowering-first and leafing-first species. To test this hypothesis, I will measure the corolla diameter of 5-10 flowers from 50-100 herbarium specimens/species and compare the flower size between predominantly flowering-first and leafing-first taxa.

\section*{Scope and implications of the project}
I have tested the methods described above in a pilot study of a subset of \textit{Prunocersus} herbarium  specimens (10-15 per species) and believe the methods are robust for both quantifying FLS patterns among species and testing the FLS hypotheses through trait associations. This pilot has also shown that with a focused effort on data collection, the majority of this work could be completed in a semester. The support of the GSAS merit fellowship would allow me to prioritize this work and make this concetrated effort required to successfully complete this project possible.\\

Upon completion, this project will fill in a major gap in the basic understanding of the functional significance of flower-leaf sequence variation in deciduous woody plants. Additionally, because FLS patterns are shifting with climate change \citep{Buonaiuto2020, Augspurger:2020aa}, improving our understanding of the function of FLS patterns is an important step for predicting the scope and implications of FLS shifts in the future.

\bibliography{..//..//sub_projs/refs/hyst_outline.bib}

\end{document}